\documentclass[oneside, 10pt]{amsart}
\usepackage{amscd, amsmath, amssymb, amsthm, amsfonts, amstext, verbatim, mathtools, xfrac, microtype, nameref, thmtools}
\usepackage[breaklinks=true]{hyperref}
\usepackage[hyperref=true, backend=bibtex, firstinits=true, citestyle=numeric-comp, sortlocale=en_US, url=false, doi=false, eprint=true, maxbibnames=4]{biblatex}
\usepackage[capitalize]{cleveref}
\usepackage[matrix,arrow,curve]{xy}
\usepackage{tikz}
\usepackage{enumitem}
\usepackage[notref,notcite]{showkeys}

\addbibresource{paper.bib}
\renewbibmacro*{volume+number+eid}{\ifentrytype{article}{\- \iffieldundef{volume}{}{Vol.~\printfield{volume},}\iffieldundef{number}{}{ No.~\printfield{number},}}}
\renewbibmacro{in:}{\ifentrytype{article}{}{\printtext{\bibstring{in}\intitlepunct}}}
\newbibmacro{string+doi}[1]{\iffieldundef{doi}{\iffieldundef{url}{#1}{\href{\thefield{url}}{#1}}}{\href{http://dx.doi.org/\thefield{doi}}{#1}}}
\DeclareFieldFormat[article, inproceedings, inbook, book, thesis]{title}{\usebibmacro{string+doi}{\mkbibquote{#1}}}
\renewcommand*{\bibfont}{\footnotesize}
\DeclareMathOperator{\St}{St}
\DeclareMathOperator{\E}{E}
\DeclareMathOperator{\T}{T}
\DeclareMathOperator{\Hom}{Hom}
\DeclareMathOperator{\Um}{Um}
\DeclareMathOperator{\colim}{colim}
\DeclareMathOperator{\GL}{GL}
\DeclareMathOperator{\SL}{SL}
\DeclareMathOperator{\Aut}{Aut}
\DeclareMathOperator{\K}{K}
\DeclareMathOperator{\rk}{rk}
\DeclareMathOperator{\Sym}{Sym}
\newcommand{\catname}[1]{{\normalfont\textbf{#1}}}

\newcommand{\rA}{\mathsf{A}}
\newcommand{\rB}{\mathsf{B}}
\newcommand{\rC}{\mathsf{C}}
\newcommand{\rD}{\mathsf{D}}
\newcommand{\rE}{\mathsf{E}}
\newcommand{\rF}{\mathsf{F}}
\newcommand{\rG}{\mathsf{G}}

\newcommand{\ZZ}{\mathbb{Z}}
\newcommand{\UU}{\hat{\mathrm{U}}}      % Unipotent subgroup on the Steinberg level
\newcommand{\StB}{\hat{\mathrm{B}}}     % Borel subgroup on the Steinberg level
\newcommand{\StH}{\hat{\mathrm{H}}}     % Torus on the Steinberg level
\newcommand{\StW}{\hat{\mathrm{W}}}     % weyl on the Steinberg level
\newcommand{\XX}{\mathcal{X}}           % (subset of) Steinberg generators
\newcommand{\RR}[1]{\mathcal{R}_{#1}}   % Steinberg relations
\newcommand{\ext}[1]{\mu_{#1}}           % Map from St(Ф, R, I) --> St(Ф, R)

\newtheorem{thm}{Theorem}
\Crefname{thm}{Theorem}{Theorems}
\numberwithin{equation}{section}

\newtheorem{lemma}{Lemma}
\numberwithin{lemma}{section}
\Crefname{lemma}{Lemma}{Lemmas}
\newlist{lemlist}{enumerate}{1} \setlist[lemlist]{label={\rm(\arabic{lemlisti})}, ref=\thelemma.(\arabic{lemlisti}),noitemsep} \Crefname{lemlisti}{Lemma}{Lemma}

\newtheorem{cor}[lemma]{Corollary}
\Crefname{cor}{Corollary}{Corollaries}

\newtheorem{prop}[lemma]{Proposition}
\Crefname{prop}{Proposition}{Propositions}

\newtheorem*{thm*}{Theorem}
\newtheorem*{lemma*}{Lemma}

\theoremstyle{definition}

\newtheorem{dfn}[lemma]{Definition}
\Crefname{dfn}{Definition}{Definitions}
\newtheorem{example}[lemma]{Example}
\Crefname{example}{Example}{Examples}

\theoremstyle{remark}

\newtheorem{rem}[lemma]{Remark}
\Crefname{rem}{Remark}{Remarks}


\title{On $K_2$-analogue of Serre problem for $\SL_4$}
\keywords {Steinberg group, $K_2$-functor, Serre problem
 {\em Mathematical Subject Classification (2010):} 19C20}
\author {Sergey Sinchuk}
\address{Chebyshev laboratory, St. Petersburg State University, St. Petersburg, Russia}
\email {sinchukss at gmail.com}
\date {\today}

\begin{document}   
\maketitle

\section{Introduction}

The aim of this note is to show that $\K_2$-analogue of Horrocks theorem holds for the unstable group $\K_2(4, R)$.
From this theorem we deduce an improved version of Tulenbayev's theorem on early stability for linear $\K_2$-groups.

Let $R$ be a commutative ring and $\St(n, R)$ be the Steinberg group of rank $n$.
Denote by $\K_2(n, R)$ (resp. $\K_1(n, R)$) the kernel (resp. cokernel) of the canonical map $\phi\colon \St(n, R) \to \GL(n, R)$.
Recall that in~\cite[Theorem~5.1]{Tu83} M.~Tulenbayev proves the following result.
%TODO: Maybe formulate the final result, i. e. early stability theorem
\begin{thm*} For $n \geq 5$ the Steinberg group functor $\St(n, -)$ satisfies $\mathbb{P}^1$-glueing, i.\,e. for every
 local ring $R$ the following square is pullback:
 \[\xymatrix{ \St(n, R)    \ar[r]\ar[d] & \ar[d] \St(n, R[t^{-1}]) \\ 
              \St(\Phi, R[t]) \ar[r]       &        \St(n, R[t, t^{-1}]).}\]
\end{thm*}
In turn, our first main result is the following.
\begin{thm} \label{thm:horrocks-sl4} The functor $\St(4, -)$ also satisfies $\mathbb{P}^1$-glueing. \end{thm}

It is not hard to see that $\mathbb{P}^1$-glueing property holds for the Steinberg group functor $\St(n, -)$ iff it holds for the functor $\K_2(n, -)$.
Therefore the above results should be considered as ``$\K_2$-analogues'' of the classical Horrocks theorem on projective modules (see e.\,g.~\cite[Ch.~IV]{Lam10}).
Recall that the latter asserts that $\mathbb{P}^1$-glueing holds for the functor $\mathrm{VB}_n$ associating to each commutative ring $R$
 the set of constant rank $n$ vector bundles over $R$ ($\mathrm{VB}_n$ thus can be considered as the unstable analogue of $K_0$-functor).

Recall that Horrocks theorem usually appears in the context of Serre problem and its analogues for other functors.
Let $\mathcal{F}\colon \catname{CRings} \to \catname{Sets}_*$ be a functor from commutative rings to pointed sets. 
We say that an analogue of Serre problem holds for $\mathcal{F}$ if for any field $k$ there is an isomorphism $\mathcal{F}(k[t_1, \ldots, t_n]) = \mathcal{F}(k)$. 
Typically, the proof of an analogue of Serre problem for a functor $\mathcal{F}$ (e.\,g. $\mathrm{VB}_n$, $K_1(n, -)$, $K_2(n, -)$) 
 breaks down into the following two steps: 
\begin{enumerate}
 \item proving that an analogue of local-global principle holds for $\mathcal{F}$ (see ??? for the definition below);
 \item proving that $\mathcal{F}$ satisfies $\mathbb{P}^1$-glueing.
\end{enumerate}

The proof of the analogue of Serre problem for the functor $\K_1(n, R)$ was for the first time obtained by A.~Suslin under the assumption $n\geq 3$, see~\cite{Su77}.
A lot of effort has been put afterwards into generalizing Suslin's results to other groups (see e.\,g.~\cite{Abe83, St-poly}).
\begin{comment}
One of the more recent results in this direction is~\cite[Theorem~1.1]{St-poly} due to A.~Stavrova, 
 in which $\mathbb{P}^1$-glueing property is proved for functors $\K_1^G$ modeled on a large class of isotropic reductive groups 
  (consisting, roughly speaking, of groups having local isotropic rank at least $2$). %TODO: Better phrasing
\end{comment}
 
The well-known counterexample of P.~Cohn shows that the $\K_1$-analogue of Serre problem fails for $\K_1(2, -)$ which shows that Suslin's result cannot be improved, see~\cite[\S~I.8]{Lam10}
Similarly, it is known that an analogue of Serre problem fails for $\K_2(3, -)$, see~\cite{We14}.
Thus, our~\cref{thm:horrocks-sl4} completely settles the remaining case $n=4$ not already covered by the results of Wendt and Tulenbayev.

Previously in~\cite{LS17} A.~Lavrenov and the author showed that the local-global principle holds for $\St(4, -)$.
As said before, this together with~\cref{thm:horrocks-sl4} allows us to show the following result, which is an improvement of Tulenbayev stability theorem.
\begin{thm} \label{thm:Serre-problem} %TODO: Fix statement
  For any field $k$ there is an isomorphism $\K_2(4, k[t_1, \ldots, t_n]) \cong \K_2(4, k)$.
  In fact Tulenbayev early stability theorem can be proved starting with the bound $n=4$... 
\end{thm}
It is clear that the positive solution of Serre problem for $\K_2(n, -)$ for $n\geq 4$ is a corollary of this result.
%TODO: Add corollary: some building in simply-connected
%TODO: if 2 is invertible and E = SL then H_2(SL_4(R)) = H_2(SL_4(R[t]))

\subsection{Acknowledgements}
The author would like to thank Anastasia Stavrova and Andrei Lavrenov for posing the problem and numerous helpful discussions regarding this paper.
The author would also like to thank ??? for financial support.

\section{Preliminaries}
\begin{comment}
Our notation and conventions follows~\cite[\S~4]{Vav09}.
Let $\Phi$ be an irreducible root system 

 that are uniquely determined by relations $\langle\varpi_i, \alpha_j^\vee \rangle = (\varpi_i, \alpha_j) = \delta_{ij}.$ %TODO: Mistake?
\end{comment}

For $g\in \GL(n, R)$ we denote by $g^*$ the contragradient matrix, i.\,e. $g^* = {g^{-1}}^t$.
We denote by $e_i$ the standard basis vector of $R^n$. For a pair of vectors $u, v \in R^n$ such that $u^tv = 0$ we denote by
$t(u, v)$ the rank $1$ matrix $e + uv^t$. Notice that $t(u, v)$ lies in $\E(n, R)$ provided $u$ or $v$ is unimodular. %TODO: Reference + other case

%TODO: Introduce notation for GL(n, R); E(n, R)

\subsection{Steinberg groups}
Let $\Phi$ be a reduced and irreducible root system of rank $\geq 2$ and $R$ be a commutative ring with $1$.
Recall that the \emph{Steinberg group} $\St(\Phi, R)$ is defined by means of generators
$\XX_{\Phi, R} = \{x_{\alpha}(\xi) \mid \xi\in R, \alpha\in\Phi\}$ and the set of relations $\RR{\Phi, R}$ 
consisting of the following two families of relations:
\begin{align}
& \phantom{[}
x_\alpha(s) x_\alpha(t) = x_\alpha(s+t),\ \alpha\in\Phi,\ s,t\in R; \label{rel:add}\\
& [x_\alpha(s), x_\beta(t)] = \prod\limits_{i,j\in\mathbb{N}}
 x_{i\alpha + j\beta}\left(N_{\alpha\beta ij}\, s^i t^j\right),\quad \alpha,\beta\in\Phi,\ \alpha\neq\pm\beta,\ s,t\in R. \label{rel:CCF}
\end{align}
The indices $i$, $j$ appearing in the right-hand side of the above relation range over
all positive natural numbers such that $i\alpha + j\beta\in\Phi$.
%The structure constants $N_{\alpha \beta i j}=\pm 1,2,3$ appearing in \eqref{rel:CCF} depend only on $\Phi$ and can be computed precisely.

It is clear the above construction is functorial in $R$, for a ring map $f\colon R\to S$ we denote the corresponding map of Steinberg groups by $f_*$.

%TODO: define Weyl elements w_{\alpha}(\varepsilon)
Recall that for $\alpha\in\Phi$, $\varepsilon\in R^*$ the semisimple root elements $h_\alpha(\varepsilon)$ are defined as $h_\alpha(\varepsilon)=w_\alpha(\varepsilon)w_\alpha(-1)$.
Denote by $\StW(\Phi, R)$ the subgroup of $\St(\Phi, R)$ generated by all elements
$w_\alpha(\varepsilon)$, $\varepsilon\in R^*$, $\alpha\in\Phi$, and by $\StH(\Phi,R)$
 the subgroup generated by all elements $h_\alpha(\varepsilon)$, $\varepsilon\in R^*$, $\alpha\in\Phi$.
We also denote by $\StB(\Phi, R)$ the product of the two subgroups $\StH(\Phi, R)$ and $\UU(\Phi^+, R)$ inside $\St(\Phi, R)$.
 
For $u, v \in R^*$ and $\alpha\in\Phi$ the element $\{u,v\}_\alpha=h_\alpha(uv)h_\alpha(u)^{-1}h_\alpha(v)^{-1}$ will be
 called a {\it Steinberg symbol}. By definition, $\Sym(\Phi, R)$ is the subgroup of $\St(\Phi, R)$ generated by all Steinberg symbols $\{u, v\}_\alpha$, $u, v \in R^*$. 
$\Sym(\Phi, R)$ is contained in $\K_2(\Phi, R)$ and is a central subgroup of $\St(\Phi, R)$ provided $\rk(\Phi)\geq 2$.

There is the following chain of obvious inclusions between the subgroups introduced above:
\begin{equation}
  \Sym(\Phi, R) \subseteq \StH(\Phi, R) \subseteq \StB(\Phi, R) \subseteq \St(\Phi, R)
\end{equation}

\subsection{Another presentation of linear Steinberg groups} \label{sec:another-presentation}
Throughout this subsection $R$ denotes a commutative ring and $n \geq 4$.

For $u \in R^n$ we denote by $D(u)$ the subset of $R^n$ consisting of all vectors $v$ which are orthogonal to $u$ (i.\,e. $u^tv = 0$) and have at least two zero entries.
Recall from 3.2 of~\cite{Ka77} that for every $u, v, w \in R^n$ such that $u^t v = 0$ there
 is a decomposition of $(w^t u) \cdot v$ into a sum of elements of $D(u)$, called {\it canonical decomposition}:
\begin{equation} \label{eq:canonical} (w^tu) \cdot v=\sum_{i<j}u_{ij} c_{ij}(v, w),\end{equation}
where $u_{ij} =e_iu_j-e_ju_i \in D(u)$ and $c_{ij}(v, w) =v_iw_j-v_jw_i \in R$.

\begin{comment}
Let $u, v \in R^n$ be such that $u^t v = 0$ and assume, moreover, that either $u$ or $v$ has at least one zero entry.
Recall from 3.8, 3.10 of~\cite{Ka77} that under these assumptions one can define certain element
 $x(u, v)$ of $\St(n, R)$ such that $\phi(x(u, v)) = t(u, v) = 1 + uv^t$.
Let us recall the standard properties of $x(u, v)$ (cf.~\cite[Lemma~1.1]{Tu83}).

\begin{itemize}
 \item If $v$ or $w$ has at least two zero entries, then
  \begin{equation} \label{item:xsmall-scalar} x(v, wa) = x(va, w), \text{ for $a\in R$} \end{equation}
 \item If $w_1$ and $w_2$ have at least one common zero entry then
 \begin{equation} \label{item:xsmall-additivity} x(v, w_1) \cdot x(v, w_2) = x(v, w_1+w_2).\end{equation}
 \item If $v$, $v'$ are simultaneously orthogonal to $w$ and $w'$ and the elements $w, w'$ both have at least two zero entries then 
  \begin{equation} \label{item:xsmall-commute} [x(v, w),\ x(v', w')] = 1 \end{equation}
\end{itemize}

Now let $I$ be a splitting ideal of $R$.
Let $u \in E(n, R)e_1$, $v \in I^n$ be vectors such that $u^tv = 0$.
Recall from \S~4 of~\cite{LS17} that one can define the following elements of $\St(n, R, I) \leq \St(n, R)$:
\begin{equation} \label{eq:sigma-definition}
 F(u, v) = \prod\limits_{i=1}^r x(u,  v_i),\ \
 S(v, u) = \prod\limits_{i=1}^r x( v_i, u).
\end{equation}
Here $\{v_r\}$, $r=1,\ldots,N$ is any collection of elements of $I^n \cap D(u)$ such that $\sum_{r=1}^N v_r = v$.
Since $u$ is unimodular, such a collection always exists by~\eqref{eq:canonical}, moreover $F(u, v)$ and $S(u, v)$ do not depend on its choice. %TODO: Give reference
\end{comment}

\begin{lemma} \label{lem:xsmall-properties}
 Let $v, w \in R^n$ be such that $v^t w = 0$ and assume, moreover, that either $v$ or $w$ has at least one zero entry.
 Under these assumptions one can define certain element $x(v, w) \in \St(n, R)$ such that $\phi(x(v, w)) = t(v, w) = 1 + vw^t$.
 The elements $x(v, w)$ enjoy the following properties:
 \begin{lemlist}
  \item \label{item:xsmall-scalar} If $v$ or $w$ has at least two zero entries, then $x(v, wa) = x(va, w)$ for $a\in R$. 
  \item \label{item:xsmall-additivity} If $w_1$ and $w_2$ have at least two zero entries of which at least one entry is common
   then $x(v, w_1) \cdot x(v, w_2) = x(v, w_1+w_2)$ and $x(w_1, v) \cdot x(w_2, v) = x(w_1 + w_2, v)$.
  \item \label{item:xsmall-commute} If $v$, $v'$ are simultaneously orthogonal to $w$ and $w'$ and the elements $w, w'$ both have at least two zero entries then 
   $[x(v, w),\ x(v', w')] = 1$.
  \item \label{item:xsmall-conj} If $g = x_{ij}(\xi)$ is a Steinberg generator and $v$ or $w$ has at least two zero entries then
   $g \cdot x(v, w) \cdot g^{-1} = x(gv, g^*w)$.
 \end{lemlist}
\end{lemma}
\begin{proof}
 See~\cite[Lemma~1.1]{Tu83}.
\end{proof}

Now assume that $I$ is an ideal of a ring $R$ which itself is a subring of a larger ring $S$. 
Let $d$ be an element of the subgroup $T(n, S)$ of diagonal matrices of $\GL(n, S)$, notice that $d^* = d^{-1}$.
Let $u \in \Um(n, R)$ and $v \in I^n$ be such that $u^tv = 0$ and let $v = \sum v_i$ 
 be a decomposition of $v$ into a sum of elements $v_i \in D(u) \cap I^n$ (e.\,g. the canonical decomposition~\eqref{eq:canonical}),
 
Under the assumption $d^{-1}u \in R^n$, $d \cdot I^n \subseteq R^n$ set 
 \begin{equation} X_d(u, v) = \prod_i x(d^{-1}u, dv_i) \in \St(n, R). \end{equation}
Under the assumption $d u\in R^n$, $d^{-1} \cdot I^n \subseteq R^n$ set 
 \begin{equation} Y_d(v, u) = \prod_i x(d^{-1} v_i, du) \in \St(n, R). \end{equation}

\begin{lemma} \label{lem:xy-wd}
 The elements $X_d(u, v)$, $Y_d(u, v)$ are well-defined, i.\,e. they do not depend on the choice of decomposition for $v$.
\end{lemma}
\begin{proof}
\begin{comment}
 Decompose each factor $x(d^{-1}u, dv_i)$ further using the canonical decomposition of $v_i$ and then 
  use relations~\eqref{item:xsmall-scalar}--\eqref{item:xsmall-commute} to reorder and collect factors
  in such a way that  arrive to the canonical decomposition of $v$ by (cf.~with Tulenbaev's argument after~\cite[Lemma~1.1]{Tu83}).
\end{comment} 
 Let $v = \sum_r v^r$ be a decomposition as above. 
 Since each $v^r$ is orthogonal to $u$ we can write the canonical decomposition  
 $v^r = \sum_{i<j} u_{ij} c_{ij}(v^r, w)$, moreover, $\sum_{r} c_{ij}(v^r, w) = c_{ij}(v, w)$.
 Now using~\cref{lem:xsmall-properties} we obtain:
 \begin{multline*} \prod\limits_r x(d^{-1}u, dv^r) = \prod\limits_{r}\prod\limits_{i<j} x(d^{-1} u, du_{ij}c_{ij}(v^r, w)) =
  \prod\limits_{i<j} x(d^{-1} u, d u_{ij}c_{ij}(v, w)). \qedhere \end{multline*}
\end{proof}

\begin{lemma} \label{lem:xy-conj} Suppose that $g = x_{hk}(\xi)$ is a generator of $\St(n, R)$ such that $m = d\phi(g)d^{-1} \in \E(n, R)$, then 
 \begin{equation*} g \cdot X_d(u, v) \cdot g^{-1} = X_d(mu, m^*v) \text{ and } g \cdot Y_d(v, u) \cdot g^{-1} = Y_d(mv, m^*u). \end{equation*}
\end{lemma}
\begin{proof}
Direct computation using~\cref{lem:xsmall-properties} (cf. with~\cite[3.14]{Ka77} or~\cite[Lemma~4.4d]{LS17}).
\begin{comment} % ***** Do not remove *****
 Choose $w\in R^n$ such that $w^t u = 1$ and write $v = \sum_{1\leq i<j\leq n} u_{ij} c_{ij}$, where $u_{ij}$ and $c_{ij} = c_{ij}(v, w)$ are as in~\eqref{eq:canonical}. 
 By~\cref{lem:xsmall-properties} we can write $g \cdot X_d(u, v) \cdot g^{-1}$ as follows:
 \begin{equation} \nonumber
   \prod\limits_{1\leq i<j\leq n} x(\phi(g) \cdot d^{-1} \cdot u, \phi(g)^* \cdot d \cdot u_{ij}c_{ij})
 = \prod\limits_{1\leq i<j\leq n} x( d^{-1} \cdot m \cdot u, d \cdot m^* \cdot u_{ij}c_{ij}).
 \end{equation} 
 Now for every factor in the right-hand side we do the following:
 if $\{i, j\} = \{h, k\}$ or $\{i, j\} \cap \{h, k\} = \emptyset$ we leave the factor unchanged,
 otherwise, if, $|\{i, j\} \cap \{h, k\}| = 1$,  say $j = h$, $i\neq k$, we further decompose it as follows:
  \begin{equation} \nonumber
     x(d^{-1} \cdot u', d \cdot m^* \cdot u_{ij} c_{ij}) = 
     x(d^{-1} \cdot u', d \cdot u'_{ij} c_{ij}) \cdot 
     x(d^{-1} \cdot u', d \cdot u'_{ki} c_{ij}).
  \end{equation}
  Here $u' = m  u = t_{jk}(\xi') \cdot u$ for some $\xi' \in R$.
  Notice that $m^* \cdot u_{ij} = t_{kj}(-\xi') \cdot u_{ij} = e_iu_j - e_ju_i + e_k\xi'u_i =u'_{ij}+u'_{ki}\xi'$. 
  
  Thus, we have written the expression in the form $\prod_{s} x(d^{-1} \cdot u', d \cdot v'_s) $, where $v'_s \in D(u')$ and
   $\sum v'_s = m^*v$. It is clear now, that the latter expression equals $X_d(m u, m^* v)$.
\end{comment} 
\end{proof}

\subsection{Relative Steinberg groups} \label{sec:relative-steinberg}
First of all, let us recall the definition of the category of pairs. 
As suggested by the notation, the objects of $\catname{Pairs}$ are pairs $(R, I)$ consisting of a unital commutative ring $R$ and an ideal $I \trianglelefteq R$.
The set of morphisms between $(R, I)$ and $(S, J)$ consists, by definition, of all ring-theoretic maps $f\colon R\to S$ such that $f(I)\subseteq J$.
There is an obvious fully faithful embedding $\iota\colon \catname{CRings} \to \catname{Pairs}$ defined on objects by $\iota(R) = (R, R)$.

Recall that there is a way to ``relativize`` the Steinberg group functor, that is define a group-valued functor $\St(\Phi, -)$ on the category of pairs
 in a way compatible with the embedding 

\begin{equation}
 \xymatrix{ 1 \ar[r] & C(\Phi, R, I) \ar[r] & \St(\Phi, R, I) \ar[r]^{\ext{I}} & \St(\Phi, R) \ar[r]^{\pi_*} & \St(\Phi, R/I) \ar[r] & 1 }
\end{equation}

In the sequel we will also need a {\it relative Steinberg symbol}, i.\,e. a map
\begin{equation*} \{-, -\}'_\alpha \colon R^* \times (1 + I)^* \to \St(\Phi, R, I) \end{equation*}
such that $ \ext{I} \{u, v\}'_\alpha = \{u, v\}_\alpha$. It is not hard to verify that the image under $\mu_I$ of the element
\begin{equation*} f(\alpha, t, u)= y_\alpha(t(u-1)) \cdot y_{-\alpha}(-t^{-1}(u^{-1}-1))^{x_\alpha(-t)} \cdot y_{\alpha}(t(u-1))^{w_\alpha(-t)}, \end{equation*}
equals $w_\alpha(tu) \cdot w_\alpha(-t)$. This allows us to set $\{t, u \}'_\alpha := f(\alpha, t, u) f(\alpha, 1, u)^{-1}.$
Now, since $\mu_I\{ t, u \}'_\alpha = w_\alpha(tu) w_\alpha(-t) w_\alpha(1) w_\alpha(-u) = \{t, u\}_\alpha,$ 
 we have defined the desired symbol.
\begin{rem}
 Our relative Steinberg symbols differs from the one defined in~\cite{Ste73} in that
  it takes values in $\St(\Phi, R, I)$ rather than $\St(\Phi, R)$. %Elaborate on this...
\end{rem}

In the sequel we will need an explicit presentation of the relative linear Steinberg group by means of generators and relations.
The presentation we formulate below should be thought of as an ''improved`` version of the presentation of the relative linear Steinberg group
 given by Tulenbaev in~\cite[Definition~1.5]{Tu83} (cf. also~\cite[Proposition~3.2]{LS17}).
We refer the reader to~\cite[Section~3]{LS17} for the discussion why this presentation is better than Tulenbaev's one
 (the short answer is: it remains useful in the case $\Phi = \rA_3$).

\begin{prop}[\text{\cite[Proposition 3.10]{LS17}}] \label{prop:rel-presentation}
 Assume that $I$ is a splitting ideal of a commutative ring $R$. 
 Then for any $\ell\geq 3$ the group $\St(\rA_\ell,\,R,\,I)$ can be presented by means of two families of generators $F(u,\,v)$, $S(v,\,u)$
  (where $u\in \E(n,\,R)e_1,$ and $v\in I^n$ are such that $u^tv=0$) subject to the following relations:
\begin{align}
&F(u,\,v)F(u,\,w)=F(u,\,v+w), \label{add4}\\
&S(u,\,v)S(w,\,v)=S(u+w,\,v), \label{add5}\\
&F(u,\,v)F(u',\,v')F(u,\,v)^{-1}=F(t(u,\,v)u',\,t(v,\,u)^{-1} v'), \label{conj3} \\
&F(ge_1,\,g^*e_2a)=S(ge_1a,\,g^*e_2),\ \text{for all}\ a\in I,\, g \in E(n, R). \label{coef-move}
\end{align}
\end{prop}
\begin{rem}
Although we will not need this, using essentially the same reasoning as in~\cite[Proposition~8]{S15} it is not hard to show that 
$\St(\rA_\ell, R, I)$ still admits the above presentation without the assumption that $I$ is splitting.
\end{rem}

\subsection{Various stuff}
\[ T_{ij}(at^{-1}) = \sigma_i T_{ij}(a) \sigma_i^{-1} = \sigma_j^{-1} T_{ij}(a) \sigma_j \]
\[ [[x_{ij}(a t^{-1}), x_{jk}(b t^{-1})],  x_{kl}(c)] = [x_{ik}(a b t^{-1}),  x_{kl}(c t^{-1})] \]
\[ \sigma_i [[x_{ij}(a), x_{jk}(b t^{-1})],  x_{kl}(c)] \sigma_i^{-1} =  \sigma_i [x_{ik}(a b),  x_{kl}(c t^{-1})] \sigma_i^{-1} \]
\begin{multline}
[[x_{ij}(at^{-1}), x_{jk}(bt^{-1})], x_{jl}(ct^{-1})] = \sigma_i [[ x_{ij}(a), x_{jk}(bt^{-1})], x_{jl}(ct^{-1})] \sigma_i^{-1} = \\
\sigma_i \sigma_k^{-1} \sigma_l^{-1} [[ x_{ij}(a), x_{jk}(b)], x_{jl}(c)] \sigma_l \sigma_k \sigma_i^{-1} = 1
\end{multline}
\[ [b, c] =1,\ [a^c, b] = c^{-1} a c b c^{-1} a^{-1} c b^{-1} =  c^{-1} a b  a^{-1} b^{-1} c = [a, b]^c \]

We need to prove that $\sigma_i$ commute with $x_{jk}(a)$ if $j\neq i, k \neq i$.


\section{Tulenbaev's factorization theorem}
Let $\Phi$ be a root system with some fixed basis of simple roots $\Pi = \{\alpha_1, \ldots, \alpha_\ell\}$.
For a root $\alpha\in\Phi$ we denote by $m_i(\alpha)$ the $i$-th coefficient in the expansion of $\alpha$ in $\Pi$,
 i.\,e. $\alpha = \sum_{i=1}^n m_i(\alpha) \alpha_i$.

We denote by $\Phi^\vee$ the corresponding dual root system, which, by definition, consists of all coroots 
 $\alpha^\vee = \frac{2}{(\alpha, \alpha)} \alpha$, where $\alpha \in \Phi$.
We denote by $P(\Phi^\vee)$ the integral lattice spanned by the \emph{fundamental co-weights $\varpi^\vee_i$} (i.\,e. weights of the dual root system).
Recall that the co-weights $\varpi^\vee_i$ are uniquely determined by the property $(\varpi_i^\vee, \alpha_j) = \delta_{ij}$.

Notice that for $\omega \in P(\Phi^\vee)$ and $\beta \in \ZZ \Phi$ one has $(\omega, \beta) \in \ZZ $.
Thus, for $\varepsilon \in R^*$ and $\omega \in P(\Phi^\vee)$ we can define a map of abelian groups $\chi(\omega, \varepsilon) \in \Hom(\ZZ \Phi, R^*)$
 via the formula $\chi(\omega, \varepsilon)(\beta) = \varepsilon ^ {(\omega, \beta)}$.

Every map of abelian groups $\chi \in \Hom(\ZZ \Phi, R^*)$ specifies an action on the set of generators $\XX_{\Phi, R}$ of the Steinberg group $\St(\Phi, R)$ via
\begin{equation*} \chi \cdot x_\alpha(\xi) = x_\alpha(\chi(\alpha) \cdot \xi),\ \alpha\in \Phi,\ \xi \in R. \end{equation*}
It is not hard to check that this action is compatible with Steinberg relations $\RR{\Phi, R}$ and 
 hence gives a well-defined automorphism of $\St(\Phi, R)$ which we denote by the same symbol $\chi$.
 
\begin{example} \label{exm:chi-linear}
For $1\leq k\leq \ell+1$ and $\varepsilon \in R^*$ denote by $d_k(\varepsilon)$ the matrix from $\GL(\ell+1, R)$ which differs from the unit matrix 
 only by having the element $\varepsilon$ on the $k$-th place of its diagonal.
Recall from~\cite[Corollary~4]{Ka77} that for any $g \in \GL(\ell+1, R)$ there exists an automorphism $\beta_g$ of $\St(\ell+1, R)$ 
 ''modeling`` the automorphism $\alpha_g \colon \GL(\ell+1, R) \to \GL(\ell+1, R)$ of inner conjugation by $g$, i.\,e. such that $\phi \beta_g = \alpha_g \phi$.

It is clear that in the linear case the map $\chi(\varpi_k, \varepsilon)$ coincides with $\beta_{d_k(\varepsilon)}$, 
 while for other Chevalley groups the maps $\chi(\omega, k)$ model automorphisms of inner conjugation by weight elements $h_\omega(\varepsilon)$
  in the sense of~\cite[\S~4]{Vav09}.
\end{example}

Let $\omega \in P(\Phi^\vee)$ be a co-weight of $\Phi$ as above.
Denote by $\XX(\omega)$ the subset of $\XX_{\Phi, A[t]}$ consisting of those generators $x_{\alpha}(\xi)$ of $\St(\Phi, A[t])$ for which 
$(\alpha, \omega) < 0$ implies that $\xi \in A[t]$ is divisible by $t^{-(\alpha, \omega)}$. Denote by $N(\omega)$ the subgroup of $\St(\Phi, A[t])$ generated by $\XX(\omega)$.

\begin{dfn} \label{dfn:sigma-pair}
By definition, a {\it $\sigma$-pair for $\omega$} is a pair of mutually inverse group homomorphisms 
$\xymatrix{ \sigma(\omega)\colon N(\omega) \ar[r] & \ar@<-1.0ex>[l] N(-\omega)\colon \sigma(-\omega) }$ satisfying
\begin{equation} \label{eq:sigmadef}
\sigma(\pm \omega)(x_\alpha(\xi)) = x_\alpha(t^{(\pm \omega, \alpha)}\cdot \xi), 
 \text{ for all } x_\alpha(\xi) \in \XX(\pm\omega).
\end{equation}\end{dfn}
It is clear that the maps $\sigma(\omega)$, $\sigma(-\omega)$ are uniquely determined by~\eqref{eq:sigmadef}, so at most one $\sigma$-pair may exist for any given $\omega$.
Moreover, the maps $\sigma(\omega), \sigma(-\omega)$ make the following diagram commute:
\begin{equation} \label{eq:sigma-diagram}
\xymatrix{ N(\omega) \ar[r]^{\sigma(\omega)}\ar@{^{(}->}[d] & N(-\omega) \ar@{^{(}->}[d] \ar[r]^{\sigma(-\omega)} & N(\omega) \ar@{^{(}->}[d] \\ 
          \St(\Phi, A[t]) \ar[d] & \St(\Phi, A[t]) \ar[d] & \St(\Phi, A[t]) \ar[d] \\
          \St(\Phi, A[t, t^{-1}]) \ar@<-0.0ex>[r]_{\chi(\omega, t)} & \St(\Phi, A[t, t^{-1}]) \ar@<-0.0ex>[r]_{\chi(-\omega, t)} & \St(\Phi, A[t, t^{-1}]).} \end{equation}  

\begin{rem} Despite the fact that property~\eqref{eq:sigmadef} determines the maps $\sigma(\pm \omega)$, $\sigma(-\omega)$ uniquely, 
 we do not know whether maps with this property actually exist.
This is caused by the fact there is no direct description of the set of relations which hold in $N(\pm\omega)$ between the generators $\XX(\pm\omega)$, 
 so we cannot be sure a priori that $\sigma(\pm\omega)$ preserve these relations. \end{rem}

\subsection{Factorization theorem} 
Throughout this section $A$ denote a commutative local ring with the maximal ideal $m$.
$j_+\colon A[t] \to A[t, t^{-1}]$. We write $j_+^*$ instead of $\St(\Phi, j_+)$.
 
Throughout this section we use the following notation.
\begin{itemize}
 \item 
\end{itemize}

Denote by $V_0$ the set $\St(\Phi, A[t]) \times \StH(A[t, t^{-1}]) \times \St(\Phi, A[t, t^{-1}], m[t, t^{-1}])$. 
We define an equivalence relation $\sim$ on $V_0$ as the minimal equivalence relation generated by the following 4 elementary transformations:
\begin{enumerate}
 \item $(a \cdot u, b, \beta) \sim (a, j_+^*(u) \cdot b, \beta)$, where $u$ is an element $\UU(\Phi^+, A[t])$;
 \item $(a \cdot \ext{m[t]}(\gamma), b, \beta) \sim (a, b, j^*_+(\gamma)^b \cdot \beta)$, where $\gamma$ is an element of $\St(\Phi, A[t], m[t])$;
 \item $(a, b \cdot u, \beta) \sim (a, b, u\cdot \beta)$, where $u$ is an element of $\UU(\Phi^+, m[t, t^{-1}])$;
 \item $(a, b \cdot \{ t, a \}_\alpha, \beta) \sim (a, b, \{ t, a \}'_\alpha \cdot \beta) $, where $a \in (1+m)^*$
  and $\{t, a\}'_\alpha$ is a relative Steinberg symbol discussed in~\cref{sec:relative-steinberg}
\end{enumerate}

\section{Proof of the main result}
For the remainder of this subsection our main goal will be to verify that an analogue of the group decomposition from \cite[Lemma~3.1f]{Tu83} holds for $\St(\Phi, A)$
 provided $\St(\Phi, R)$ admits at least one $\sigma$-pair.

Denote by $\widetilde{W}$ the subgroup of $\St(\Phi, A)$ generated by elements $w_\alpha(1)$.
The subgroup $\widetilde{W}$ acts on $\St(\Phi, A)$ via conjugation, for example, 
 the action on the generators of $\mathcal{X}_{\Phi, A}$ can be expressed via the formula ${}^{w_\alpha(1)} \!x_{\beta}(\xi) = x_{\sigma_{\alpha}(\beta)} ( \pm \xi)$.
It is easy to see that the action of $\widetilde{W}$ on the set of root subgroups $\{X_\alpha \mid \alpha \in \Phi \}$ of $\St(\Phi, R)$ coincides with the obvious
 action of the Weyl group $W(\Phi)$ on it. Since $N_i$ and $N^i$ are generated by root subgroups we immediately obtain the following statement.
\begin{lemma} The set of elements in the orbit of $N_i$ (resp. $N^i$) under the conjugation action of $\widetilde{W}$ 
is in bijective correspondence with 
the set of adjacency classes of $W(\Phi)$ with respect to its Levi subgroup $W_i = W(\Delta_i)$. %TODO: Unsure whether it is true at all :)
\end{lemma}
In the sequel we denote by $\Sigma$ the set of elements in $\widetilde{W}$ whose projections in $W(\Phi)$ form the set of representatives for
 right adjacency classes of $W(\Phi)$ with respect to $W_i$. %We can assume these representatives to be short

\subsection{Construction of \texorpdfstring{$\sigma$}{sigma}-pair for the Steinberg group of rank 3}
The aim of this subsection is to show that for the root system $\Phi=\rA_3$ and $\omega = \varpi_1^\vee$
 there exists a $\sigma$-pair in the sense of~\cref{dfn:sigma-pair}.
Notice that for $\Phi = \rA_\ell, \ell \geq 4$ and $\omega=\varpi_k^\vee$, $k=1,\ldots,\ell+1$ 
the corresponding $\sigma$-pair has been constructed by Tulenbaev in the beginning of~\S~3 in~\cite{Tu83} 
 (where the corresponding maps and subgroups are denoted $\xymatrix{\delta_k \ar[r] \colon N_k & \ar@<-1.0ex>[l] N^k \colon \delta_k^{-1}}$).

To simplify notation, throughout this section we set $N_1 = N(\varpi_1)$, $N^1 = N(-\varpi_1)$.
We only construct one of the desired maps from the definition of $\sigma$-pair, namely $\sigma(\varpi_1^\vee) \colon N_1 \to N^1$,
 for shortness we denote it by $\sigma_1$.
 
In order to define $\sigma_1$ we will use the description of the subgroup $N_1 = N(\varpi_1^\vee)$ given in terms of ``another presentation''.
We start with the following simple observation.
\begin{lemma} \label{lem:n1-decomp} For $n\geq 4$ there is an isomorphism $N_1 \cong N_{0} \rtimes P_1^-(A)$, 
 where $N_{0}$ denotes the subgroup $\St(n, A[t], tA[t])$ and $P_1^-(A)$ is the subgroup of $\St(n, A)$ generated by $x_{ij}(\xi)$ with $i\neq 1$.
\end{lemma}

Recall that the universal property of semidirect products gives for any group $H$ acting on a group $N$ via $\phi \colon H \to \Aut(N)$ 
and any group homomorphisms $f_N\colon N \to G$, $f_H\colon H \to G$ satisfying 
\begin{equation} \label{eq:coherence-condition} f_N(\phi(h)(n)) = f_H(h) f_N(n) f_H(h)^{-1},\ (n\in N,\ h\in H) \end{equation} 
a unique map $f\colon N \rtimes H \to G$ extending $f_N$ and $f_H$.

Thus, by the above lemma, in order to construct the map $\sigma_1$ we need to construct two maps 
\[ (\sigma_1)_{P_1^-(A)} \colon P_1^-(A) \to N^1, \ \ (\sigma_1)_{N_{0}} \colon N_{0} \to N^1\]
and then verify that they satisfy~\eqref{eq:coherence-condition}.

It is easy to define the first map $(\sigma_1)_{P_1^-(A)}$, indeed, using the decomposition $P_1^-(A) \cong U^-_1 \rtimes L_1$ where 
\[U^-_1 = \langle x_{i1}(\xi) \mid i\neq 1,\ \xi\in A \rangle \text{ and } L_1 = \langle x_{ij}(\xi) \mid i,  j \neq 1,\ \xi\in A\rangle \]
we apply the universal property of semidirect products once again 
and define $(\sigma_1)_{P_1^-(A)}$ by requiring that it acts identically on $L_1$ (notice that $L_1 \subseteq N^1$) %TODO: Language???
and acts on elements of $U^-_1$ via the formula $(\sigma_1)_{P_1^-(A)}(x_{i1}(\xi))= x_{i1}(t\xi)$.

To define the map $(\sigma_1)_{N_{1,0}}$ we invoke the presentation of the relative Steinberg group $\St(n, A[t], tA[t])$ given in~\cref{prop:rel-presentation}.
%and the elements constructed in~\cref{sec:another-presentation}. in the special case $R=A[t]$, $I=tA[t]$.
%We only verify the second assertion for $F(u, v)$  and only in the special case $g = x_{hk}(\xi) \in \mathcal{X}^1$, $g' = t_{hk}(\xi') \in \pi(\mathcal{X}_1)$.

Denote by $\delta_1$ the matrix $d_1(t) \in \GL(A[t, t^{-1}])$ from~\cref{exm:chi-linear}.
Now for $u \in E(n, A[t])$ and $v \in tA[t]$ we can define the map $\sigma_1$ on the generators of $N_{0}=\St(n, A[t], tA[t])$ 
 using the elements defined in~\cref{sec:another-presentation}:
\begin{equation*}
 (\sigma_1)_{N_0} (F(u, v)) = X_{\delta_1 \cdot t^{-1}}(u, v),\ (\sigma_1)_{N_0} (S(v, u)) = Y_{\delta_1}(v, u).
\end{equation*}

\begin{comment}
\begin{equation}
 \sigma_1(F(u, v)) = \prod\limits_{i=1}^r x(\delta_1^{-1} \cdot t u, \delta_1 \cdot t^{-1} v_i),\ \
 \sigma_1(S(v, u)) = \prod\limits_{i=1}^r x(\delta_1^{-1} \cdot v_i, \delta_1 \cdot u).
\end{equation}

\begin{lemma} \label{lem:cor-conj}
 \begin{lemlist}
   \item \label{item:correctness} The elements $\sigma_1(F(u, v))$ and $\sigma_1(S(v, u))$ are well-defined, i.\,e. they do not depend on the choice of decomposition for $v$.
   \item \label{item:conjugation} For any $g \in N^1$ one has 
      \begin{equation}
          \nonumber {}^g\left(\sigma_1(F(u, v))\right) = \sigma_1\left(F(g' \cdot u, {g'}^{*} \cdot v)\right),\ \ 
                    {}^g\left(\sigma_1(S(u, v))\right) = \sigma_1\left(S(g' \cdot u, {g'}^{*} \cdot v)\right),                    
      \end{equation}
      where $g' = {}^{\delta_1}\pi(g).$
 \end{lemlist}
\end{lemma}
\end{comment}

\begin{prop}
 If $A$ is a local ring them the map $(\sigma_1)_{N_0}$ preserves relations~\eqref{add4}--\eqref{coef-move}.
 In particular, the map $(\sigma_1)_{N_0}$ is well-defined.
\end{prop}
\begin{proof}
 For~\eqref{add4}--\eqref{add5} this is an immediate corollary of~\cref{item:xsmall-additivity} and~\cref{lem:xy-wd}
 For~\eqref{conj3} this follows from the second part of the lemma applied in the special case $g = \sigma_1(F(u, v)) \in N^1$.
\end{proof}

It remains to verify that $\sigma_1$ preserves~\eqref{coinc}.
Let us do this at first in the special case when $g^*$ belongs to the subset $G_0 = H_{12}(R) \cdot U^+_1(R) \cdot U^-_1(R)$.
Here we denote by $H_{12}(R)$ the subgroup of $T(n, R)$ generated by semisimple root elements $ h_{12}(\xi)$, $\xi \in R^*$.

Since the only nonzero components of $u = g^* \cdot e_2$ are $u_1$ and $u_2$ we can present $v = g \cdot e_1$ as a sum $v' + v''$, where 
 $v'_i=0$, $i>2$ and $v''_1=v''_2=0$.
Since $v', v'' \in D(u)$ and $u \in D(v)$, from the definition of $\sigma_1$ we obtain that:
\begin{multline} \label{eq:special-case}
 \sigma_1(S(g \cdot e_1a, g^* \cdot e_2)) = x(\delta_1^{-1} \cdot v'a, \delta_1\cdot  u) \cdot x(\delta_1^{-1}\cdot v''a, \delta_1 \cdot u) = \\
  = x(\delta_1^{-1} va, \delta_1 \cdot u) = x(\delta_1^{-1} vat^{-1}, \delta_1 \cdot ut) = \sigma_1(F(g \cdot e_1, g^* \cdot e_2a)) 
\end{multline}

To obtain the assertion in the general case notice that for $R$ local the group $\E(n, R)$ admits the following decomposition:
\[\E(n, R) = \mathrm{EP}_1(R) \cdot H_{12}(R) \cdot U^-_1(R) \cdot U^+_1(R).\]
This can be either obtained as a corollary of Gauss decomposition or can be proved by a direct calculation. %TODO: Give reference

Applying transpose-inverse automorphism $(-)^*$ to the above decomposition and invoking~\cref{lem:n1-decomp} we obtain that $\E(n, R[t]) = \pi(N_1) \cdot G_0$.
Thus, every $g \in \E(n, R[t])$ can be factored as $\pi(n) \cdot h$ for some $n\in N_1$ and $h \in G_0$.
Since $\pi(n) = {}^{\delta_1} \pi(n')$ for some $n' \in N^1$ it remains to apply~\cref{item:conjugation} and~\eqref{eq:special-case} to obtain the required assertion:
\begin{multline} \nonumber \sigma_1(F(g \cdot e_1, g^* \cdot e_2a)) = {}^{n'}(\sigma_1(F(h \cdot e_1, h^* \cdot e_2a))) = \\
                           = {}^{n'}(\sigma_1(S(h \cdot e_1a, h^* \cdot e_2))) = \sigma_1(S(g \cdot e_1a, g^* \cdot e_2)). \end{multline}

Thus, we have completed the construction of the map $(\sigma_1)_{N_{1,0}}$.
It is not hard to verify that the condition~\eqref{eq:coherence-condition} is satisfied with these definitions,
 therefore, we have also completed the construction of the map $\sigma_1$.

\subsection{Proof of the theorem on \texorpdfstring{$\mathbb{P}^1$}{P1}-glueing for $K_2$}

\section{Applications}

\subsection{Bruhat--Tits buildings}

\subsection{\texorpdfstring{$A^1$}{A1}-invariance of the fundamental group}

\DeclareRobustCommand{\VAN}[2]{#2}
\printbibliography

\end{document}
