\documentclass[oneside, 10pt]{amsart}
\usepackage{amscd, amsmath, amssymb, amsthm, amsfonts, amstext, verbatim, mathtools, xfrac, microtype, nameref, thmtools}
\usepackage[breaklinks=true]{hyperref}
\usepackage[hyperref=true, backend=bibtex, firstinits=true, citestyle=numeric-comp, sortlocale=en_US, url=false, doi=false, eprint=true, maxbibnames=4]{biblatex}
\usepackage[capitalize]{cleveref}
\usepackage[matrix,arrow,curve]{xy}
\usepackage{tikz}
\usepackage{enumitem}

\addbibresource{paper.bib}
\renewbibmacro*{volume+number+eid}{\ifentrytype{article}{\- \iffieldundef{volume}{}{Vol.~\printfield{volume},}\iffieldundef{number}{}{ No.~\printfield{number},}}}
\renewbibmacro{in:}{\ifentrytype{article}{}{\printtext{\bibstring{in}\intitlepunct}}}
\newbibmacro{string+doi}[1]{\iffieldundef{doi}{\iffieldundef{url}{#1}{\href{\thefield{url}}{#1}}}{\href{http://dx.doi.org/\thefield{doi}}{#1}}}
\DeclareFieldFormat[article, inproceedings, inbook, book, thesis]{title}{\usebibmacro{string+doi}{\mkbibquote{#1}}}
\renewcommand*{\bibfont}{\footnotesize}

\newtheorem{proposition}{Proposition}
\newtheorem{theorem}{Theorem}
\newtheorem{corollary}{Corollary}
\newtheorem{lemma}{Lemma}
\theoremstyle{remark} %\newtheorem{rem}[lemma]{Remark} \Crefname{rem}{Remark}{Remarks}

\theoremstyle{definition}
\newtheorem{df}[lemma]{Definition} \Crefname{df}{Definition}{Definitions}
\newtheorem{example}[lemma]{Example} \Crefname{example}{Example}{Examples}
\newtheorem{rem}[lemma]{Remark}
\DeclareMathOperator{\St}{St}

\title{On the presentation of orthogonal groups over polynomial rings}
\keywords {Steinberg group, $K_2$-functor, Serre problem
 {\em Mathematical Subject Classification (2010):} 19C20}
\author {Sergey Sinchuk}
\address{Chebyshev laboratory, St. Petersburg State University, St. Petersburg, Russia}
\email {sinchukss at gmail.com}
\date {\today}

\begin{document}
\maketitle
\section{Introduction}
\section{Preliminaries}
\subsection{Isogenous forms of the orthogonal group}
\subsection{Grothendieck--Witt groups}
\section{Proof of the main result}
\subsection{Overview of the proof}
\subsection{First injectivity theorem}
\subsection{Second injectivity theorem}

Statement \[x_{\alpha}(aX^{-1}) g x_{\alpha}(-a(1+am)^{-1}X^{-1}) \cdot \{X, 1+am\} \in G_+.\]

$g = x_{-\alpha}(mX)$, 

$y := x_{\alpha}(aX^{-1}) g x_{\alpha}(-a(1+am)^{-1}X^{-1})$, 

$z = x_{-\alpha}(-m(1+am)^{-1}X) y h_{\alpha}((1+am)^{-1})$,

$\alpha = h_{ik}^{-1}(X)$.

Using $\pi(z) = 1$ we get 
\begin{multline}z = z^\alpha = \\ x_{ji}(-m(1+am)^{-1}) x_{ij}(a) x_{ji}(m) \cdot x_{ij}(-a(1+am)^{-1})\{X, (1+am)^{-1}\} \cdot h_{ij}((1+am)^{-1}) \in G. \end{multline}

We need to introduce notation for certain subgroups of $\St(A[X, X^{-1}])$:
\begin{align}
 G_+   & = \mathrm{Im}(\St(\Phi, A[X], Xm[X]) \to \St(A[X, X^{-1}]))\\
 G_+^0 & = \mathrm{Im}(\St(\Phi, A[X], m[X]) \to \St(A[X, X^{-1}]))
\end{align}
Notice that for $ g\in G_+$ the element $ev_{X=0}(g)^{-1}g$ lies in $G_+^0$.
\begin{lemma} Assume that $\Phi$ is simply-laced.
 For any $a\in A$, $\alpha\in \Phi$, $g_0 \in G^0_+$ there exist $m'\in m$ and $g'\in G_+$ such that 
 \[x_{\alpha}(aX^{-1}) g_0  = g' x_\alpha(-a(1+m')X^{-1}) \cdot \{X, 1+m'\}.\]
\end{lemma}
\begin{proof}
 Choose some parabolic subset $S$ of roots in $\Phi$ in such a way that its special part $\Sigma(S)$ does not contain the root $\alpha$.
 In view of~\cref{lem:Zgen} the subgroup $G_+^0$ is generated by images in $\St(\Phi, A[X, X^{-1}]$ of the elements of the set $\mathcal{Z}(\Sigma(S), A[X], Xm[X])$.
 
 Without loss of generality we may assume that $g_0$ is either $z_\beta(s, \xi)$ for some $\beta\in \Sigma(S)$ for some $s\in Xm[X]$, $\xi\in A[X]$ or $x_\gamma(s)$ for some $\gamma \in \Phi$.
 \begin{itemize}
  \item {\it Case $\alpha \perp \beta$. } In this case the required statement is obvious (with $m'=0$, $g'=g_0$).
  \item {\it Case $\alpha \not \perp \beta \in \Phi$.} In this case either $\alpha + \beta$ or $\alpha - \beta$ lies in $\Phi$.
  Assume, for example, the former (the argument in the other case is similar).
  The required assertion now follows from the following equality (cf. Sinchuk 16' paper Lemma 9):
  %\[ x_\beta(-aX^{-1}) z_\alpha(s, \xi) x_\beta(aX^{-1}) = x_\beta(-saX^{-1}\xi) \cdot x_{\alpha+\beta}(N_{\alpha,\beta}saX^{-1})\cdot z_\alpha(s, \xi).\]
  \[ x_\alpha(-aX^{-1}) z_\beta(s, \xi) =  \left( z_\beta(s, \xi) \cdot x_{\alpha+\beta}(N_{\beta,\alpha}saX^{-1}) \cdot x_\alpha(-s\xi aX^{-1}) \right) \cdot x_\alpha(-aX^{-1}).\]
  \item {\it Case $\alpha = \beta$}
  \item {\it Case $\alpha = -\beta$}
 \end{itemize}

\end{proof}

\section{Proof of Rehmann's presentation theorem}

\printbibliography
\end{document}
