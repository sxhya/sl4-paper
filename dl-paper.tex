\documentclass[oneside, 10pt]{amsart}
\usepackage{amscd, amsmath, amssymb, amsthm, amsfonts, amstext, verbatim, mathtools, xfrac, microtype, nameref, thmtools}
\usepackage[breaklinks=true]{hyperref}
\usepackage[hyperref=true, backend=bibtex, firstinits=true, citestyle=numeric-comp, sortlocale=en_US, url=false, doi=false, eprint=true, maxbibnames=4]{biblatex}
\usepackage[capitalize]{cleveref}
\usepackage[matrix,arrow,curve]{xy}
\usepackage{tikz}
\usepackage{enumitem}

\addbibresource{paper.bib}
\renewbibmacro*{volume+number+eid}{\ifentrytype{article}{\- \iffieldundef{volume}{}{Vol.~\printfield{volume},}\iffieldundef{number}{}{ No.~\printfield{number},}}}
\renewbibmacro{in:}{\ifentrytype{article}{}{\printtext{\bibstring{in}\intitlepunct}}}
\newbibmacro{string+doi}[1]{\iffieldundef{doi}{\iffieldundef{url}{#1}{\href{\thefield{url}}{#1}}}{\href{http://dx.doi.org/\thefield{doi}}{#1}}}
\DeclareFieldFormat[article, inproceedings, inbook, book, thesis]{title}{\usebibmacro{string+doi}{\mkbibquote{#1}}}
\renewcommand*{\bibfont}{\footnotesize}

\newtheorem{proposition}{Proposition}
\newtheorem{theorem}{Theorem}
\newtheorem{corollary}{Corollary}
\newtheorem{lemma}{Lemma}
\theoremstyle{remark} 

\theoremstyle{definition}
\newtheorem{df}[lemma]{Definition} \Crefname{df}{Definition}{Definitions}
\newtheorem{example}[lemma]{Example} \Crefname{example}{Example}{Examples}
\newtheorem{rem}[lemma]{Remark}
\DeclareMathOperator{\St}{St}

\title{On the presentation of orthogonal groups over polynomial rings}
\keywords {Steinberg group, $K_2$-functor, Serre problem
 {\em Mathematical Subject Classification (2010):} 19C20}
\author {Sergey Sinchuk}
\address{Chebyshev laboratory, St. Petersburg State University, St. Petersburg, Russia}
\email {sinchukss at gmail.com}
\date {\today}

\begin{document}
\maketitle
\section{Introduction}
\section{Preliminaries}
$\langle \alpha, \beta \rangle = (\alpha, \beta^\vee) = (\alpha, \frac{2\beta}{(\beta, \beta)})$
Our convention for Cartan matrices is 
$A_{ij} = \langle \alpha_i, \alpha_j \rangle = (\alpha_i, \alpha_j^\vee)$.

\subsection{Steinberg groups}
Definition of Steinberg symbol is:
\[ \{ s, t \}_\alpha = h_\alpha(s) \cdot h_\alpha(t) \cdot h_\alpha(st)^{-1} \]

Then Matsumoto's Lemmas~5.2 and 5.4 assert that:
\[ {}^{h_\alpha(t)}\!x_\beta(u) = x_\beta(t^{\langle \beta,  \alpha \rangle}u)\]
\[ {}^{h_\alpha(t)}\!h_\beta(u) = h_\beta(t^{\langle \beta, \alpha \rangle} \cdot u) \cdot h_\beta(t^{\langle \beta,  \alpha \rangle})^{-1}. \]
\[[h_\alpha(s), h_\beta(t)] = \{s, t^{\langle \alpha, \beta \rangle}\}_\alpha = \{t, s^{\langle \beta, \alpha \rangle}\}_\beta^{-1}\]

\begin{comment}
Suppose for a moment that $\langle \alpha, \beta \rangle = -1$ and  $\langle \beta, \alpha \rangle = -1$ then
\[ \{s, t^{-1} \} = \{s,  t^{-1}\}_\alpha = \{t, s^{-1} \}_\beta^{-1} = \{s^{-1}, t\} \]
In particular, $\{s, s^{-1}\} = \{s, s^{-1}\}^{-1}$ 
\end{comment}

\subsection{Isogenous forms of the orthogonal group}
\subsection{Grothendieck--Witt groups}
\section{Proof of the main result}
\subsection{Overview of the proof}
\subsection{First injectivity theorem}
\subsection{Second injectivity theorem}

Recall the definition of the Dennis--Stein symbol (cf. III.5.11 within Weibel's book).
\begin{equation} \label{eq:dennis-stein}
 \langle r,s \rangle _ \alpha = x_{-\alpha}\left(-\frac{s}{1 - rs}\right) \cdot x_{\alpha}(-r) \cdot x_{-\alpha}(s) \cdot x_{\alpha}\left(\frac{r}{1-rs}\right) \cdot h_{\alpha}(1 - rs)^{-1},
\end{equation} 
where $r, s$ are arbitrary elements of $R$ such that $1 - rs\in R^*$ then

\begin{comment}
\[x_{\alpha}(aX^{-1}) x_{-\alpha}(mX) x_{\alpha}(-a(1+am)^{-1}X^{-1}) \cdot \{X, 1+am\} \in G_+.\]
$y := x_{\alpha}(aX^{-1}) x_{-\alpha}(mX) x_{\alpha}(-a(1+am)^{-1}X^{-1})$, 
$z = x_{-\alpha}(-m(1+am)^{-1}X) y h_{\alpha}((1+am)^{-1})$,
Using $\pi(z) = 1$ we get 
\begin{multline}z = z^{h_{ik}^{-1}(X)} = \\ x_{ji}(-m(1+am)^{-1}) x_{ij}(a) x_{ji}(m) \cdot x_{ij}(-a(1+am)^{-1})\{X, (1+am)^{-1}\} \cdot h_{ij}((1+am)^{-1}) \in G. \end{multline}
\end{comment}

We need to introduce notation for certain subgroups of $\St(A[X, X^{-1}])$:
\begin{align}
 G_+^0 & = \mathrm{Im}(\St(\Phi, A[X], XM[X]) \to \St(A[X, X^{-1}]))\\
 G_+   & = \mathrm{Im}(\St(\Phi, A[X], M[X]) \to \St(A[X, X^{-1}]))
\end{align}
Notice that for $ g\in G_+$ the element $ev_{X=0}(g)^{-1}g$ lies in $G_+^0$.

\begin{rem} Our goal is to rewrite $x_\alpha(aX^{-1}) \cdot z_\beta(f_1(X), f_2(X))$, where $f_1(X) \in M[X]$, $f_2(X) \in A[X]$ as
 $g \cdot x_\alpha(\frac{aX^{-1}}{1+am}) \cdot \{ 1 + am, X \}$ for some $g \in G_+$
 and $m \in M$. At first assume that $\Phi$ is simply-laced. \end{rem}
\begin{enumerate}
 \item {\it Case $\alpha \pm \beta \not \in \Phi \cup \{ 0\}$.} (Is this the same thing as $\alpha \perp \beta$?)
  Obviously, in this case $g = z_\beta(f_1(X), f_2(X))$, $m=0$.
 \begin{comment} 
 \item {\it Case $\alpha + \beta \in \Phi$, $f_1(X) = Xf_1'(X)$, $f_2(X) = 0$.}
 \[ x_\alpha(aX^{-1}) \cdot x_\beta(Xf_1'(X)) = x_{\alpha+\beta}(af_1'(X))x_\beta(Xf_1'(X)) x_\alpha(aX^{-1}) \]
 \end{comment}
 \item {\it Case $\alpha + \beta \in \Phi$, $f_1(X) = Xf_1'(X)$.}
 Start from this formula (centrality of $K_2$-paper):
 \begin{multline}
 x_\alpha(aX^{-1}) \cdot z_\beta(f_1(X), f_2(X)) = \\ = x_\alpha(af_1'(X)f_2(X)) \cdot x_{\alpha+\beta}(-N_{\beta, \alpha} af_1'(X)) \cdot z_\beta(f_1(X), f_2(X)) \cdot x_\alpha(aX^{-1}); 
 \end{multline}
 \item {\it Case $\beta - \alpha \in \Phi$, $f_1(X) = Xf_1'(X)$.}
 \begin{multline}
  x_{\alpha}(aX^{-1}) \cdot z_\beta(f_1(X), f_2(X)) = \\ = x_{\alpha}(-af_1'(X)f_2(X))\cdot x_{\alpha-\beta}(-N_{\beta,-\alpha} af_1'(X)f_2(X)^2) \cdot z_\beta(f_1(X), f_2(X)) \cdot x_{\alpha}(aX^{-1});
 \end{multline}
 \item {\it Case $\beta = -\alpha$, $f_1(X) = mX$, $f_2(X)=0$}. %CASE 1
For any $\alpha \in \Phi$ one can choose $\gamma$ such that $\langle \alpha, \gamma \rangle = -1$
(except in the sole case when $\alpha$ is a long root of a symplectic root system).
Now direct computation using~\eqref{eq:dennis-stein} and centrality of symbols shows that:
\begin{multline}
 x_\alpha(aX^{-1}) \cdot x_{-\alpha}(X m) %= x_\alpha(X^{\langle \alpha, \gamma \rangle}a) x_{-\alpha}(X^{-\langle \alpha, \gamma \rangle}m) 
 = {}^{h_\gamma(X)}(x_\alpha(a) x_{-\alpha}(m)) = \\
 = {}^{h_\gamma(X)}\left( x_{-\alpha}\left(\frac{m}{1+am}\right) \cdot \langle -a, m\rangle_\alpha \cdot h_\alpha(1+am) \cdot x_\alpha\left(\frac{a}{1+am}\right) \right) = \\
 x_{-\alpha}\left(\frac{mX}{1+am}\right) \cdot \langle -a, m\rangle_\alpha \cdot h_\alpha(X^{-1}(1+am))\cdot h_\alpha(X^{-1})^{-1} \cdot x_{\alpha}\left(\frac{aX^{-1}}{1+am}\right) = \\
 x_{-\alpha}\left(\frac{mX}{1+am}\right) \cdot \langle -a, m\rangle_\alpha \cdot \{1+am, X^{-1}\}^{-1} \cdot h_\alpha(1+am)\cdot x_{\alpha}\left(\frac{aX^{-1}}{1+am}\right) = \\
 = x_{-\alpha}\left(\frac{mX}{1+am}\right) \cdot \langle -a, m\rangle_\alpha \cdot h_\alpha(1+am) \cdot x_{\alpha}\left(\frac{aX^{-1}}{1+am}\right) \cdot \{1+am, X\}.
\end{multline}
 \item Essentially this allows one to cover the case $\beta = -\alpha$, $f_1(X) = Xm$ (with the same answer)
\end{enumerate}
 
\begin{lemma} Assume that $\Phi$ is simply-laced.
 For any $a\in A$, $\alpha\in \Phi$, $g_0 \in G^0_+$ there exist $m'\in m$ and $g'\in G_+$ such that 
 \[x_{\alpha}(aX^{-1}) g_0  = g' x_\alpha(-a(1+m')X^{-1}) \cdot \{X, 1+m'\}.\]
\end{lemma}
\begin{proof}
 Choose some parabolic subset $S$ of roots in $\Phi$ in such a way that its special part $\Sigma(S)$ does not contain the root $\alpha$.
 In view of~\cref{lem:Zgen} the subgroup $G_+^0$ is generated by images in $\St(\Phi, A[X, X^{-1}]$ of the elements of the set $\mathcal{Z}(\Sigma(S), A[X], Xm[X])$.
 
 Without loss of generality we may assume that $g_0$ is either $z_\beta(s, \xi)$ for some $\beta\in \Sigma(S)$ for some $s\in Xm[X]$, $\xi\in A[X]$ or $x_\gamma(s)$ for some $\gamma \in \Phi$.
 \begin{itemize}
  \item {\it Case $\alpha \perp \beta$. } In this case the required statement is obvious (with $m'=0$, $g'=g_0$).
  \item {\it Case $\alpha \not \perp \beta \in \Phi$.} In this case either $\alpha + \beta$ or $\alpha - \beta$ lies in $\Phi$.
  Assume, for example, the former (the argument in the other case is similar).
  The required assertion now follows from the following equality (cf. Sinchuk 16' paper Lemma 9):
  %\[ x_\beta(-aX^{-1}) z_\alpha(s, \xi) x_\beta(aX^{-1}) = x_\beta(-saX^{-1}\xi) \cdot x_{\alpha+\beta}(N_{\alpha,\beta}saX^{-1})\cdot z_\alpha(s, \xi).\]
  \[ x_\alpha(-aX^{-1}) z_\beta(s, \xi) =  \left( z_\beta(s, \xi) \cdot x_{\alpha+\beta}(N_{\beta,\alpha}saX^{-1}) \cdot x_\alpha(-s\xi aX^{-1}) \right) \cdot x_\alpha(-aX^{-1}).\]
  \item {\it Case $\alpha = \beta$}
  \item {\it Case $\alpha = -\beta$}
 \end{itemize}

\end{proof}

\section{Proof of Rehmann's presentation theorem}

\printbibliography
\end{document}
