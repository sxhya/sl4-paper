\documentclass[oneside, 8pt]{amsart}
\usepackage{amscd, amsmath, amssymb, amsthm, amsfonts, amstext, verbatim, mathtools, xfrac, microtype, nameref, thmtools}
\usepackage[breaklinks=false,unicode]{hyperref}
\usepackage[backend=biber, bibencoding=utf8, giveninits=true, citestyle=numeric-comp, sortlocale=en_US, url=false, doi=false, eprint=true, maxbibnames=4]{biblatex}
\usepackage[capitalize]{cleveref}
\usepackage[matrix,arrow,curve]{xy}
\usepackage{tikz}
\usepackage{enumitem}
%\usepackage[notref,notcite]{showkeys}

\addbibresource{paper.bib}
\renewbibmacro*{volume+number+eid}{\ifentrytype{article}{\- \iffieldundef{volume}{}{Vol.~\printfield{volume},}\iffieldundef{number}{}{ No.~\printfield{number},}}}
\renewbibmacro{in:}{\ifentrytype{article}{}{\printtext{\bibstring{in}\intitlepunct}}}
\newbibmacro{string+doi}[1]{\iffieldundef{doi}{\iffieldundef{url}{#1}{\href{\thefield{url}}{#1}}}{\href{https://dx.doi.org/\thefield{doi}}{#1}}}
\DeclareFieldFormat[article, inproceedings, inbook, book, online]{title}{\usebibmacro{string+doi}{\mkbibquote{#1}}}
\renewcommand*{\bibfont}{\footnotesize}

\newtheorem{theorem}{Theorem}
\newtheorem*{theorem*}{Theorem}
\newtheorem{lemma}{Lemma}
\newtheorem{prop}[lemma]{Proposition}
\newtheorem{corollary}[lemma]{Corollary}
\newtheorem{externaltheorem}[lemma]{Theorem}
\theoremstyle{remark} 


\theoremstyle{definition}
\numberwithin{lemma}{section}
\numberwithin{prop}{section}
\numberwithin{corollary}{section}
\numberwithin{externaltheorem}{section}
\newtheorem{df}[lemma]{Definition} \Crefname{df}{Definition}{Definitions}
\newtheorem{example}[lemma]{Example} \Crefname{example}{Example}{Examples}
\newtheorem{rem}[lemma]{Remark}
\newtheorem{cond}[lemma]{Condition}


\newlist{proplist}{enumerate}{1} \setlist[proplist]{label=(\roman{proplisti}), ref=\thethm.(\roman{proplisti}),noitemsep} \Crefname{proplisti}{Proposition}{Propositions}
\newlist{lemlist}{enumerate}{1} \setlist[lemlist]{label=(\roman{lemlisti}), ref=\thelem.(\roman{lemlisti}),noitemsep} \Crefname{lemlisti}{Lemma}{Lemmas}

\DeclareMathOperator{\Ker}{Ker}
\DeclareMathOperator{\Img}{Im}
\DeclareMathOperator{\St}{St}
\DeclareMathOperator{\GG}{G}
\DeclareMathOperator{\E}{E}
\DeclareMathOperator{\HH}{H}
\DeclareMathOperator{\K}{K}
\DeclareMathOperator{\KO}{KO}
\DeclareMathOperator{\GW}{GW}
\DeclareMathOperator{\colim}{colim}
\newcommand{\inv}{^{-1}}

\newcommand{\XX}{\mathcal{X}}           % (subset of) Steinberg generators
\newcommand{\RR}[1]{\mathcal{R}_{#1}}   % Steinberg relations
\newcommand{\catname}[1]{{\normalfont\textbf{#1}}} %Category name
\newcommand{\myol}[2][3]{{}\mkern#1mu\overline{\mkern-#1mu#2}}

\newcommand{\ZZ}{\mathbb{Z}}
\newcommand{\rA}{\mathsf{A}}
\newcommand{\rB}{\mathsf{B}}
\newcommand{\rC}{\mathsf{C}}
\newcommand{\rD}{\mathsf{D}}
\newcommand{\rE}{\mathsf{E}}
\newcommand{\rF}{\mathsf{F}}
\newcommand{\rG}{\mathsf{G}}

\numberwithin{equation}{section}

\title{A Horrocks-type theorem for even orthogonal $K_2$}
\keywords {Steinberg group, $K_2$-functor, Quillen--Suslin theorem, Horrocks theorem, $\mathbb{P}^1$--glueing. {\em Mathematical Subject Classification (2010):} 19C20}
\author{Andrei Lavrenov}
\address{Mathematisches Institut der Universit\"at M\"unchen, Theresienstr. 39, D-80333 M\"unchen}
\email{avlavrenov at gmail.com}

\author {Sergey Sinchuk}
\address{Chebyshev laboratory, St. Petersburg State University, St. Petersburg, Russia}
\email {sinchukss at gmail.com}
\date {\today}

\begin{document}
\begin{abstract} We prove a $\mathbb{P}^1$-glueing theorem for even-dimensional orthogonal Steinberg groups. This result transfers to the orthogonal case an earlier result of M.~Tulenbaev and is also an analogue of an earlier result of A.~Suslin and V.~Kopeiko. \end{abstract}
\maketitle
\section{Introduction}
Recall that the classical Serre problem on projective modules asks if any projective module over a polynomial ring $R = k[x_1,\ldots, x_n]$ over a field $k$ is free. Thшы problem was positively settled by D.~Quillen~\cite{Qu76} and A.~Suslin~\cite{Su76}, and its solution played an important role in the development of algebraic K-theory. We also refer the reader to the textbook~\cite{Lam10} for a comprehensive account on the problem, its history and the subsequent solution.

After the original Serre problem had been solved, numerous analogous questions drew the attention of specialists (see e.\,g.~\cite{Su77, Su82, Abe83, Tu83, Lam10, St-poly, St-Ded}). For example, A.~Suslin formulated and solved the so-called $K_1$-analogue of Serre's problem. This result asserts that the functor $\K_{1,n}(R) = \mathrm{GL}_{n}(R)/\E_n(R)$ has the property $\K_{1,n}(k[x_1, \ldots x_n]) = \K_{1,n}(k) = k^\times$ for all fields $k$ and $n \geq 3$ see~\cite[Corollary~7.11]{Su77}. Suslin's results were subsequently generalized to $\K_1$-functors modeled on other linear groups (see the definition below). For example, for even-dimensional orthogonal groups the corresponding result was obtained by A.~Suslin and V. Kopeiko in~\cite{Su82}, while for more general types of Chevalley groups of rank $\geq 2$ this is a result of E.~Abe, see~\cite{Abe83}. Recently A.~Stavrova has obtained probably the most general results in this direction: she solved the analogue of Serre problem for the functor $\K_1^G$ modeled on arbitary isotropic reductive group scheme $G$ of isotropic rank $\geq 2$ over a field (see~\cite[Theorem~1.2]{St-poly}) and also generalized Abe's result to Dedekind domains of arithmetic type (see~\cite[Corollary~1.2]{St-Ded}).

Recall that to every irreducible root system $\Phi$ and a commutative ring $R$ one can associate two groups: the {\it simply-connected Chevalley group} $G(\Phi, R)$ (see~\cite[\S~3]{St71} or~\cite{VP}) and the {\it Steinberg group} $\St(\Phi, R)$ (see~\cref{sec:Steinberg-intro} for the definition). There is a well-defined map $\pi \colon \St(\Phi, R) \to G(\Phi, R)$ sending each generator $x_\alpha(\xi)$ to the elementary root unipotent $t_\alpha(\xi)$. The cokernel and the kernel of this map are, by definition, the $\K_1(\Phi, R)$ and $\K_2(\Phi, R)$-functors modeled on Chevalley group $G(\Phi, -)$, see~\cite{St78}.

It turns out that an assertion similar to Serre problem also holds for the functor $\K_2$. More precisely, in~\cite{Tu83} M. Tulenbaev demonstrated an ``early stability theorem'' from which the isomorphism $\K_2(\rA_\ell, k[x_1, \ldots x_n]) \cong \K_2(\rA_\ell, k) = \K^\mathrm{M}_2(k)$ follows for $\ell \geq 4$. Notice that $\K_2(\rA_\ell, R)$ here is just another notation for the unstable linear functor $\K_2(\ell+1, R)$.

While numerous results on the $\K_1$-analogue of Serre's problem have appeared in the literature since~\cite{Su77} (see e.\,g.~\cite{Su82, Abe83, St-poly, St-Ded}), little progress has been made on the $\K_2$-analogue. It has been conjectured by M.~Wendt, see~\cite[Vermutung~6.22]{Vo11} that the $\K_2$-analogue of Serre problem holds for $\K_2(\Phi, -)$ for all $\Phi$ of rank $\geq 3$, however this conjecture still remains open for $\Phi$ different from $\rA_\ell$, $\ell \geq 4$.

In~\cite{LS17} the authors have shown that Steinberg groups $\St(\Phi, R)$ satisfy Quillen--Suslin local-global principle provided $\Phi$ is simply-laced and has rank $\geq 3$. This result is one of the ingredients needed in the proof of the $\K_2$-analogue of Serre problem. The aim of the present article is to make yet another step towards its solution, namely to prove an analogue of Horrocks theorem~\cite{Ho64} for Steinberg groups.

Our main result is, thus, the following theorem, which is the orthogonal analogue of~\cite[Theorem~5.1]{Tu83} and the $\K_2$-analogue of~\cite[Theorem~6.8]{Su82} (cf. also with~\cite[Theorem~VI.5.2]{Lam10} and~\cite[Theorem~1.1]{St-poly}).
\begin{theorem}[Horrocks theorem for orthogonal $\K_2$]\label{thm:main} Let $A$ be a commutative ring in which $2$ is invertible. Then for any $\ell \geq 7$ the following commutative square is a pullback square in which all maps are injective:
\begin{equation} \nonumber
  \xymatrix{ \KO_2(2\ell, A) \ar[r] \ar[d] & \KO_2(2\ell, A[X]) \ar[d] \\ \KO_2(2\ell, A[X\inv]) \ar[r] & \KO_2(2\ell, A[X, X\inv]).}  
\end{equation} 
Moreover, the same assertion holds if one replaces the functor $\KO_2(2\ell, -)$ with $\K_2(\rD_\ell, -)$ or $\St(\rD_\ell, -)$. \end{theorem}
In the above statement $\KO_2(2\ell, -)$ denotes the unstable orthogonal $\K_2$-functor (see~\cref{sec:quillen}).

The proof of~\cref{thm:main} goes as follows. We notice that it suffices to prove the $\St(\rD_\ell, -)$-variant of the theorem. Moreover, the proof of the injectivity of
$j_+ \colon \St(\rD_\ell, A[X]) \to \St(\rD_\ell, A[X, X\inv])$ turns out to be the hardest part. After invoking the local-global principle~\cite[Theorem~2]{LS17} the proof reduces to the special case when $A$ is local.
Now if $M$ is the maximal ideal of $A$, the proof of the injectivity of $j_+$ comes down to proving injectivity of the following two maps:
\[ \xymatrix{ \St(\rD_\ell, A[X^{-1}]) \ar[r] & \St(\rD_\ell, A[X\inv] + M[X]) \ar[r] & \St(\rD_\ell, A[X, X\inv]). }\]
The injectivity of the second map is obtained in~\cref{thm41}. This is the only place in our proof which invokes the assumption that $2$ is invertible. The proof of~\cref{thm41} depends on the stability theorem for higher orthogonal K-groups and also on certain basic computations with Grothendieck--Witt groups (these groups include the stable orthogonal K-groups as a special case, see~\eqref{GW-concrete}). Finally, the injectivity of the first map is obtained in~\cref{thm:P1glueing}, which is a direct generalization of~\cite[Proposition~4.3]{Tu83}. This part of the proof is obtained in somewhat greater generality and is applicable to all simply-laced root systems $\Phi$ containing a subsystem of type $\rA_4$.
%$\rD_\ell$, $\ell=5,6$ and also $\rE_\ell$, $\ell=6,7,8$.

\subsection{Acknowledgements} 
The authors of this paper were also supported by ``Native towns'', a social investment program of PJSC ``Gazprom Neft''.
The second-named author also acknowledges the financial support of RFBR grant No 18-31-20044.

The authors would like to express their gratitude to A.~Stavrova, A.~Stepanov and N.~Vavilov for their useful comments and interest in this work.

\section{Preliminaries}
\subsection{Steinberg groups} \label{sec:Steinberg-intro}
Let $\Phi$ be a reduced and irreducible root system of rank $\geq 2$ and $R$ be a commutative ring with $1$. Recall that in this case the \emph{Steinberg group} $\St(\Phi, R)$ can be defined by means of generators $x_{\alpha}(s)$ and relations:
\begin{align}
& \phantom{[}
x_\alpha(s) \cdot x_\alpha(t) = x_\alpha(s+t),\ \alpha\in\Phi,\ s,t\in R; \label{rel:add}\\
& [x_\alpha(s), x_\beta(t)] = \prod
 x_{i\alpha + j\beta}\left(N_{\alpha,\beta ij}\, s^i t^j\right),\quad \alpha,\beta\in\Phi,\ \alpha\neq-\beta,\ s,t\in R. \label{rel:CCF}
\end{align}
The indices $i$, $j$ appearing in the right-hand side of the above relation range over
all positive natural numbers such that $i\alpha + j\beta\in\Phi$.
The constants $N_{\alpha \beta i j}$ appearing in the right-hand side of \eqref{rel:CCF} are integers equal to $\pm 1,\pm 2,\pm 3$, they are called the {\it structure constants} of the Chevalley group $G(\Phi, R)$. Several different methods of computing signs of these constants have been proposed in the literature, see e.\,g.~\cite{V00}, \cite[\S~9]{VP}. 

For an additive subgroup $A\subseteq R$ and $\alpha \in \Phi$ we denote by $X_\alpha(A)$ the corresponding {\it root subgroup} of $\St(\Phi, R)$, i.\,e. the subgroup generated by all $x_\alpha(a)$, $a \in A$.

Whenever we speak of the Steinberg group $\St(\Psi, R)$ parametrized by a root subsystem $\Psi \subset \Phi$ we imply that the choice of structure constants 
 for $\St(\Psi, R)$ is compatible with that for $\St(\Phi, R)$ (i.\,e. the mapping $x_\alpha(\xi) \mapsto x_\alpha(\xi)$ yields a group homomorphism $\St(\Psi, R) \to \St(\Phi, R)$).

In this paper we will be mostly interested in the case when the Dynkin diagram of $\Phi$ is {\it simply-laced}, i.\,e. does not contain double bonds. In this case the defining relations of $\St(\Phi, R)$ have the following simpler form:
\begin{align}
&x_{\alpha}(a)\cdot x_{\alpha}(b)=x_{\alpha}(a+b), \tag{R1}\\
&[x_{\alpha}(a),\,x_{\beta}(b)]=x_{\alpha+\beta}(N_{\alpha\beta} \cdot ab),\text{ for }\alpha+\beta\in\Phi, \tag{R2} \\
&[x_{\alpha}(a),\,x_{\beta}(b)]=1,\text{ for }\alpha+\beta\not\in\Phi\cup0. \tag{R3}
\end{align}
In the above formulae $a, b \in R$ and the integers $N_{\alpha, \beta} = N_{\alpha, \beta, 1, 1} = \pm 1$ are the structure constants of the Lie algebra of type $\Phi$. Although there is still some degree of freedom in their choice, they always must satisfy the relations, indicated in the following lemma (cf. \cite[\S~14]{VP}).
\begin{lemma} Suppose $\Phi$ is simply laced and $\alpha, \beta$ are roots of $\Phi$ such that $\alpha+\beta\in \Phi$, then holds
\begin{equation} \label{eq:simplest} N_{\alpha, \beta} = -N_{\beta,\alpha} = - N_{-\alpha, -\beta} = N_{\beta, -\alpha-\beta} = N_{-\alpha-\beta, \alpha}. \end{equation}
If, moreover, $\gamma \in \Phi$ is such that $\alpha,\beta,\gamma$ form a basis of a root subsystem of type $\rA_3$ then one has
\begin{equation} \label{eq:cocycle} N_{\beta,\gamma} \cdot N_{\alpha, \beta+\gamma} = N_{\alpha+\beta, \gamma} \cdot N_{\alpha, \beta}. \end{equation} \end{lemma}
In our computations below we will be using identities~\eqref{eq:simplest} without further reference.

For $\alpha\in\Phi$ and $s \in R^\times$ we define certain elements $w_\alpha(s), h_\alpha(s)$ of $\St(\Phi, R)$ (the latter ones are sometimes called {\it semisimple root elements}):
\begin{align*} w_\alpha(s) & =  x_\alpha(s) \cdot x_{-\alpha}(-s^{-1}) \cdot x_\alpha(s), \\ h_\alpha(s) & =  w_\alpha(s) \cdot w_\alpha(-1).  \end{align*}

Recall from~\cite[Lemma~5.2]{Ma69} that the following relations hold for semisimple root elements:
\begin{align} \label{eq:conj-h-x} {}^{h_\alpha(t)}\!x_\beta(u) & = x_\beta(t^{\langle \beta,  \alpha \rangle}u), \\ \label{eq:conj-h-h} {}^{h_\alpha(t)}\!h_\beta(u) & = h_\beta(t^{\langle \beta, \alpha \rangle} \cdot u) \cdot h_\beta(t^{\langle \beta,  \alpha \rangle})^{-1}. \end{align}
Here $\langle \beta, \alpha \rangle$ denotes the integer $\tfrac{2(\beta, \alpha)}{(\alpha, \alpha)}$.

\subsection{$\K_2$-groups and symbols} In our computations we use two families of explicit elements of $\K_2(\Phi, R)$ called {\it Steinberg and Dennis--Stein symbols}. Notice that our notational conventions for symbols follow~\cite{DS73} and {\it not} more modern textbooks such as~\cite{Kbook}. Recall that Steinberg symbols are defined for arbitrary $s, t \in R^\times$ as follows:
\begin{equation} \label{eq:steinberg} \{ s, t \}_\alpha = h_\alpha(st) \cdot h_\alpha^{-1}(s) \cdot h_\alpha^{-1}(t). \end{equation}
In turn, Dennis--Stein symbols are defined for arbitrary $a, b\in R$ satisfying $1 + ab \in R^\times$:
\begin{equation} \label{eq:dennis-stein}  \langle a,b \rangle _ \alpha = x_{-\alpha}\left(\tfrac{- b}{1 + ab}\right) \cdot x_{\alpha}(a) \cdot x_{-\alpha}(b) \cdot x_{\alpha}\left(\tfrac{- a}{1+ab}\right) \cdot h_{\alpha}^{-1}(1 + ab). \end{equation} 
Dennis--Stein symbol $\langle a, b \rangle_\alpha$ can be expressed through Steinberg symbols in the special case when either $a$ or $b$ is an invertible element of $R$. More specifically, the following formulae hold (cf.~\cite[p.~250]{DS73}).
\begin{equation} \label{DS-S-relationship} \langle a, b \rangle_\alpha = \{-a, 1+ab\}_\alpha\text{ for } a, 1+ab\in R^\times,\ \
 \{ s, t \}_\alpha = \left\langle -s, \tfrac{1 - t}{s} \right\rangle_\alpha\text{ for } s, t\in R^\times. \end{equation}
Steinberg and Dennis--Stein symbols depend only on the length of $\alpha$, in particular they do not depend on $\alpha$ if $\Phi$ is simply-laced. If $\Phi$ happens to be {\it nonsymplectic}, i.\,e. $\Phi \neq \rA_1, \rB_2, \rC_{\geq 3}$, Steinberg symbols are antisymmetric and bimultiplicative, i.\,e. they satisfy the following identities: \begin{equation} \label{eq:symbol-properties} \{ u, st \} = \{ u, s\} \{ u, t \}, \ \{ u, v \} = \{ v, u\}^{-1}. \end{equation}
For these and other properties of symbols we refer the reader to~\cite{DS73}.

Recall that the classical Matsumoto theorem (see~\cite[Theorem~5.10]{Ma69}) allows one to compute the group $\K_2(\Phi, R)$ in the special case when $R=k$ is a field.
Using the modern language of Milnor--Witt K-theory (see~\cite{Mo04}) it can be formulated as the following computation:
\[\K_2(\Phi, k) = \left\{\begin{array}{ll} \K_2^\mathrm{MW}(k)& \text{if $\Phi$ is symplectic,}\\ \K_2^\mathrm{M}(k) & \text{otherwise.}\end{array}\right. \]
In the following lemma we recall the computation of the group $\K_2(\Phi, R)$ in the case $R=k[X, X\inv]$.
\begin{lemma}[Hurrelbrink--Morita--Rehmann]\label{K2-laurent-field} Let $\Phi$ be a reduced irreducible root system of type $\neq \rG_2$ and $k$ be arbitrary field. Then there is a split exact sequence of abelian groups
\[ \xymatrix{ 0 \ar[r] & \K_2(\Phi, k) \ar[r] & \K_2(\Phi, k[X, X^{-1}]) \ar[r] & H(\Phi, k) \ar[r] & 0,}\text{ in which} \]
\[ H(\Phi, k) = \left\{\begin{array}{ll} \K_1^\mathrm{MW}(k)& \text{if $\Phi$ is symplectic,}\\ \K_1(k) \cong k^\times & \text{otherwise.}  \end{array}\right. \]  \end{lemma}
\begin{proof} Let us first consider the case of nonsymplectic $\Phi$, in which one can find a long root $\alpha\in \Phi$ in such a way that there is a commutative diagram of abelian groups
\[\xymatrixcolsep{5pc}\xymatrix{k^\times \ar[r]^-{h} \ar[rrd]_{\{-, X\}} & \K_2(\Phi, k[X, X^{-1}]) \ar[r] & \K_2(\Phi, k(X)) \ar[d]^{\cong} \\
                                                                                      &                                 & \K_2^\mathrm{M}(k(X)),} \]
in which $h = \{ -, X \}_{\alpha}$ and the vertical map is an isomorphism by Matsumoto theorem.
Notice that the diagonal map is split by the obvious residue homomorphism and therefore is injective. This, in turn, implies that $h$ is also injective.
The assertion of the lemma now follows from~\cite[Satz~3]{Hur77} which asserts that for a nonsymplectic $\Phi$ holds $\K_2(\Phi, k[X, X\inv]) = \mathrm{Im}(h) \oplus \K_2(\Phi, k)$.

Consider now the case when $\Phi$ is symplectic. In this case the assertion of the lemma is just a reformulation of~\cite[Theorem~B]{MR91},
which asserts that for $\ell \geq 1$ one has $\K_2(\rC_\ell, k[X, X\inv]) \cong \K_2(\rC_\ell, k) \oplus P(k)$, where
$P(k)$ is the set $k^\times \times I_2(k)$ with the group structure given by
\[ (u, y) \cdot (v, z) = (uv, y + z - \langle\langle u, v\rangle\rangle).\]
Here $I^2(k)$ stands for the second power of the fundamental ideal $I(k)$ in the Witt ring $W(k)$ of $k$.
Recall from~\cite{Mo04} that $\K_1^\mathrm{MW}(k)$ is isomorphic to the pullback of the diagram:
\[ \xymatrix{ \K_1^\mathrm{MW}(k) \ar[r] \ar[d] & I(k) \ar[d] \\ \K_1(k) \ar[r] & I(k)/I^2(k), } \]
in other words, it consists of pairs $[u, x]$ such that $x - \langle \langle u \rangle \rangle \in I^2(k)$.
It is easy to verify that the map $[u, x] \mapsto (u, \langle\langle u \rangle\rangle - x)$ defines an isomorphism of $\K_1^\mathrm{MW}(k)$ and $P(k)$. \end{proof}

\begin{rem} The above lemma can be considered as the unstable version of~\cite[Lemma~4.1.1]{AF17} (compute~$\mathrm{KSp}_2(k[X, X\inv])$ using~\eqref{GW-concrete} and~\cref{bass-ft} below). \end{rem}

\begin{lemma} \label{field-injectivity} For $\Phi\neq\rG_2$ the map $\St(\Phi, k[X]) \to \St(\Phi, k[X, X^{-1}])$ is injective. Moreover, the intersection of the images of $\St(\Phi, k[X])$ and $\St(\Phi, k[X\inv])$ inside $\St(\Phi, k[X, X\inv])$ coincides with the image of $\St(\Phi, k)$. \end{lemma}
\begin{proof} The first assertion follows from consideration of the following commutative diagram:
\[\xymatrix{ & \K_2(\Phi, k) \ar[dl]_{\cong} \ar@{^{(}->}[dr] & \\
               \K_2(\Phi, k[X]) \ar[rr] &               & \K_2(\Phi, k[X, X^{-1}]),} \]
in which the left arrow is an isomorphism by the Korollar of~\cite[Satz~1]{Re75} and the right arrow is split injective by~\cref{K2-laurent-field}.

Let us verify the second assertion. Let $g$ be an element of the intersection of $\St(\Phi, k[X])$ and $\St(\Phi, k[X\inv])$ inside $\St(\Phi, k[X, X\inv])$.
Clearly, the image of $g$ in $G(\Phi, k[X, X\inv])$ lies in $G(\Phi, k)$, therefore there exists $g_0 \in \St(\Phi, k)$ such that $gg_0^{-1} \in \K_2(\Phi, k[t]) = \K_2(\Phi, k)$. Thus, we conclude that $g \in \St(\Phi, k)$.
\end{proof}

\subsection{Relative Steinberg groups and unstable K-groups} \label{sec:quillen}
In this subsection we recall the definitions and basic facts pertaining to the theory of relative central extensions developed by J.-L.~Loday in~\cite{Lo78}. The main goal of this subsection is to show that Loday's theory can be applied to unstable Steinberg groups, and that the resulting relative unstable Steinberg groups have many of the properties of their stable counterparts. Some of the results of this subsection have been briefly mentioned in~\cite{S15} (cf. e.\,g. Corollaries 3--4).

Recall that the category of (commutative) pairs $\catname{Pairs}$ is defined as follows.
Its objects are pairs $(R, I)$, in which $R$ is a commutative ring and $I$ is an ideal of $R$. A morphism of pairs $f \colon (R, I) \to (R', I')$ is, by definition, a ring map $f \colon R \to R'$ such that $f(I) \subseteq I'$. Notice that the mapping $(R, I) \mapsto (R \to R/I)$ defines a functor from $\catname{Pairs}$ to the morphism category $\catname{CRings}^\rightarrow$.
If $(R, I)$ is such that $R$ is a local ring with maximal ideal $I$, we call such pair a {\it local pair}.

There is an obvious fully faithful embedding $\catname{CRings} \to \catname{Pairs}$ sending $R$ to $(R, R)$. For a given functor $S \colon \catname{CRings} \to \catname{Groups}$ a {\it relativization} of $S$ is any functor $\widetilde{S} \colon \catname{Pairs} \to \catname{Groups}$ extending $S$ in the obvious sense. Relativization of a functor is not unique.

Recall that {\it the double ring} $D_{R, I}$ of a pair $(R, I)$ is, by definition, the pullback ring $R \times_{R/I} R$. In other words, it is the ring consisting of pairs of elements of $R$ congruent modulo $I$. Denote by $p_0, p_1, \Delta$ the two obvious projections and the diagonal map $\xymatrix{D_{R,I} \ar@<1pt>[r] \ar@<-2.5pt>[r] & \ar@<-4pt>[l] R}$. It is clear that $p_0 \Delta = p_1 \Delta = id_{R}$.

Let $S \colon \catname{CRings} \to \catname{Groups}$ be a functor. Set $G_i = \Ker(S(p_i))$ and define {\it Loday's relativization} $S(R, I)$ as $ G_0 / [G_0, G_1]$. The map $S(p_1)$ induces a natural transformation $S(R, I) \to S(R)$. We denote this map by $\mu = \mu_{R,I}$ and its kernel by $C_S(R, I)$: \begin{equation} \label{LodayRelativization} \xymatrix{ 1 \ar[r] & C_S(R, I) \ar[r] & S(R, I) \ar^{\mu}[r] & S(R) \ar[r] & S(R/I) \ar[r] & 1.} \end{equation}

\begin{df} By definition, the {\it relative Steinberg group} $\St(\Phi, R, I)$ is the result of application of Loday's relativization to the functor $\St(\Phi, -)$. Notice that $\St(\Phi, R, I)$ is not a subgroup of $\St(\Phi, R)$ but rather its central extension by the abelian group $C_{\St(\Phi, -)}(R, I)$. For shortness we rename the latter group to $C(\Phi, R, I)$. \end{df}

Our next goal is to obtain a homological interpretation of the group $C(\Phi, R, I)$.
In order to do this, we need to recall some additional notation and terminology.

First of all, recall that a {\it central extension} of a group $G$ is a surjective map $\widetilde{G} \to G$, whose kernel is contained in the center of $\widetilde{G}$. 
A morphism of central extensions is a group-theoretic map $\widetilde{G} \to \widetilde{G}'$ over $G$.
A central extension is said to be {\it universal} if it is an initial object of the category of central extensions of $G$.

Recall that a {\it crossed module} is a triple $(M, N, \mu)$ consisting of a group $N$ acting on itself by left conjugation, an $N$-group $M$ and a map
 $\mu \colon M\to N$ of $N$-groups satisfying {\it Peiffer identity} $\mu(m) \cdot m' = m m' m^{-1}$. It can be shown that the image of $\mu$ is always
  a normal subgroup of $N$ and that the kernel of $\mu$, which we denote by $L$, is always contained in the center of $M$.
 
Let $\nu \colon N \twoheadrightarrow Q$ be a surjective group-theoretic map.
A {\it relative central extension of $\nu$} is, by definition,
a crossed module $(M, N, \mu)$ such that the cokernel of $\mu$ is $\nu$:
\begin{equation} \label{RelativeCentralExtension}
 \xymatrix{ 1 \ar[r] & L \ar[r] & M \ar^{\mu}[r] & N \ar^{\nu}[r] & Q \ar[r] & 1} \end{equation}

A morphism $(M, \mu) \to (M', \mu')$ of two relative central extensions of $\nu$ is, by definition, an $N$-group homomorphism $f\colon M \to M'$ such that $\mu' f = \mu$. 
A relative central extension is said to be {\it universal} if it is an initial object of the category of relative central extensions of $\nu$. 

It turns out that the set $Ext(Q, N; L)$ of isomorphism classes of relative central extensions of $\nu$ by an abelian group $L$ can be classified by means of a certain cohomological invariant called {\it characteristic class}. More precisely, \cite[Th{\'e}or{\`e}me~1]{Lo78} asserts that there is a well-defined bijection $\xi \colon Ext(Q, N; L) \to \HH^3(Q, N; L)$.
 
For the rest of this subsection $S \xrightarrow{\pi} P \subseteq G$ is a triple of group-valued functors on the category of commutative rings satisfying the following assumptions:
\begin{enumerate} [label=(A\arabic*)]
 \item \label{req:left-exact} $G(D_{R, I}) \cong G(R) \times_{G(R/I)} G(R)$.
 \item \label{req:coeq} For every pair $(R, I)$ the coequalizer of $S(p_0), S(p_1)$ is precisely $S(R) \to S(R/I)$.
 \item \label{req:subfunc} $P(R)$ is a perfect normal subgroup of $G(R)$.
 \item \label{req:uce} The map $ \pi_R \colon S(R) \to P(R)$ is a universal central extension for all $R$. In particular, $\HH_1(S(R), \ZZ) = \HH_2(S(R), \ZZ) = 0$.
\end{enumerate}

\begin{lemma}\label{lem:relativeH3}
 For every pair $(R, I)$ the map $\mu \colon S(R, I) \to S(R)$ is a universal relative central extension of $\nu \colon S(R) \to S(R/I)$. The group $C_S(R, I)$ is naturally isomorphic to the relative homology group $\HH_3(S(R), S(R/I), \ZZ)$.
\end{lemma}
\begin{proof}
The action of $S(R)$ on $S(D_{R, I})$ given by ${}^g h = S(\Delta)(g) \cdot h \cdot S(\Delta)(g)^{-1}$ induces an action of $S(R)$ on $S(R, I)$.
The map $\mu \colon S(R, I) \to S(R)$ from~\eqref{LodayRelativization} is an $S(R)$-map with respect to this action.
From~\ref{req:left-exact} and $\varphi(G_i) \subseteq \Ker(G(p_i))$ we obtain that 
$G_0 \cap G_1 \subseteq \Ker(\pi_{D_{R, I}})$ hence it is a central subgroup of $S(D_{R,I})$ by~\ref{req:uce}. Thus, we have verified the assumptions of~\cite[Proposition~6]{Lo78} which asserts that the map $\mu$ is a universal relative central extension of the coequalizer $\nu = \mathrm{coeq}(d_0, d_1)$. Since $\nu$ coincides with $S(R) \to S(R/I)$ by~\ref{req:coeq}, we have completed the proof of the first assertion of the lemma.

Set $N = S(R)$, $Q = S(R/I)$, $C = \HH_3(Q, N; \ZZ)$. Recall from the proof of~\cite[Th{\'e}or{\`e}me~2]{Lo78} that to every relative central extension $(M, \mu)$ of $\nu$ with kernel $L$ one can associate a map of abelian groups $C \to L$. This map is obtained from the characteristic class $\xi(M, \mu)$ via the isomorphism $\HH^3(Q, N; L) \cong \mathrm{Hom}(C, L)$ of the universal coefficients theorem.

In the special case $M = S(R, I)$ this construction produces a map $C \to C_{S}(R, I)$ whose naturality in $(R, I)$ follows from~\cite[Proposition~3]{Lo78}. 
This map is an isomorphism by~\cite[Th{\'e}or{\`e}me~2]{Lo78}. \end{proof}

We retain our notation for the functors $S, P$ and $G$.
For $i\geq 1$ we define the unstable Quillen K-functors $\K_{i}^{G, P}$ via
\begin{equation} \label{plus-constr} \K_i^{G,P}(R) = \pi_i(BG(R)^+_{P(R)}). \end{equation}
It is not hard to obtain the following concrete description of these functors in the cases $i=1,2,3$.
\begin{lemma}\label{lem:lowerKgroups} There are natural isomorphisms \begin{enumerate}
 \item $\K_1^{G,P}(R) \cong G(R) / P(R)$;
 \item $\K_2^{G,P}(R) \cong \Ker(S(R) \to G(R))$;
 \item $\K_3^{G,P}(R) \cong \HH_3(S(R), \ZZ).$ \end{enumerate} \end{lemma}
\begin{proof} The first claim is obvious, the second and the third claim follow from~\ref{req:subfunc} and~\ref{req:uce} using the standard properties of the plus-construction, see \cite[\S~IV.1]{Kbook}   (cf. Exercises~1.8--1.9 ibid.) \end{proof}

Now let us give an example of the triple $(G, P, S)$ playing a key role in the present paper.
We denote by $\mathrm{O}_{2n}(R)$ and $\mathrm{EO}_{2n}(R)$ the orthogonal group of rank $n$ over a ring $R$ and its elementary subgroup, respectively
 (see e.\,g.~\cite{Su82} for the definition of these groups).
Now set $G_n = \mathrm{O}_{2n}(-)$, $P_n = \mathrm{EO}_{2n}(R)$, $S_n = \St(\rD_n, -)$ and let $\pi$ be the obvious projection $S_n \to P_n$.

It is easy to see that functors $G_n$, $P_n$ and $S_n$ satisfy \ref{req:left-exact} and \ref{req:coeq}.
By~\cite{Su82} the requirement~\ref{req:subfunc} is also satisfied for $n \geq 3$.
Finally, from~\cite[Corollary~5.4]{St71} and~\cite[Theorem~1]{LS17} it follows that $S_n$ and $P_n$ satisfy~\ref{req:uce} for $n \geq 5$.
We use the notation $\KO_i(2n, R)$ as a shorthand for $\K_i^{G_n, P_n}(R)$. 

Notice that $\KO_2(2n, R) = \Ker(\St(\rD_\ell, R) \to \mathrm{SO}(2n, R))$ contains $\K_2(\rD_\ell, R)$, but the converse is not generally true. On the other hand, from~\cref{lem:lowerKgroups} it follows that $\KO_3(2n, R)$ is isomorphic to the group $K_3^{G, P}$ in which $G = G(\rD_n, -) = \mathrm{Spin}(2n, -)$ and $P = \mathrm{Epin}(2n, -)$ is its elementary subfunctor.

We conclude this subsection with the following stability result (see~\cite[Theorem~9.4]{Pa89}).
\begin{externaltheorem}[Panin] \label{Panin-stability}
 Let $R$ be either a field, principal ideal domain or a Dedekind domain. Set $a = 1,2$ or $3$ in each of these three cases, respectively.
 Then the stability map $\KO_i(2n, R) \to \KO_i(2(n+1), R)$ is an epimorphism for $n \geq b$ 
 and an isomorphism for $n \geq b + 1$, where $b = \mathrm{max}(2i, a+i-1)$. \end{externaltheorem}

\section{An injectivity theorem for Steinberg groups} \label{firstPart}
We start this section by recalling basic notation and facts pertaining to the theory of Grothendieck--Witt groups. Recall that this theory, developed by M.~Schlichting, is a modern broad generalization of the classical hermitian K-theory of rings. We refer the reader to~\cite[\S~2]{FRS12} and~\cite[\S~2]{AF17} for an introduction to Grothendieck--Witt groups.

For our purposes it suffices to restrict attention to the affine case, in which  the Grothendieck--Witt groups $\GW_i^{[k]}(R)$ for $i \geq 1,\ [k] \in \ZZ/4\ZZ$ can be considered simply as a shorthand for the following 4 groups:
\begin{equation} \label{GW-concrete} \GW_i^{[k]}(R) = \left\{\begin{array}{ll} \KO_i(R), & k = 0 \\ U_i(R), & k = 1 \\ \mathrm{KSp}_i(R), & k = 2 \\ {}_{-1}\!U_i(R), & k = 3. \end{array}\right. \end{equation}
Here $\mathrm{KO}_i(R)$ denotes the usual orthogonal K-group defined via~\eqref{plus-constr} with $G(R) = O_\infty(R)$ and $P(R) = [G(R), G(R)]$.
Replacing the stable orthogonal group with the stable symplectic group one can also define the symplectic K-groups $\mathrm{KSp}_i(R)$.
We refer the reader to~\cite{Ka80} for the definition and properties of the groups ${}_{\pm 1}\!U_i(R)$. We will not use these definitions directly.

The following result, which is a special case of~\cite[Theorem~9.13]{Sch16} of M.~Schlichting, plays a key role in the proof of~\cref{thm41}.
\begin{externaltheorem}[Bass Fundamental Theorem]\label{bass-ft} Suppose that $R$ is a regular ring such that $2 \in R^\times$, 
then for any $i\geq 1$, $k\in \ZZ/4\ZZ$ there is a natural split exact sequence of abelian groups \[ \xymatrix{ 0 \ar[r] & \GW_i^{[k]}(R) \ar[r] & \GW_i^{[k]}(R[X, X^{-1}]) \ar[r]  & \GW_{i-1}^{[k-1]}(R) \ar[r] & 0.} \] \end{externaltheorem}
We will need only the special case $k=0$ of the above theorem, in which case it turns into an earlier result of J.~Hornbostel, see~\cite[Corollary~5.3]{Ho05}.

For the rest of this section let us fix the following notation.
Let $A$ be arbitrary commutative local ring with maximal ideal $M$ and residue field $k$.
Denote by $B = B_{A, M}$ the subring $A[X^{-1}] + M[X]$ of the ring $R = A[X, X^{-1}]$ and
by $I$ the ideal $M[X, X^{-1}]$ of $B$ (it is clear that $I$ is also an ideal of $R$).

\begin{lemma} \label{lem:prop41}
Assume additionally that the residue field $k$ is of characteristic $\neq 2$.
Then the canonical map $f\colon C(\rD_\ell, B, I) \to C(\rD_\ell, R, I)$ is surjective for $\ell \geq 7$. \end{lemma}
\begin{proof}
Writing the starting portion of the homology long exact sequence for the map $\St(\rD_\ell, R) \to \St(\rD_\ell, R/I)$ 
and using the isomorphisms of~\cref{lem:relativeH3} and~\cref{lem:lowerKgroups} we obtain the following commutative diagram:
\begin{equation*}\xymatrix{
 \KO_3(2\ell, B) \ar[r] \ar[d] & \KO_3(2\ell, k[X]) \ar[d]_{f'} \ar@{->>}[r] & \ar[d]_{f} C(\rD_\ell, B, I) \\
 \KO_3(2\ell, R) \ar[r]        & \KO_3(2\ell, k[X, X^{-1}]) \ar@{->>}[r]        & C(\rD_\ell, R, I).}\end{equation*}
By~\cref{Panin-stability} theorem the map $f'$ can be identified with the canonical map $\GW_3^{[0]}(k[X]) \to \GW_3^{[0]}(k[X, X^{-1}])$.
By~\cref{bass-ft} $\GW_3^{[0]}(k[X, X^{-1}]) \cong \GW_3^{[0]}(k) \oplus \GW_2^{[3]}(k),$
but since the group $\GW_2^{[3]}(k)$ is trivial by~\cite[Lemma~2.2]{FRS12}, the map $f'$ (and hence $f$) is surjective.
\end{proof} 
 
We will also need the following property of relative Steinberg groups which is a special case of a more general property discussed in~\cite[\S~2]{LS17}.
\begin{lemma}\label{lem:lemma32} Let $\Phi$ be a simply-laced root system of rank $\geq 3$,
Consider the following commutative square of canonical maps.
\[ \xymatrix{
    \St(\Phi, B, I) \ar[r] \ar[d] & \St(\Phi, B) \ar[d] \\
    \St(\Phi, R, I) \ar[r] \ar@{-->}^t[ur] & \St(\Phi, R) } \]
Then there exists a diagonal map $t$ which makes the diagram commute.   
\end{lemma} 
\begin{proof}
 Notice that $R$ is isomorphic to the principal localisation of $B$ at $X$
  and that $I$ is uniquely $X$-divisible in the sense of~\cite[\S~4]{LS17}.
 Thus, in the special case $\Phi = \rA_3$ the assertion of the lemma follows from~\cite[Theorem~3]{LS17}.
 In the general case the assertion of the lemma is a corollary of amalgamation theorem~\cite[Theorem~9]{S15}.
\end{proof}

\begin{externaltheorem} \label{thm41} Suppose that $2 \in A^\times$. Then for $\ell \geq 7$ the canonical map $\St(\rD_\ell, B) \to \St(\rD_\ell, R)$ is injective. \end{externaltheorem}
\begin{proof}
 Consider the following commutative diagram with exact rows, in which the lifting $t$ is obtained from~\cref{lem:lemma32}:
\begin{equation*} \xymatrix{
 C(\rD_\ell, B, I) \ar[r]^{\lambda_B} \ar@{->>}[d]_{f} & \St(\rD_\ell, B, I) \ar[r]^{\mu_B} \ar[d]_{g} &
 \St(\rD_\ell, B) \ar[r]^{\nu_B} \ar[d]_{h} & \St(\rD_\ell, k[X]) \ar[d]_{i} \\
 C(\rD_\ell, R, I) \ar[r]^{\lambda_R}         & \St(\rD_\ell, R, I) \ar[r]^{\mu_R} \ar@{-->}[ur]^{t}&
 \St(\rD_\ell, R) \ar[r]^-{\nu_R}        & \St(\rD_\ell, k[X, X^{-1}]).
}\end{equation*}
Let $a$ be an element of $\Ker(h)$. Since $i$ is injective by~\cref{field-injectivity}, the element $a$
 also lies in $\Ker(\nu_B)$ and hence comes from some $b \in \St(\rD_\ell, B, I)$ via $\mu_B$.
Since $g(b) \in \Ker(\mu_R)$ there exists some $c \in C(\rD_\ell, R, I)$ such that $\lambda_R(c) = g(b)$. 
By~\cref{lem:prop41}, $f$ is surjective, therefore $c = f(d)$ for some $d \in C(\rD_\ell, R, I)$.
The required assertion now follows from the following computation:
 \[ 1 = \mu_B\lambda_B(d) = tg\lambda_B(d) = t\lambda_Rf(d) =t(g(b)) = \mu_B(b) = a. \qedhere \]
\end{proof}

\section{Elementary calculations in relative Steinberg groups}
Throughout this section $\Phi$ denotes an irreducible root system of rank $\geq 2$, $R$ a commutative ring, and $I, J$ denote a pair of ideals of $R$.
Unless stated otherwise we assume $\Phi$ to be simply laced.
We denote by $\overline{\St}(\Phi, R, I)$ the kernel of the map $\St(\Phi, R) \to \St(\Phi, R/I)$.
This group coincides with the image in $\St(\Phi, R)$ of the relative group $\St(\Phi, R, I)$ defined in~\cref{sec:quillen}.

\subsection{Generators of relative Steinberg groups}
Denote by $\St(\Phi, I)$ the subgroup of $\St(\Phi, R)$ generated as a group by root unipotents of level $I$.
It is clear that $\overline{\St}(\Phi, R, I)$ contains $\St(\Phi, I)$ and, in fact, is its normal closure.
We also denote by $\myol{H}(\Phi, R, I)$ the subgroup of $\overline{\St}(\Phi, R, I)$ generated by the semisimple root elements $h_\alpha(u)$ and symbols $\{u, v\}$, $u \in (1+I)^\times$, $v \in R^\times$, $\alpha\in \Phi$.

We define the following two families of elements of $\overline{\St}(\Phi, R, I)$:
\begin{itemize}
 \item $z_\alpha(s, \xi) := x_\alpha(s)^{x_{-\alpha}(\xi)}$ defined for $\xi \in R$, $s \in I$;
 \item $c_\alpha(s, t) = [x_\alpha(s), x_{-\alpha}(t)]$ defined for $s \in I$, $t \in J$.
\end{itemize}

\begin{lemma}\label{Zrels} The elements $z_\alpha(s, \xi)$ satisfy the following relations for all $\xi, \eta\in R$, $s\in I$:
\begin{enumerate} 
\item\label{Z1} $z_{\alpha}(s, \xi) ^ {x_{-\alpha}(\eta)} = z_{\alpha}(s, \xi + \eta)$;
\item\label{Z2} $z_{\beta}(s, \xi) ^ {x_{\alpha}(\eta)} = x_{\alpha} (- s\xi \eta) \cdot x_{\alpha+\beta} (N_{\beta, \alpha}\cdot s\eta)     \cdot z_{\beta}(s, \xi)\ \text{if}\ \alpha + \beta \in \Phi$;
\item\label{Z3} $z_{\beta}(s, \xi) ^ {x_{\alpha}(\eta)} = x_{\alpha} (s\xi \eta) \cdot x_{\alpha-\beta} (N_{\beta,-\alpha}\cdot s\xi^2\eta) \cdot z_{\beta}(s, \xi)\ \text{if}\ \alpha - \beta \in \Phi$;

\item\label{Z4} $z_{\beta}(s, \xi) ^ {x_{\alpha}(\eta)} = z_{\beta}(s, \xi)\ \text{if}\ \alpha\perp\beta$;
\item If $\alpha+\beta\in\Phi$ then holds:
\begin{multline} \label{Z5} z_{\alpha+\beta}(s\eta, \xi) = x_\alpha(\epsilon s)\cdot x_{-\beta}(-s\xi) \cdot x_{\beta}(s\xi\eta^2) \cdot x_{\alpha+\beta}(s \eta) \cdot \\ \cdot z_\alpha(-\epsilon s, -\epsilon \xi\eta) \cdot
  x_{-\alpha}(-\epsilon s\xi^2\eta^2) \cdot x_{-\alpha-\beta}(- s \xi^2 \eta) \cdot z_{-\beta}(s\xi, -\eta)\text{ where $\epsilon = N_{\alpha,\beta}$.}\end{multline}
\end{enumerate} \end{lemma}
\begin{proof}
The first four assertions are contained in~\cite[Lemma~9]{S15}, so it remains to verify the last assertion.
Direct computation shows that
\begin{multline} \nonumber
  z_{\alpha+\beta}(s\eta, \xi) = [x_\alpha(\epsilon s)^{x_{-\alpha-\beta}(\xi)}, x_\beta(\eta)^{x_{-\alpha-\beta}(\xi)}] =
  [x_\alpha(\epsilon s) x_{-\beta}(-s\xi), x_{\beta}(\eta) x_{-\alpha}(\epsilon \xi\eta)] = \\ 
  = x_\alpha(\epsilon s) \cdot x_{-\beta}(-s\xi) \cdot z_\alpha(-\epsilon s, -\epsilon \xi\eta)^{x_{\beta}(-\eta)} \cdot z_{-\beta}(s\xi, -\eta)^{x_{-\alpha}(-\epsilon \xi\eta)},
\end{multline} 
and the required assertion follows from~\eqref{Z2}.
\end{proof}

For a subset of roots $U \subseteq \Phi$ we denote by $\mathcal{Z}(U, R, I)$ the subset of roots consisting of elements $x_\alpha(s)$, $s \in I$, $\alpha \in \Phi$ and $z_\alpha(s, \xi)$, $\alpha \in U$, $s\in I$, $\xi \in R$.

Let us mention an immediate application of the just proved lemma.
First of all, recall the following two results which give two different generating sets for the group $\overline{\St}(\Phi, R, I)$ (notice that both results are applicable for arbitrary $\Phi$)
\begin{externaltheorem}[Stein--Tits--Vaserstein] \label{thm:Tits} The group $\overline{\St}(\Phi, R, I)$ is generated (as an abstract group) by elements $z_\alpha(s, \xi)$, $\alpha \in \Phi$, $s \in I$, $\xi \in R$. \end{externaltheorem} \begin{proof} See e.\,g.~\cite[Theorem 2]{Va86}. \end{proof}

Recall that a closed root subset $S \subseteq \Phi$ is called {\it parabolic} (resp. {\it reductive}, resp. {\it special}) if $S \cup -S = \Phi$ (resp. $S = -S$, resp. $S \cap (-S) = \varnothing$).
The {\it special part} $\Sigma_S$ of a parabolic subset $S$, by definition, consists of all $\alpha \in S$ such that $-\alpha \not\in S$.
\begin{externaltheorem}[Stepanov] \label{thm:Stepanov} 
Let $S \subseteq \Phi$ be a parabolic subset of $\Phi$. Then the group $\overline{\St}(\Phi, R, I)$ is generated by the set $\mathcal{Z}(\Sigma_S, R, I)$.
 \end{externaltheorem} \begin{proof} See~\cite[Lemma~4]{S15}. \end{proof}

\begin{rem} We claim that in the simply-laced case the stronger \cref{thm:Stepanov} can be deduced from~\cref{thm:Tits} by means of~\cref{Zrels}. Indeed, consider the operator $d \colon 2^\Phi \to 2^\Phi$ of root subsets given by $d(U) = U \cup (U - U)\cap \Phi$, $U \subseteq \Phi$. In other words, $d$ adjoins to $U$ all differences of roots from $U$ which are themselves roots. It is not hard to show that for any parabolic subset $S \subseteq \Phi$ the subset $\Sigma_S$ has the property that $d^n(\Sigma_S) = \Phi$ for some $n>1$ (in fact, $n=2$). It remains to see that relation~\eqref{Z5} immediately implies that every group $G$ containing $\mathcal{Z}(U, R, I)$ also contains $\mathcal{Z}(dU, R, I)$. \end{rem}

\begin{lemma} \label{Crels}
The elements $c_\alpha(s, t)$ satisfy the following relations for all $s\in I,\ t\in J,\ \xi\in R$.
 \begin{enumerate}
 \item \label{C1} $[c_\beta(s, t),\ x_{\alpha}(\xi)] = x_{\alpha}(- st\xi) \cdot x_{\alpha+\beta}(N_{\alpha,\beta}\cdot s^2t\xi)$ if $\alpha+\beta \in \Phi$;
 \item \label{C2} $[c_\beta(s, t),\ x_{\alpha}(\xi)] = x_{\alpha}(st\xi + s^2t^2\xi) \cdot x_{\alpha-\beta}(N_{-\alpha, \beta}\cdot st^2\xi)$ if $\alpha-\beta \in \Phi$;  
 \item \label{C3} $[c_\beta(s, t),\ x_{\alpha}(\xi)] = 1$ if $\alpha \perp \beta$;  
 \item \label{C4} If $\alpha+\beta\in\Phi$ then holds:
  \[c_{\alpha+\beta}(s, t\xi) = [x_{\beta}(st),\ x_{-\beta}(\xi)] ^ {x_{\alpha+\beta}(-s) x_{-\alpha}(\epsilon t)} \cdot c_{\alpha}(\epsilon s\xi, -\epsilon t)^{-1} \cdot x_{-\beta}(-st\xi^2),\]
  where $\epsilon = N_{\alpha,\beta}$.
 \end{enumerate}
\end{lemma}
\begin{proof}
The first assertion follows from the following computation:
\begin{multline} \nonumber [c_{\beta}(s, t),\ x_{\alpha}(\xi)]  = [x_{\beta}(s),\ x_{-\beta}(t)] \cdot [x_{-\beta}(t),\ x_{\beta}(s)\cdot x_{\alpha + \beta}(N_{\alpha, \beta} \cdot s\xi)] = \\
 = {}^{x_\beta(s)}\![x_{-\beta}(t),\ x_{\alpha+\beta}(N_{\alpha, \beta} \cdot s\xi)] = x_{\alpha}(- st\xi) \cdot x_{\alpha+\beta}(N_{\alpha,\beta} \cdot s^2t\xi). \end{multline}

For the proof of the other assertions we will need the following commutator identities:
\begin{align}
 [x, yz]^y =& [y^{-1}, x] \cdot [x, z], \label{rel43} \\
  [[x, y], z] =& \left([x^{-1}, [ y^{-1}, z]] ^ {y^{-1}} \cdot [y, [ z^{-1}, x^{-1}]] ^ {z^{-1}} \right)^{x^{-1}}. \label{HW-variant}
\end{align}
 
The second assertion follows directly from~\eqref{HW-variant}:
\begin{multline} \nonumber
[[x_\beta(s),\ x_{-\beta}(t)],\ x_{\alpha}(\xi)] = \\
= \left([x_\beta(-s),\ [x_{-\beta}(-t),\ x_{\alpha}(\xi)]] ^ {x_{-\beta}(-t)} \cdot [x_{-\beta}(t),\ [ x_{\alpha}(-\xi),\ x_\beta(-s)]] ^ {x_{\alpha}(-\xi)}\right)^{x_\beta(-s)} = \\
= x_{\alpha}(-N_{\beta, \alpha-\beta} N_{\alpha,-\beta} \cdot s t \xi) ^ {x_{-\beta}(-t) \cdot x_\beta(-s)} = x_{\alpha}(s t \xi) ^ {x_{-\beta}(-t) \cdot x_\beta(-s)} = \\
= x_{\alpha}(st\xi) \cdot x_{\alpha-\beta}(-N_{\alpha,-\beta}\cdot st^2\xi)^{x_\beta(-s)} = x_{\alpha}(st\xi + s^2t^2 \xi) \cdot x_{\alpha-\beta}(N_{-\alpha,\beta}\cdot st^2\xi). \end{multline}

Finally, the last assertion can be verified via the following direct computation, which uses~\eqref{rel43} and \eqref{HW-variant} (with both sides of the equality inverted):
\begin{multline} \nonumber [x_{\alpha+\beta}(s),\ x_{-\alpha-\beta}(t\xi)] = [x_{\alpha+\beta}(s),\ [x_{-\alpha}(-\epsilon t),\ x_{-\beta}(\xi)]] = \\ 
= \left([[x_{\alpha+\beta}(-s),\ x_{-\alpha}(\epsilon t)],\ x_{-\beta}(\xi)] ^ {x_{\alpha+\beta}(-s)} \cdot  [[ x_{-\beta}(-\xi),\ x_{\alpha+\beta}(s)],\ x_{-\alpha}(\epsilon t)]^{x_{-\beta}(-\xi)}\right)^{x_{-\alpha}(\epsilon t)} = \\
= [x_{\beta}(st),\ x_{-\beta}(\xi)] ^ {x_{\alpha+\beta}(-s) x_{-\alpha}(\epsilon t)} \cdot [x_{\alpha}(\epsilon s\xi),\ x_{-\alpha}(\epsilon t)]^{x_{-\beta}(-\xi)x_{-\alpha}(\epsilon t)}  = \\
= [x_{\beta}(st),\ x_{-\beta}(\xi)] ^ {x_{\alpha+\beta}(-s) x_{-\alpha}(\epsilon t)} \cdot [x_{\alpha}(\epsilon s\xi),\ x_{-\alpha}(\epsilon t) x_{-\alpha-\beta}(t\xi)]^{x_{-\alpha}(\epsilon t)} = \\
= [x_{\beta}(st),\ x_{-\beta}(\xi)] ^ {x_{\alpha+\beta}(-s) x_{-\alpha}(\epsilon t)} \cdot [x_{-\alpha}(-\epsilon t),\ x_{\alpha}(\epsilon s\xi)] \cdot [x_{\alpha}(\epsilon s\xi),\ x_{-\alpha-\beta}(t\xi)] = \\
= [x_{\beta}(st),\ x_{-\beta}(\xi)] ^ {x_{\alpha+\beta}(-s) x_{-\alpha}(\epsilon t)} \cdot c_{\alpha}(\epsilon s\xi, -\epsilon t)^{-1} \cdot x_{-\beta}(-st\xi^2). \qedhere
\end{multline}
\end{proof}

\subsection{Computation of the kernel of the map of evaluation at $0$.}
Let $A$ be a local ring with maximal ideal $M$.
The aim of this subsection is to describe a generating set for the kernel of the map $ev_{X=0}^*\colon\overline{\St}(\Phi, A[X], M[X]) \to \overline{\St}(\Phi, A, M)$
induced by the ring homomorphism of evaluation at $0$. We denote this kernel by $K(A[X], M[X])$.

It is obvious that $K(A[X], M[X])$ contains the subgroup $\overline{\St}(\Phi, A[X], XM[X])$.
It turns out that, although $K(A[X], M[X])$ is generally strictly larger than $\overline{\St}(\Phi, A[X], XM[X])$,
 it contains very few extra generators, which can all be explicitly described (see~\cref{Kgen} and the corollary that follows it). 

It follows from~\cref{Kdecomp1} below that $K(A[X], M[X])$ coincides with the double commutator subgroup $[\overline{\St}(\Phi, A[X], M[X]), \overline{\St}(\Phi, A[X], XA[X])].$ Thus, if we replace relative Steinberg groups in the statement of~\cref{Kgen} with relative elementary groups, the resulting assertion turns into a special case of a much more general recent result of N.~Vavilov and Z.~Zhang (cf.~\cite[Theorem~1]{VZ18}).

Since $ev_{X=0}^*$ admits a section, we can consider $\overline{\St}(\Phi, A, M)$ and $\St(\Phi, M)$ as subgroups of $\overline{\St}(\Phi, A[X], M[X])$,
 moreover, one has $\overline{\St}(\Phi, A[X], M[X]) = \overline{\St}(\Phi, A, M) \cdot K(A[X], M[X]).$
\begin{lemma} \label{Kdecomp1} The following decomposition holds
 \[ K(A[X], M[X]) = \overline{\St}(\Phi, A[X], XM[X]) \cdot \left[\St(\Phi, XA[X]),\ \overline{\St}(\Phi, A, M)\right].\] \end{lemma}
\begin{proof} Fix $g \in K(A[X], M[X])$ and write it as $g(X) = \prod_i z_{\alpha_i}(f_i(X), \xi_i(X))$ for some $f_i(X) = f_i(0) + Xf_i'(X) \in M[X]$, $\xi_i(X) = \xi_i(0) + X\xi_i'(X) \in A[X]$.
 It is clear that modulo $\overline{\St}(\Phi, A[X], XM[X])$ the element $g(X)$ is congruent to $g_1(X) = \prod_i z_{\alpha_i}(f_i(0), \xi_i(X)).$ 
 
 Now each factor $z_{\alpha_i}(f_i(0), \xi_i(X))$ can be written as follows:
 \[z_{\alpha_i}(f_i(0), \xi_i(0))^{x_{-\alpha_i}(X\xi'_i(X))} = [x_{-\alpha_i}(-X\xi'_i(X)),\ z_{\alpha_i}(f_i(0), \xi_i(0))] \cdot z_{\alpha_i}(f_i(0), \xi_i(0)).\]
 It follows from the formula $[g,\ h]^{h_1} = [h_1^{-1},\ g][g,\ h_1^{-1}h]$ that the subgroup \[C_0 := \left[\St(\Phi, XA[X]),\ \overline{\St}(\Phi, A, M)\right]\] is normalized by $\overline{\St}(\Phi, A, M)$. Thus, we conclude that $g_1(X)$ is congruent to $\prod_i z_{\alpha_i}(f_i(0), \xi_i(0)) = g(0) = 1$ modulo $C_0$,
 which implies the assertion. \qedhere \end{proof}

For a closed root subset $S \subseteq \Phi$ we denote by $U(S, M)$ the subgroup of $\St(\Phi, A)$ generated by root subgroups $X_\alpha(M)$ corresponding to all $\alpha \in S$. We denote by $\Phi^+$ (resp. $\Phi^-$) the subsets of positive (resp. negative) roots of $\Phi$ with respect to some chosen order on $\Phi$.
 
\begin{externaltheorem}[Stein] \label{thm:Stein} One has $\overline{\St}(\Phi, A, M) = U(\Phi^+, M) \cdot \myol{H}(\Phi, A, M) \cdot U(\Phi^-, M).$ \end{externaltheorem} \begin{proof} See~\cite[Theorem~2.4]{Ste73}. \end{proof}

\begin{prop} \label{Kgen} The subgroup $K(A[X], M[X])$ is generated as an abstract group by the subgroup $\overline{\St}(\Phi, A[X], XM[X])$ and
 the elements $[x_\alpha(m), x_{-\alpha}(X\xi)]$, $m \in M$, $\xi \in A[X]$, $\alpha \in \Phi$. \end{prop}
\begin{proof} From~\eqref{eq:conj-h-x} we obtain that $\myol{H}(\Phi, A, M)$ normalizes both $\St(\Phi, XA[X])$ and $\St(\Phi, M)$ and, moreover, that $[\myol{H}(\Phi, A, M),\ \St(\Phi, XA[X])] \subseteq \overline{\St}(\Phi, A[X], XM[X])$. 

Denote by $C_1$ the commutator subgroup $[\St(\Phi, XA[X]),\ \St(\Phi, M)]$.
It is clear that for $g \in \St(\Phi, XA[X])$, $h \in \myol{H}(\Phi, A, M)$, $u^+ \in U(\Phi^+, M)$, $u^- \in U(\Phi^-, M)$ holds:
\[ [g,\ h u^+ u^-] = [g,\ h] \cdot [{}^{h}\!g,\ {}^{h}\!(u^+u^-)] \in \St(\Phi, A[X], XM[X]) \cdot C_1.\]
Since $C_0$ is generated by the above commutators and $\overline{\St}(\Phi, A[X], XM[X])$ is a normal subgroup of $\St(\Phi, A[X])$
we obtain that $C_0 \subseteq \overline{\St}(\Phi, A[X], XM[X]) \cdot C_1$ and consequently that
$K(A[X], M[X]) = \overline{\St}(\Phi, A[X], XM[X]) \cdot C_1.$
 
It is clear that modulo $\overline{\St}(\Phi, A[X], XM[X])$ the commutator subgroup $C_1$ is generated by elements of the form $[x_\alpha(m),\ x_{-\alpha}(X\xi)]^g$, where $m \in M$, $\xi \in A[X]$, $g \in \St(\Phi, A[X])$.
Thus, it remains to show that commutators $[[x_\alpha(m),\ x_{-\alpha}(X\xi)],\ g]$ belong to $\overline{\St}(\Phi, A[X], XM[X])$.
Since the latter subgroup is normal it suffices to prove this inclusion in the special case when $g$ is a member of some generating set for $\St(\Phi, A[X])$.
Clearly, the set consisting of $x_\beta(\xi)$, $\xi \in A[X]$, $\beta \neq \pm \alpha$ is such a generating set, 
 and in this case the required inclusions follow from~\eqref{C1}--\eqref{C3} of~\cref{Crels}. \end{proof}

\begin{corollary} \label{Kgen-strong} For a local pair $(A, M)$ and arbitrary fixed root $\gamma$ of an irreducible simply-laced root system $\Phi$ the subgroup $K(A[X], M[X])$ is generated as a group by $\overline{\St}(\Phi, A[X], XM[X])$ and the elements $c_{\gamma}(m, X\eta)$, where $m \in M$, $\eta \in A[X]$. \end{corollary}
\begin{proof} Substituting $\xi = 1$, $s = m$, $t = X\eta$ into relation~\eqref{C4} of~\cref{Crels} we obtain that modulo 
 $\overline{\St}(\Phi, A[X], XM[X])$ the element $c_{\alpha + \beta}(m, X\eta)$ is equivalent to $c_{\alpha}(-\epsilon m, -\epsilon X \eta)^{-1}$, $\epsilon = N_{\alpha, \beta}$. The assertion of the corollary now easily follows from the irreducibility of $\Phi$. \end{proof}   

\section{Proof of the main result}
The main result of this section is~\cref{thm:P1glueing}, which is a direct generalization of~\cite[Proposition~4.3]{Tu83}. 
The object playing a key role in its proof is a certain action of the group $G = \St(\Phi, A[X\inv] + M[X])$ on a certain set $\overline{V}$, which is defined in~\cref{sec:V-construction}. Later, we will see that $\overline{V}$ is, in fact, a set-theoretic $G$-torsor.
To be able to write an explicit formula for this action we need two major ingredients. The first one is ~\cref{lemma33}, which gives a presentation of $G$ with much fewer generations and relations than in the original presentation~\eqref{rel:add}-\eqref{rel:CCF}. The other ingredients are certain subgroups $P_\alpha(0)$, $P_\alpha(*)$ of $\St(\Phi, A[X, X\inv])$ modeled after the nameless groups from~\cite[Lemma~3.4]{Tu83}. The definition and properties of these groups are given in Sections~\ref{sec:Pa0-basic}--\ref{sec:Pa0-Steinberg}.

\subsection{Presentation of Steinberg groups by homogeneous generators}
\label{sec:presentation}
Let $M\trianglelefteq A$ be an ideal of a commutative ring $A$.
We consider $A[t, t\inv]$ as a $\mathbb{Z}$-graded ring in which $t$ has degree $1$.
We denote by $B = B(A, M)$ the subring $A[t] + M[t\inv] \subseteq A[t, t\inv]$ with the induced grading.
As an $A$-module $B$ decomposes as $\oplus_{d\in\mathbb Z}B_d$ where $B_d=M \cdot t^d$ for $d<0$, and $B_d=A \cdot t^d$ for $d\geq0$. Obviously, $B = A[t]$ in the case $M=0$ and $B = A[t, t\inv]$ in the case $M=A$.

Whenever the coefficient $\xi$ of a Steinberg generator $g = x_\alpha(\xi)$ of $\St(\Phi, B)$ is a homogeneous element of $B$, i.\,e. $\xi \in B_d$ for some $d \in \mathbb{Z}$,
 we call the corresponding generator $g$ {\it homogeneous of degree $d$}.
It is not hard to show that $\St(\Phi, B)$ can be presented by the set of all homogeneous Steinberg generators modulo the following set of Steinberg relations
(below $a, a' \in B_d$, $b\in R_e$ and $d,e \in \mathbb{Z}$): 
\begin{align}
&\,\,x_{\alpha}(a)\cdot x_{\alpha}(a') =  x_{\alpha}(a+a'),                        & \tag{R$1_d$} \\
&\,[x_{\alpha}(a),\,x_{\beta}(b)]= x_{\alpha+\beta}(N_{\alpha, \beta} \cdot ab),   & \alpha + \beta \in \Phi,\ \tag{R$2_{d,e}$} \\
&\,[x_{\alpha}(a),\,x_{\beta}(b)]= 1,                                              & \alpha - \beta \in \Phi,\ \tag{R$3^\angle_{d,e}$} \\
&\,[x_{\alpha}(a),\,x_{\beta}(b)]= 1,                                              & \alpha \perp \beta.\ \tag{R$3^\bot_{d,e}$}
\end{align}
By the degree of a Steinberg relation we mean the maximum of degrees of generators that appear in the relation.
For example, the degree of every relation of type $\text{R2}_{d,e}$ is $\max(d,e,d+e)$ 
 while the degree of a relation of type $\text{R3}^\bot_{d,e}$ or $\text{R3}^\angle_{d,e}$ is $\max(d,e)$.

For $n\geq 1$ we define ``truncated'' Steinberg group $\St^{\leq n}(\Phi, B)$ by means of the set $\mathcal{X}_{\leq n}^\Phi$ of homogeneous Steinberg generators of degree $\leq n$ and the set of Steinberg relations $\mathcal{R}_{\leq n}^\Phi$ of degree $\leq n$. We denote by $F(\mathcal{X}^\Phi_{\leq n})$ the free group on $\mathcal{X}^\Phi_{\leq n}$.

The following lemma asserts that most of the relations of type $\text{R3}^\bot_{d,e}$ of positive degree in this presentation of $\St^{\leq n}(\Phi, B)$ 
 are superfluous and can be omitted.
\begin{lemma}\label{superfluous-relations}
 For every simply-laced root system $\Phi$ of rank $\geq 3$ and every $n \geq 1$ one can exclude from
 the presentation of $\St^{\leq n}(\Phi, B)$ all relations of type $\text{R3}_{d,e}^\bot$ whenever $\max(0,d) + \max(0,e) > 1$.
\end{lemma}
\begin{proof}
For the proof we will need the following commutator identities:
\begin{align} \label{eq:H1ii} [xy, z] = {}^x[y, z] \cdot [x,z];&\\
 \label{eq:H1iii} [x,z] = 1 \text{ implies } [x, [y,z]] = [[x,y],{}^yz].& \end{align}
 
The proof is based on the following observation: every relation of type $\text{R3}^\bot_{d,e}$ of degree
 $\geq 2$ is a consequence of some relation of type $\text{R3}^\bot$ of strictly smaller degree.
Let us fix some relation $[x_\alpha(at^d),\ x_\gamma(bt^e)] = 1$ of type $\text{R3}^\bot_{d,e}$ for some $\alpha\perp\gamma$. Denote this relation by $R$.

We can find $\beta \in \Phi$ forming an obtuse angle with both $\alpha$ and $\gamma$ (see e.\,g.~\cite[Lemma~3.1.2]{RS76}).
Without loss of generality we may assume $e \geq d$ and $e > 0$. We need to consider two cases.
\begin{enumerate}
\item In the case $0 < d \leq e \leq n$ the relation $R$ is a consequence of some relation of type $\text{R3}^\bot_{0,e-d}$:
\begin{align} x_{\alpha+\beta+\gamma}(-\epsilon_1 \epsilon_2 \cdot abt^e) = 
[x_{\beta+\gamma}(\epsilon_1 \cdot bt^{e-d}), [x_{-\beta}(t^d), x_{\alpha+\beta}(-\delta_1 \cdot a)] ] & \text{ by $\text{R2}_{d,0}$, $\text{R2}_{d, e-d}$} \label{first-computation} \\ 
 = [[x_{\beta+\gamma}(\epsilon_1 \cdot bt^{e-d}), x_{-\beta}(t^d)], {}^{x_{-\beta}(t^d)} x_{\alpha+\beta}(- \delta_1 \cdot a)] & \text{ by~\eqref{eq:H1iii}, $\text{R3}^\bot_{0,e-d}$} \nonumber \\ 
 = {}^{x_{-\beta}(t^d)} [x_{\gamma}(- bt^e), x_{\alpha+\beta}(-\delta_1 \cdot a)] & \text{ by $\text{R2}_{e-d,d}$, $\text{R3}^\angle_{d,e}$} \nonumber \\ 
 = {}^{x_{-\beta}(t^d)} [x_{\beta + \gamma}(\epsilon_1 \cdot bt^{e-d}), x_{\alpha}(at^d)] & \text{ by $\text{R2}_{e,0}$, $\text{R2}_{e-d,d}$} \nonumber \\ 
 = [x_{\gamma}(bt^e) \cdot x_{\beta + \gamma}(\epsilon_1 \cdot bt^{e-d}), x_{\alpha}(at^d)] & \text{ by $\text{R2}_{e-d,d}$, $\text{R3}^\angle_{d,d}$} \nonumber \\ 
 = {}^{x_\gamma(bt^e)}x_{\alpha+\beta+\gamma}(-\epsilon_1\epsilon_2 \cdot abt^e) \cdot [x_\gamma(bt^e), x_\alpha(at^d)] & \text{ by~\eqref{eq:H1ii}, $\text{R2}_{e-d,d}$} \nonumber \\
 = x_{\alpha+\beta+\gamma}(-\epsilon_1\epsilon_2 \cdot abt^e) \cdot [x_\gamma(bt^e), x_\alpha(at^d)] & \text{ by $\text{R3}^\angle_{e,e}$}, \nonumber \end{align}
 where $\epsilon_1 = N_{\beta,\gamma}$, $\epsilon_2 = N_{\alpha,\beta+\gamma}$, $\delta_1 = N_{\alpha,\beta}$ and in the 4th equality we use~\eqref{eq:cocycle}.
  
\item In the case $d \leq 0 \leq e \leq n$ the relation $R$ is a consequence of some relation of type $\text{R3}^\bot_{1,d+e-1}$:
\begin{align*} [x_\alpha(at^d), x_{\gamma}(bt^{e})] = [x_\alpha(at^d), [x_{\beta+\gamma}(b t^{e-1}), x_{-\beta}(-\epsilon_1 t)]] & \text{ by~$\text{R2}_{e-1,1}$ } \\
 = [[x_\alpha(at^d), x_{\beta+\gamma}(bt^{e-1})], {}^{x_{\beta+\gamma}(bt^{e-1})}\!x_{-\beta}(-\epsilon_1 t)] & \text{ by \eqref{eq:H1iii} and $\text{R3}^\angle_{d,1}$ } \\
 = {}^{x_{\beta+\gamma}(bt^{e-1})}\![x_{\alpha+\beta+\gamma}(\epsilon_2abt^{d+e-1}), x_{-\beta}(-\epsilon_1 t)] & \text{ by $\text{R2}_{d,e-1}$ and $\text{R3}^\angle_{e-1,d+e-1}$  } \\
 = 1 & \text{ by $\text{R3}^\bot_{1,d+e-1}$,} \end{align*} where $\epsilon_1 = N_{\beta,\gamma}$, $\epsilon_2 = N_{\alpha,\beta+\gamma}$.
 \end{enumerate}                                                           

The assertion of the lemma now follows from the above observation by induction on the degree of $R$ and the fact
 that by~\eqref{first-computation} relation $\text{R3}^\bot_{1,1}$ is a consequence of $\text{R3}^\bot_{0,0}$.
\end{proof}

The following proposition is the main result of this subsection and also a direct generalization of~\cite[Lemma~3.3]{Tu83}.
\begin{prop} \label{lemma33} For $\Phi=\rA_{\geq 4}, \rD_{\geq 5}, \rE_{6,7,8}$ and $n \geq 1$ the map $i_n\colon \St^{\leq n}(\Phi, B) \to \St^{\leq n+1}(\Phi, B)$, induced by the natural embedding of generators, is an isomorphism. In particular, the obvious map $\St^{\leq 1}(\Phi, B) \to \St(\Phi, B)$ is an isomorphism. \end{prop}
\begin{proof}
 We need to construct a map $j_n$ which would be the inverse of $i_n$. 
 We start with a map $\widetilde{j}_n^\Phi \colon F\langle \mathcal{X}^\Phi_{\leq n+1} \rangle \to \St^{\leq n}(\Phi, B)$ defined via
 \[ \widetilde{j}^{\Phi}_n(x_\alpha(at^k)) = \begin{cases} x_\alpha(at^k), & k\leq n; \\
      [x_{\alpha - \beta} (N_{\alpha-\beta, \beta} \cdot at^{k-1}), x_{\beta}(t)], & k = n+1, \end{cases} \]
 where $\beta$ is any root of $\Phi$ forming a sharp angle with $\alpha$.
 Standard argument (cf. \cite[Proposition~1.1]{Re75} or~\cite[Proposition~3.2.2]{RS76}) shows that $\widetilde{j}^\Phi_n$ does not depend on the choice of $\beta$.
  
 Set $\mathcal{R}^\Phi_{n+1} = \mathcal{R}^\Phi_{\leq n+1} \setminus \mathcal{R}^\Phi_{\leq n}$. It suffices to verify that the image of every relator $R \in \mathcal{R}^\Phi_{n+1}$ under $\widetilde{j}^\Phi_n$ is a trivial element of $\St^{\leq n}(\Phi, B)$. In the special case $\Phi=\rA_{\geq 4}$ this has already been demonstrated by Tulenbaev in~\cite[Lemma~3.3]{Tu83}, so in this case the proof of the proposition is complete. We will deduce the assertion in the remaining cases $\Phi=\rD_\ell,\rE_\ell$ from the special case $\Phi=\rA_4$ of Tulenbaev's result.
 
 Let $R$ be a relation from $\mathcal{R}^\Phi_{n+1}$. By~\cref{superfluous-relations} we may assume that $R$ is not of type $\text{R3}^\bot$, therefore the roots $\alpha, \beta$ appearing in $R$ are contained in a root subsystem of $\Phi$ of type $\rA_2$. Our assumptions on $\Phi$ guarantee that there exists some root subsystem $\Psi$ of type $\rA_4$ containing $\alpha$ and $\beta$. Consider the following commutative diagram in which the vertical arrows are induced by the embedding $\Psi\subseteq\Phi$. 
  \begin{equation} \nonumber \xymatrix{
 F(\mathcal{X}_{\leq n+1}^\Psi) \ar[r]^{j_n^\Psi} \ar[d] & \St^{\leq n}(\Psi, B) \ar[d] \\
 F(\mathcal{X}_{\leq n+1}^\Phi) \ar[r]^{j_n^\Phi} & \St^{\leq n}(\Phi, B) }
 \end{equation}
The relation $R$ lies in the image of the left arrow, therefore it comes from some relation $R' \in \mathcal{R}^\Psi_{n+1}$. The image of $R'$ in $\St^{\leq n}(\Psi, B)$ under $j_n^\Psi$ is trivial by Tulenbaev's result. But this implies that the image of $R$ under $\widetilde{j}_n^\Phi$ is also trivial and hence that $\widetilde{j}_n^\Phi$ gives rise to the desired map $j_n$.
\end{proof}

\begin{rem}
 Notice that in the case $\Phi=\rD_\ell$ the pair $\{\alpha_{\ell-1}, \alpha_{\ell}\}$ of orthogonal simple roots cannot be embedded into a root subsystem of type $\rA_4$. This explains why we needed to exclude relations $\text{R3}^\bot$ from the presentation of $\St^{\leq n}(\Phi, B)$ in the proof of the above proposition.
\end{rem}
\begin{rem}
 In the special case $M=A$, $B = A[t, t\inv]$ the assertion of the above proposition also holds in the cases $\Phi=\rA_2, \rA_3, \rD_4$. This is a consequence of the presentation of D.~Allcock applied to the affine untwisted Steinberg group $\St(\Phi, A[t, t\inv]) \cong \St(\widetilde{\Phi}^{(1)}, A)$. Allcock's presentation implies that $\St(\Phi, A[t, t\inv])$ can be presented using only generators and relations of degree $\leq 1$ with respect to both $t$ and $t^{-1}$, see~\cite[Corollary~1.3]{A13}.
 
 In the cases $\Phi = \rA_3, \rD_4$, $M \neq A$ it is still possible to prove the injectivity of $i_n$ starting from $n\geq 2$ using a variation of the argument of Rehmann--Soul{\'e} (cf. the lower bound for $m$ in~\cite[3.2.1]{RS76}). However, apparently, it is not possible to establish the injectivity of $i_1$ in the specified cases using arguments similar to~\cite{RS76}.
\end{rem}

\subsection{The subgroups $P_\alpha(0)$, $P_\alpha(*)$ and their properties.} \label{sec:Pa0-basic}
For a root system $\Phi$ consider the following subsets of $\Phi$:
\begin{align} Z_+(\alpha) & = \{ \beta \in \Phi \mid \langle \alpha, \beta \rangle > 0 \}, \\
   Z_0(\alpha) & = \{ \beta \in \Phi \mid \alpha + \beta \not\in \Phi,\ \langle \alpha, \beta \rangle = 0 \}, \\
   Z(\alpha)   & = Z_0(\alpha) \sqcup Z_+(\alpha). \end{align}
Clearly, $Z_0(\alpha)$ (resp. $Z_+(\alpha)$) is a reductive (resp. special) subset of $\Phi$.
   
We denote by $Z_\alpha(A, M)$ the subgroup of $\overline{\St}(\Phi, A, M)$ generated by elements
 $x_{\beta}(m),\ \beta \in Z_+(\alpha)$ and $z_{\gamma}(m, \zeta),\ \gamma \in Z_0(\alpha),$ where $m \in M$, $\zeta \in A$.
It is not hard to see that \[Z_\alpha(A, M) = \Img\left(\overline{\St}(Z_0(\alpha), A, M) \to \overline{\St}(\Phi, A, M)\right) \rtimes U(Z_+(\alpha), M). \]
It is clear, that $Z_\alpha(A, M)$ centralizes the root subgroup $X_\alpha(A)$ (cf. \cite[984]{St71}).

For the rest of this subsection $\Phi$ is a simply-laced root system of rank $\geq 3$ and $\alpha$ is a fixed root of $\Phi$. Notice that in the simply-laced case the assumption $\alpha+\beta\not\in \Phi$ in the definition of $Z_0(\alpha)$ is superfluous, i.\,e. $Z_0(\alpha) = \{ \alpha\in\Phi \mid \alpha \perp \beta \}$
(cf.~\cite[Proposition~5.7]{St71}).

\begin{rem}\label{Z-DS} Notice that our assumptions on the rank of $\Phi$ guarantee that $Z_0(\alpha)$ is nonempty. In particular, if $A$ is a local ring with maximal ideal $M$ the group $Z_\alpha(A, M)$ contains relative Dennis--Stein symbols $\langle a, m \rangle$ for $a\in A$, $m\in M$.
By~\eqref{DS-S-relationship} relative Steinberg symbols $\{a, 1+m\}$ are also contained in $Z_\alpha(A, M)$ for all $a\in A^\times$, $m\in M$. \end{rem}

\begin{df} \label{defP0}
Let $M$ be an ideal of a local ring $A$. We denote by $P_{\alpha, M}(0)$ the subgroup of $\St(\Phi, A[X, X^{-1}])$ generated by the images of the following elements of $\overline{\St}(\Phi, A[X], M[X])$ under the map $j_+ \colon \St(\Phi, A[X]) \to \St(\Phi, A[X, X^{-1}])$:
\begin{enumerate} \item $z_{\beta}(Xf, \xi),\ \beta \in \Phi \text{ such that }\alpha + \beta \in \Phi\text{ or } \alpha - \beta \in \Phi;$
 \item $z_{\beta}(f, X\xi),\ \alpha - \beta \in \Phi;$
 \item $z_{\beta}(f, \xi),\ \beta \perp \alpha;$
 \item $x_{-\alpha}(X^2f);$
 \item $x_{\alpha}(f)$. \end{enumerate}
In the above formulae and for the rest of this subsection $f \in M[X]$ and $\xi \in A[X]$. 

We also denote by $P_{\alpha, M}(*)$ the subgroup of $\St(\Phi, A[X, X^{-1}])$ generated by $P_{\alpha, M}(0)$ and the elements $x_{-\alpha}(mX)$, $m \in M$.
\end{df}

Almost always we will be using the above definition in the situation when $M$ is precisely the maximal ideal of $A$.
The only exception to this is~\cref{P0-conj-invariant} where the above subgroups are also used for $M=A$.

\begin{lemma}\label{P0_normal} The subgroup $P_{\alpha, M}(0)$ is normal in $P_{\alpha, M}(*)$. In particular, there is a short exact sequence of groups, which is split by the map $m \mapsto x_{-\alpha}(mX)$ (we denote by $(M, +)$ the additive group of the ideal $M$):
\[\xymatrix{1 \ar[r] & P_\alpha(0) \ar[r] & P_\alpha(*) \ar@{->>}[r]^{p_\alpha} & (M, +) \ar@<5pt>@{-->}[l] \ar[r] & 1}.\] \end{lemma}
\begin{proof} We need to verify that the conjugate by $x_{-\alpha}(mX)$ to every generator $g$ of $P_\alpha(0)$ listed in~Definition~\ref{defP0} belongs to $P_\alpha(0)$.
The assertion is obvious for the generators of type 3 and 4.

Suppose $g$ has type 1 or 2 and $\alpha - \beta \in \Phi$. By~\cref{Zrels} we obtain that
\begin{align} z_{\beta}(Xf, \xi) ^ {x_{-\alpha}(mX)} = & x_{-\alpha} (- mX^2f\xi) \cdot x_{\beta-\alpha} (N_{\beta, -\alpha}\cdot mX^2f) \cdot z_{\beta}(Xf, \xi), \label{eq3-1} \\
  z_{\beta}(f, X\xi) ^ {x_{-\alpha}(mX)} = & x_{-\alpha} (- mX^2f\xi ) \cdot x_{\beta-\alpha} (N_{\beta, -\alpha}\cdot mXf) \cdot z_{\beta}(f, X\xi), \label{eq3-2} \end{align}
The expressions in the right hand sides of~\eqref{eq3-1} and~\eqref{eq3-2} are products of generators of type 4, 1, 1 and 4, 1, 2, respectively.  

Suppose $g$ has type 1 and  $\alpha + \beta \in \Phi$. By~\cref{Zrels} 
\begin{equation} \label{eq3-3} z_{\beta}(Xf, \xi) ^ {x_{-\alpha}(mX)} = x_{-\alpha} (mX^2f\xi ) \cdot x_{-\alpha-\beta} (N_{\beta,\alpha}\cdot mX^2f\xi^2) \cdot z_{\beta}(Xf, \xi). \end{equation}
and the latter expression is a product of generators of type $4, 1, 1$.

Finally, suppose $g$ has type $5$.
Substituting in~\eqref{Z5} $s = -\epsilon f$, $\xi = -\epsilon m$, $\eta=X$ and expressing $z_\alpha(f, mX)$ through other terms we obtain that
\begin{multline} \label{eq:zalpha} z_\alpha(f, mX) = x_{\alpha+\beta}(\epsilon Xf) \cdot x_{\beta}(-mX^2 f) \cdot x_{-\beta}(mf) \cdot x_\alpha(f) \cdot \\ 
 \cdot z_{\alpha+\beta}(-\epsilon X f, -\epsilon m) \cdot z_{-\beta}(-mf, -X) \cdot x_{-\alpha-\beta}(-\epsilon m^2X f) \cdot x_{-\alpha}(-m^2X^2 f), \end{multline}
and the latter expression is a product of generators of type 1, 1, 2, 5, 1, 2, 1, 4. \end{proof}

\begin{rem}\label{rem:c-DS} Notice that $P_{\alpha, M}(0)$ contains the elements $c_{\beta}(f, X\xi)$ for all $\beta \in Z(\alpha)$ (they can be factored as products of two elements of type 2 or 3). It is also easy to check that $P_{\alpha, M}(0)$ contains the image of $Z_\alpha(A, M)$ under the natural embedding of $\St(\Phi, A, M) \hookrightarrow \St(\Phi, A[X, X\inv])$.
In particular, if $M$ is the maximal ideal of $A$, the subgroup $P_{\alpha, M}(0)$ contains relative Dennis--Stein and Steinberg symbols. \end{rem}

The following lemma shows that $P_{\alpha, M}(*)$ is sufficiently large.
\begin{lemma} \label{Pstar-large} Suppose that $(A, M)$ is a local pair. Then the subgroup $P_\alpha(*)$ contains the image of $K(A[X], M[X])$ in $\St(\Phi, A[X, X^{-1}])$. \end{lemma}
\begin{proof} Clearly, $P_\alpha(*)$ contains the elements $x_\beta(Xf)$ for all $\beta \in \Phi$ and $z_\beta(Xf, \xi)$ for $\beta \in \Phi \setminus \{\pm \alpha\}$, $f \in M[X]$, $\xi\in A[X]$ hence by Stepanov theorem $P_\alpha(*)$ contains whole $j_+(\overline{\St}(\Phi, A[X], XM[X]))$. The required assertion now follows from~\cref{Kgen-strong} and the preceding remark. \end{proof}

\begin{rem} \label{Pstar-char} The above lemma allows us to characterize $P_{\alpha, M}(*)$ as follows: it consists precisely of images under $j_+$ of the elements 
 $g \in \overline{\St}(\Phi, A[X], M[X])$ for which $ev_{X=0}^*(g)$ lies in the subgroup $Z_\alpha(A, M)$. \end{rem} 

\begin{rem} \label{rem:palpha} It follows from~\cref{P0_normal} and~\cref{Pstar-large} that the value of the function $p_\alpha$ from the statement of~\cref{P0_normal} on an element $g\in K(A[X], M[X])$ can be computed via the following procedure. 
Start with any presentation of $g$ as a product of elements $z_{\beta}(Xf, \xi)$ for $\beta \in \Phi$ and $c_\delta(f, X\xi)$ for some fixed $\delta\in Z_0(\alpha)$. 
Now pick among these factors those that correspond to the root $\beta = -\alpha$ (i.\,e. pick all factors $z_{-\alpha}(Xf_i, \xi_i)$). 
Now $p_\alpha(g)$ is precisely the sum of constant terms of the polynomials $f_i$. \end{rem}

\begin{lemma} \label{P0-conj-invariant} Suppose $(A, M)$ is a local pair. Then for any $\beta \in Z(\alpha)$ and $b \in A$ the subgroups $P_{\alpha, M}(0)$ and $P_{\alpha, M}(*)$ are stable under conjugation by $x_\beta(b)$. \end{lemma}
\begin{proof}
Notice that both $Z_\alpha(A, M)$ and $K(A[X], M[X])$ are stable under the specified conjugation, which implies the assertion for $P_{\alpha, M}(*)$. 
To obtain the assertion for $P_{\alpha, M}(0)$ consider the following commutative diagram. \[ \xymatrix{ P_{\alpha, M}(0) \ar@{^{(}->}[r] \ar@{^{(}->}[d] & P_{\alpha, M}(*) \ar@{->>}[r]^{p_{\alpha, M}} \ar@{^{(}->}[d] & (M, +) \ar@{^{(}->}[d] \\ P_{\alpha, A}(0) \ar@{^{(}->}[r] & P_{\alpha, A}(*) \ar@{->>}[r]^{p_{\alpha, A}} & (A, +)} \]
Notice that $x_\beta(b) \in P_{\alpha, A}(0)$ therefore for $g \in P_{\alpha, M}(0)$ one has \[p_{\alpha, M}(x_\beta(b) \cdot g \cdot x_\beta(-b)) = p_{\alpha, A}(x_\beta(b) \cdot g \cdot x_\beta(-b)) = p_{\alpha, A}(g) = 0,\] which implies the assertion.\end{proof}

For the rest of this section $M$ denotes the maximal ideal of $A$, so everywhere below we shorten the notation for $P_{\alpha, M}(*)$ (resp. $P_{\alpha, M}(0)$) to just $P_{\alpha}(*)$ (resp. $P_{\alpha}(0)$). For $m \in M$ denote by $P_\alpha(m)$ the coset $P_\alpha(0) \cdot x_{-\alpha}(mX)$.
From~\cref{P0_normal} it follows that $P_\alpha(*)$ coincides with the union of all $P_\alpha(m)$, $m\in M$. This allows us to make the following definition. 

\begin{df} Define the map $S_\alpha(a, -) \colon P_\alpha(*) \to \St(\Phi, A[X, X^{-1}])$ on each coset $P_\alpha(m)$, $m \in M$ via the following formula:
\[ S_\alpha(a, g) = x_\alpha(aX^{-1})\cdot g \cdot x_\alpha\left(-\tfrac{aX^{-1}}{1 + am}\right) \cdot \{X, 1+ am\}.\] \end{df}
It follows immediately from the definition of $S_\alpha(a, -)$ that for $g_1 \in P_\alpha(m)$, $g_2 \in P_\alpha(*)$ holds
\begin{equation} \label{eq:Smult} S_\alpha(a, g_1\cdot g_2) = S_\alpha(a, g_1) \cdot S_\alpha\left(\tfrac{a}{1+am}, g_2\right).\end{equation}

It is clear that the restriction of the map $S_\alpha(a, g)$ to the subgroup $P_\alpha(0)$ coincides with the map of left conjugation by $x_\alpha(aX^{-1})$.

\begin{lemma}\label{P0_conj} The subgroup $P_\alpha(0)$ is stable under conjugation by $x_\alpha(aX^{-1})$ for arbitrary $a\in A$. \end{lemma}
\begin{proof} We need to verify that the conjugate by $x_\alpha(aX^{-1})$ to every generator $g = z_\beta(f, \xi)$ from~\cref{defP0} lies in $P_\alpha(0)$. 
The assertion is clear for generators of type 3 and 5.
First of all, consider the case $\alpha + \beta \in \Phi$, which is only possible for the generator of type 1:
\begin{equation}\label{eq:Zconj1} z_{\beta}(Xf, \xi) ^ {x_{\alpha}(aX^{-1})} = x_{\alpha} (- afg) \cdot x_{\alpha+\beta} (N_{\beta, \alpha}\cdot af) \cdot z_{\beta}(Xf, \xi). \end{equation}
Clearly, the expression in the right hand side is a product of generators of type 5, 2, 1. 

Now consider the case $\alpha - \beta \in \Phi$, which is possible for the generators of type 1 and 2.
Computing the conjugates of these generators using~\cref{Zrels} we obtain that
\begin{align} z_{\beta}(Xf, \xi) ^ {x_{\alpha}(aX^{-1})} = &  x_{\alpha} (af\xi) \cdot x_{\alpha-\beta} (N_{\beta,-\alpha}\cdot af\xi^2) \cdot z_{\beta}(Xf, \xi), \label{eq:Zconj2} \\
z_{\beta}(f, X\xi) ^ {x_{\alpha}(aX^{-1})} = & x_{\alpha} (af\xi) \cdot x_{\alpha-\beta} (N_{\beta,-\alpha}\cdot aXf\xi^2) \cdot z_{\beta}(f, X\xi). \nonumber \end{align}
The expressions in the right-hand side are products of generators of type 5, 2, 1 and 5, 1, 2.

Let us verify the assertion for the generator of type 4.
Choose $\beta\in \Phi$ such that $\alpha+\beta \in \Phi$. 
Substituting $s = Xf$, $\eta = X$, $\xi = aX^{-1}$ and $\alpha = -\alpha - \beta$ into~\eqref{Z5} we obtain that
\begin{multline} \label{eq:Zconj3} z_{-\alpha}(X^2f, aX^{-1}) = x_{-\alpha-\beta}(\epsilon Xf) \cdot x_{-\beta}(-af) \cdot x_{\beta}(aX^2 f) \cdot x_{-\alpha}(X^2f) \cdot \\
 \cdot z_{-\alpha-\beta}(-\epsilon Xf, -\epsilon a) \cdot x_{\alpha+\beta}(-\epsilon a^2 Xf) \cdot x_{\alpha}(- a^2 f) \cdot z_{-\beta}(a f, -X), \end{multline}
where $\epsilon = N_{-\alpha-\beta,\beta}$. 
Clearly, the expression in the right-hand side is a product of generators of type 1, 2, 1, 4, 1, 1, 5, 2. \end{proof}

\begin{lemma} \label{lem:Tulenbaev-formula} 
One has $S_\alpha(a, x_{-\alpha}(mX)) = x_{-\alpha}\left(\tfrac{mX}{1+am}\right) \cdot \langle a, m\rangle_\alpha \cdot h_\alpha(1+am).$
\end{lemma}
\begin{proof} Since $\Phi$ is nonsymplectic, we can choose $\gamma \in \Phi$ such that $\langle \alpha, \gamma \rangle = -1$. Direct computation using~\eqref{eq:conj-h-x}--\eqref{eq:dennis-stein},\eqref{eq:symbol-properties} shows that
 \begin{multline*}
 x_\alpha(aX^{-1}) \cdot x_{-\alpha}(mX) = {}^{h_\gamma(X)}(x_\alpha(a) \cdot x_{-\alpha}(m)) = \\
 = {}^{h_\gamma(X)}\left( x_{-\alpha}\left(\tfrac{m}{1+am}\right) \cdot \langle a, m\rangle_\alpha \cdot h_\alpha(1+am) \cdot x_\alpha\left(\tfrac{a}{1+am}\right) \right) = \\
 = x_{-\alpha}\left(\tfrac{mX}{1+am}\right) \cdot \langle a, m\rangle_\alpha \cdot h_\alpha(X^{-1}(1+am))\cdot h_\alpha^{-1}\left(X^{-1}\right) \cdot x_{\alpha}\left(\tfrac{aX^{-1}}{1+am}\right) = \\
 = x_{-\alpha}\left(\tfrac{mX}{1+am}\right) \cdot \langle a, m\rangle_\alpha \cdot \{X^{-1}, 1+am\} \cdot h_\alpha(1+am)\cdot x_{\alpha}\left(\tfrac{aX^{-1}}{1+am}\right) = \\
 = x_{-\alpha}\left(\tfrac{mX}{1+am}\right) \cdot \langle a, m\rangle_\alpha \cdot h_\alpha(1+am) \cdot x_{\alpha}\left(\tfrac{aX^{-1}}{1+am}\right) \cdot \{1+am, X\}, \end{multline*}
which implies the assertion. \end{proof}

\begin{corollary}\label{SR:additivity} For $g \in P_\alpha(m)$ holds $S_\alpha(a, g) \cdot h_\alpha^{-1}(1 + am) \in P(\alpha, \tfrac{m}{1 + am})$.
Consequently one has \[ S_\alpha(a+b, g) = S_\alpha\left(b,\ S_\alpha(a, g) \cdot h_\alpha^{-1}(1 + am)\right)\cdot h_\alpha(1+am). \]\end{corollary} \begin{proof}
Fix an element $g \in P_\alpha(m)$ and write it $g = g_0 \cdot x_{-\alpha}(mX)$ for some $g_0 \in P_\alpha(0)$.
Now by~\eqref{eq:Smult} and~\cref{P0_conj} one has
\begin{multline} \nonumber S_\alpha(a, g) \cdot h_{\alpha}^{-1}(1+am) = S_\alpha(a, g_0) \cdot S_\alpha(a, x_{-\alpha}(mX)) \cdot h_{\alpha}^{-1}(1+am) = 
\\ = S_\alpha(a, g_0) \cdot x_{-\alpha}\left(\tfrac{mX}{1+am}\right) \cdot\langle a, m \rangle \in P_\alpha\left(\tfrac{m}{1+am}\right). \end{multline}

The second assertion can be verified directly using~\eqref{eq:symbol-properties}. \end{proof}

\subsection{The subgroups $K(\alpha, \beta)$ and their properties} \label{sec:Pa0-Steinberg}
Throughout this subsection $\alpha, \beta$ denote a fixed pair of roots of $\Phi$ forming a sharp angle (i.\,e. $\langle \alpha, \beta \rangle = 1$). We denote by $\Psi$ the subsystem of type $\rA_2$ generated by $\alpha$ and $\beta$.
\begin{df} Denote by $K(\alpha, \beta)$ the subgroup of $\St(\Phi, A[X, X^{-1}])$ generated by the following elements (as before, $\xi \in A[X]$, $f\in M[X]$):
 \begin{enumerate}
  \item $z_\gamma(Xf, \xi)$, for all $\gamma \in \Phi \setminus \Psi$;
  \item $x_{-\alpha}(X^2f)$, $x_{-\beta}(X^2f)$;
  \item $x_{\alpha}(Xf)$, $x_\beta(Xf)$;
  \item $x_{\alpha-\beta}(Xf)$, $x_{\beta-\alpha}(Xf)$;
  \item $c_{\delta}(f, X\xi)$ for some fixed root $\delta \in \Phi \setminus \Psi$. \end{enumerate} \end{df}
  
\begin{prop} \label{K-a-b} Every element $g \in K(A[X], M[X])$ can be written as a product $g_0 \cdot x_{-\alpha}(mX) \cdot x_{-\beta}(m'X)$ for some $m, m' \in M$ and $g_0 \in K(\alpha, \beta)$. 
Consequently, $K(\alpha, \beta)$ coincides with the intersection $K(A[X], M[X]) \cap P_\alpha(0) \cap P_\beta(0)$. 
In particular, $K(\alpha, \beta)$ does not depend on the choice of the root $\delta$. \end{prop}
\begin{proof} Denote by $\Sigma$ the special part of the parabolic subset $S = \Psi \cup \Phi^+$. 
Applying Stepanov theorem to $S$ we obtain that every element of $\St(\Phi, A[X], XM[X])$ can be presented as a product of $z_\gamma(Xf, \xi)$, 
$\gamma \in \Sigma(S) \subseteq \Phi\setminus \Psi$ and $x_\gamma(Xf)$, $\gamma \in \Psi$. By~\cref{Kgen-strong} we can write $g$ as a product of these generators and generators of type 5.
  
 The next step of the proof is to decompose each factor of the form $x_{\epsilon}(Xf)$, $\epsilon \in \{-\alpha,-\beta\}$ appearing in this product as
 $x_{\epsilon}(X^2f') \cdot x_{\epsilon}(mX)$ and then move $x_{\epsilon}(mX)$ to the rightmost position conjugating by $x_{\epsilon}(mX)$ all factors along the way.
 Notice that no terms of the form $x_{\epsilon}(mX)$, $\epsilon \in \{ -\alpha, -\beta \}$ may appear after simplification of these conjugates, so this process is guaranteed to terminate.
 To see this notice that the extra terms in the right-hand sides of \eqref{eq3-1} and~\eqref{eq3-3} 
 (as well as all terms in the right-hand sides of~\eqref{C1},\eqref{C2} of~\cref{Crels} if one substitutes corresponding values into them) have terms of degree at least $2$ in $X$. 
  
 As a side effect of this transformation, however, factors $x_\alpha(Xf)$ and $x_\beta(Xf)$ appearing in $g$ might transform into factors of the form $z_\alpha(Xf, X\xi)$, and $z_\beta(Xf, X\xi)$. We can express each factor $z_\alpha(Xf, X\xi)$ through the generators of $K(\alpha, \beta)$ as follows. Pick a root $\gamma \in \Phi\setminus \Psi$ forming an obtuse angle with $\alpha$ and then invoke~\eqref{eq:zalpha}:
 \begin{multline} z_\alpha(Xf, X\xi) = x_{\alpha+\gamma}(\epsilon X^2f) \cdot x_{\gamma}(-X^3 \xi f) \cdot x_{-\gamma}(Xf\xi) \cdot x_\alpha(Xf) \cdot \\ 
 \cdot z_{\alpha+\gamma}(-\epsilon X^2 f, -\epsilon \xi) \cdot z_{-\gamma}(-Xf\xi, -X) \cdot x_{-\alpha-\gamma}(-\epsilon X^2\xi^2 f) \cdot x_{-\alpha}(-X^3 f \xi^2),\ \text{for } \epsilon = N_{\alpha, \gamma}. \end{multline}
 The expression in the right-hand side is a product of generators of type 1, 1, 1, 3, 1, 1, 1, 2. Similar argument also applies to $z_\beta(Xf, X\xi)$.  
 
 After collecting terms $x_\epsilon(mX)$, $\epsilon \in \{ -\alpha, -\beta \}$ in the right-hand side of the resulting expression we obtain the desired decomposition for $g$.
 
 It is clear from the definition that $K(\alpha, \beta) \subseteq P_\alpha(0) \cap P_\beta(0) \cap K(A[X], M[X])$. The reverse inclusion follows from the just proved decomposition and the fact that for $g\in K(A[X], M[X])$ one has $p_{\alpha}(g) = m$ and $p_\beta(g) = m'$ (cf.~\cref{rem:palpha}). \end{proof}
 
We need one more technical definition. We denote by $Z_{\alpha, \beta}$ the subgroup of $\St(\Phi, A[X, X^{-1}])$ generated by elements $z_\gamma(f, X\xi)$, where $\gamma \in Z_+(\alpha) \setminus \{ \alpha - \beta \}$.
  
\begin{lemma} \label{image-K-a-b} The image of $K(\alpha, \beta)$ under $S_{\alpha}(a, -)$ is contained in the subgroup of $\St(\Phi, A[X, X^{-1}])$  generated by $K(\alpha, \beta)$ and $Z_{\alpha, \beta}$. \end{lemma}
\begin{proof} To simplify notation we call the generators of the subgroup $Z_{\alpha,\beta}$ ``generators of type Z``. By~\eqref{eq:Zconj1}, \eqref{eq:Zconj2} the conjugate to each generator $z_\gamma(Xf, \xi)$ of type 1 by $x_\alpha(aX^{-1})$ is a product of generators of type Z, Z, 1.
By~\eqref{eq:Zconj3} the conjugate to the generator $x_{-\alpha}(X^2f)$ is a product of generators of type 1, Z, 1, 2, 1, 1, Z, Z\@.
The conjugation by $x_\alpha(aX^{-1})$ fixes both generators of type 3 and transforms $x_{-\beta}(X^2f)$ into a product of generators of type 2, 1. As for the remaining generators of type 5, \cref{Crels} shows that the corresponding conjugate can be written as a product of generators of type Z, Z, 5. \end{proof}  

\begin{corollary} The image of $K(\alpha, \beta)$ under $S_\alpha(a, -)$ is contained in $P_\beta(0)$. \end{corollary}
\begin{proof} Follows from the above lemma and the fact that the generators of $Z_{\alpha,\beta}$ are generators of type 2 for $P_\beta(0)$ in the sense of~\cref{defP0}.
 Indeed, if $\gamma \in Z_+(\alpha) \setminus \{ \alpha - \beta \}$ is such that $\langle \beta, \gamma \rangle = -1$, then $ \langle \alpha - \beta, \gamma \rangle \geq 2$, hence
  $\gamma = \alpha - \beta$, a contradiction. \end{proof}

\begin{lemma} \label{SR:sharp} For $g \in K(A[X], M[X]) \cap P_\alpha(m) \cap P_\beta(m')$ and $a \in A$ holds \[S_\alpha(a, g)\cdot  h^{-1}_\alpha(1 + am) \cdot x_{\alpha-\beta}(-N_{\alpha,-\beta}\cdot am') \in P_\beta\left(\tfrac{m'}{1 + am}\right). \] \end{lemma}
\begin{proof} By~\cref{K-a-b} $g$ can be presented as $ g_0 \cdot x_{-\alpha}(mX) \cdot x_{-\beta}(m'X)$ for some $g_0\in K(\alpha, \beta)$, therefore from~\eqref{eq:Smult} and~\cref{lem:Tulenbaev-formula} we obtain that \begin{multline} \nonumber S_\alpha(a, g) = S_\alpha(a, g_0) \cdot S_\alpha(a, x_{-\alpha}(mX)) \cdot S_\alpha\left(\tfrac{a}{1+am}, x_{-\beta}(m'X)\right) = \\ = S_\alpha(a, g_0) \cdot x_{-\alpha}\left(\tfrac{mX}{1+am}\right) \cdot \langle a, m \rangle \cdot h_\alpha(1+am) \cdot x_{-\beta}(m'X) \cdot x_{\alpha-\beta}\left(\tfrac{N_{\alpha, -\beta} \cdot am'}{1+am}\right) = \\ = S_\alpha(a, g_0) \cdot x_{-\alpha}\left(\tfrac{mX}{1+am}\right) \cdot \langle a, m \rangle \cdot  x_{-\beta}\left(\tfrac{m'X}{1+am}\right) \cdot x_{\alpha-\beta}\left(N_{\alpha, -\beta} \cdot am'\right) \cdot h_\alpha(1+am). \end{multline}
The assertion now follows from the preceding corollary.
\end{proof}
  
\begin{lemma} \label{conj-K-a-b} For $b \in A$ one has $K(\alpha, \beta)^{x_{\beta - \alpha}(b)} \subseteq K(\alpha, \beta)$. \end{lemma}
\begin{proof} By~\eqref{Z2}--\eqref{Z4} of~\cref{Zrels} the conjugate to the generator $g=z_\gamma(Xf, \xi)$ of type 1 by $x_{\beta-\alpha}(b)$ either coincides with $g$ or is a product of generators of type 4, 1, 1. Now if $g = x_\epsilon(Xf)$, $\epsilon \in \{\pm \alpha, \pm \beta\}$ is a generator of type 2 or 3, the conjugation by $x_{\beta-\alpha}(a)$ either fixes $g$ or transforms it into a product of two generators of type 2 or 3. By~\eqref{C1}--\eqref{C3} of~\cref{Crels} the conjugate to a generator $g = c_\delta(f, X\xi)$ of type 5 either coincides with $g$ or is a product of generators of type 5, 4, 1. Finally, from~\cref{rem:palpha} we obtain that $z_{\alpha-\beta}(Xf, b) \in K(A[X], M[X]) \cap P_\alpha(0) \cap P_{\beta}(0) = K(\alpha, \beta)$. \end{proof} 

\begin{lemma} For $g \in K(\alpha, \beta)$ and $a, b\in A$ the conjugate to $S_\alpha(a, g)$ by $x_{\beta-\alpha}(b)$ belongs to $P_\alpha(0)$. \end{lemma}
\begin{proof} By~\cref{image-K-a-b} and~\cref{conj-K-a-b} it remains to show that the conjugate to each generator $g = z_\gamma(f, X\xi)$, $\gamma \in Z_+(\alpha) \setminus \{ \alpha - \beta \}$ of $Z_{\alpha, \beta}$ by $x_{\beta-\alpha}(b)$ belongs to $P_\alpha(0)$. Throughout the proof we understand generator types according to~\cref{defP0}. It is clear that $g$ itself is a generator of type 2, therefore it remains to consider the case $\gamma\not \perp \beta - \alpha$. 
 
 In the case $\alpha - \beta - \gamma \in \Phi$ we obtain from \eqref{Z2} of~\cref{Zrels} that
\[ z_{\gamma}(f, X\xi) ^ {x_{\beta-\alpha}(b)} = x_{\beta-\alpha} (- bXf\xi) \cdot x_{-\alpha + \gamma + \beta} (N_{\gamma, \beta -\alpha}\cdot bf)     \cdot z_{\gamma}(f, X\xi)\ \text{if}\ \alpha - \beta - \gamma \in \Phi. \] Notice that $\langle \alpha - \beta - \gamma, \alpha \rangle = \langle \alpha, \alpha \rangle - \langle \beta, \alpha \rangle - \langle \gamma, \alpha \rangle \leq 2 - 1 - 1 = 0$, 
 hence $\langle \alpha - \beta - \gamma, \alpha \rangle = -1,0$ and the expression in the right-hand side is a product of generators of type 1, 2, 2 or 1, 3, 2.

Now suppose that the other alternative holds, namely that $\alpha - \beta + \gamma \in \Phi$.
Since $\langle \alpha - \beta + \gamma, \alpha \rangle = \langle \alpha , \alpha \rangle - \langle \beta, \alpha \rangle + \langle \gamma, \alpha \rangle \geq 2 - 1 + 1 = 2$
we obtain that $\alpha - \beta + \gamma = \alpha$. Thus, by \eqref{Z3} of~\cref{Zrels}
\[z_{\gamma}(f, X\xi) ^ {x_{\beta-\alpha}(b)} = x_{\beta-\alpha} (bXf\xi) \cdot x_{-\alpha} (N_{\gamma,\alpha - \beta}\cdot bX^2 f\xi^2) \cdot z_{\gamma}(f, X\xi). \]
The expression in the right-hand side is a product of generators of type 1, 4, 2. \end{proof}

\begin{lemma} \label{SR:obtuse} For $g \in K(A[X], M[X]) \cap P_\alpha(m) \cap P_\beta(m')$ and $a, b\in A$ the element \[ x_{\beta - \alpha}(b) \cdot S_\alpha(a, g) \cdot h_{\alpha}^{-1}(1+am)\cdot h_{\beta - \alpha}((1 + \epsilon abm')^{-1})\cdot x_{\beta - \alpha}(-b(1 + \epsilon abm')) \]
 belongs to $P_\alpha\left(\tfrac{m - \epsilon bm'}{1+am}\right)$, where $\epsilon = N_{\alpha, -\beta}$.
\end{lemma}
\begin{proof} Set $c = 1 + \epsilon a b m'$, $s_0 = \langle a, m \rangle$, $s_1 = \langle \epsilon am', -bc^{-1} \rangle$.
From the definition of $S_\alpha(a, -)$ we obtain that
 \begin{multline} \label{eq:SRobtuse-computation}
 S_\alpha(a, x_{-\beta}(m'X) \cdot x_{-\alpha}(mX)) \cdot h_{\alpha}^{-1}(1+am)\cdot h_{\beta-\alpha}(c^{-1})\cdot x_{\beta - \alpha}(-bc) = \\ 
 = S_\alpha(a, x_{-\beta}(m'X)) \cdot S_\alpha(a, x_{-\alpha}(mX)) \cdot h_\alpha^{-1} (1 + am) \cdot h_{\beta - \alpha}(c^{-1}) \cdot x_{\beta - \alpha}(-bc) = \\
 = x_{-\beta}(m'X) \cdot x_{\alpha - \beta}(\epsilon am') \cdot x_{\beta - \alpha}(-bc^{-1}) \cdot x_{-\alpha}\left(\tfrac{mX}{1+am}\right) \cdot s_0 \cdot h_{\beta - \alpha}(c^{-1}) = \\
 = x_{-\beta}(m'X) \cdot x_{\beta - \alpha}(-b) \cdot s_1 \cdot h_{\alpha-\beta}(c^{-1}) \cdot x_{\alpha-\beta}\left(\epsilon acm'\right) \cdot x_{-\alpha}\left(\tfrac{mX}{1+am}\right) \cdot s_0 \cdot h_{\beta - \alpha}(c^{-1}) = \\
 =  x_{\beta - \alpha}(-b) \cdot x_{-\beta}(m'X) \cdot x_{-\alpha}(-\epsilon bm'X) \cdot s_1 \cdot x_{\alpha-\beta}(\epsilon ac^{-1}m') \cdot x_{-\alpha}\left(\tfrac{cmX}{1+am}\right) \cdot s_0 = \\
 = x_{\beta-\alpha}(-b) \cdot x_{-\beta}\left(m'X(1 - \epsilon abm'c^{-1})\right) \cdot x_{\alpha-\beta}(\epsilon ac^{-1}m') \cdot s_1 \cdot s_0 \cdot x_{-\alpha}\left(-\epsilon bm' X + \tfrac{cmX}{1+am}\right).
\end{multline}
Denote by $h_0$ the product of all terms except the first and the last one in the above formula. It is clear that $h_0$ lies in $P_\alpha(0)$. Using~\cref{K-a-b} we can decompose $g = g_0 \cdot x_{-\beta}(m'X) \cdot x_{-\alpha}(mX)$ for some $g_0 \in K(\alpha, \beta)$. The required assertion now follows from \eqref{eq:Smult}, \eqref{eq:SRobtuse-computation} and the preceding lemma:
\begin{multline} \nonumber
 x_{\beta - \alpha}(b) \cdot S_\alpha(a, g) \cdot h_{\alpha}^{-1}(1+am)\cdot h_{\beta - \alpha}(c^{-1})\cdot x_{\beta - \alpha}(-bc) = \\
  = x_{\beta - \alpha}(b) \cdot S_\alpha(a, g_0) \cdot x_{\beta-\alpha}(-b) \cdot h_0 \cdot x_{-\alpha}\left(\tfrac{m -\epsilon bm'}{1+am}\right) \in P_\alpha(\tfrac{m -\epsilon bm'}{1+am}). \qedhere
\end{multline}
\end{proof}

\subsection{Construction of a $\St(\Phi, B)$-torsor.} \label{sec:V-construction}
Throughout this subsection $\Phi$ denotes arbitrary irreducible simply-laced root system of rank $\geq 3$, unless stated otherwise. Let $A$ be a local ring with maximal ideal $M$ and residue field $k$. As in~\cref{firstPart} we denote by $B$ the ring $A[X\inv] + M[X]$ considered as a graded subring of the ring of Laurent polynomials $R = A[X, X\inv]$.

For shortness we denote the subgroup $\overline{\St}(\Phi, A, M)$ by $G_M^0$.
Notice that $G_M^0$ can be considered as a subgroup of $\St(\Phi, R)$.
\[ \xymatrix{ G_M^0 \ar@{^{(}->}[r]^{i_+} \ar@{^{(}->}[d]^{i_-} & \overline{\St}(\Phi, A[X], M[X]) \ar^{j_+^M}[d] \ar@{^{(}->}[r] & \St(\Phi, A[X]) \ar[dl]^{j_+} \\
              \St(\Phi, A[X\inv]) \ar[r]^{j_-} & \St(\Phi, R). &  } \] 
We denote by $\overline{G}^{\geq 0}_M$ the image of the map $j_+^M$.               
Denote by $\overline{V}_T$ the quotient of the set of triples 
\[V_T = \overline{G}_M^{\geq 0} \times \St(\Phi, A[X\inv]) \times (1+M)^\times\] by the equivalence relation given by $(p \gamma, h, u)_T \sim (p, \gamma h, u)_T$, $\gamma \in G^0_M$. We denote the image of $(p, h, u)_T$ in $\overline{V}_T$ by $[p, h, u]$.              
$\overline{V}_T$ is precisely the set upon which M.~Tulenbaev constructs an action of $\St(\Phi, B)$ in~\cite[Proposition~4.3]{Tu83}.

Sometimes it will be more convenient for us to work with another set $\overline{V}$ isomorphic to $\overline{V}_T$. Denote by $V$ the subset of $\St(\Phi, R) \times \St(\Phi, A[X\inv]) \times (1 + M)^\times$ consisting of those triples $(g, h, u)$ for which $p(g, h, u) := g \cdot j_-(h) \cdot \{ X, u \}$ belongs to $\overline{G}_M^{\geq 0}$. 

We let $h_0 \in G_M^0$ act on $V$ on the right by $(g, h, u) \cdot h_0 = (g, h \cdot i_-(h_0), u)$. This action is well defined since $G^0_M \subset \overline{G}^{\geq 0}_M$.
We denote by $\overline{V}$ the set of orbits of this action of $G_M^0$ and use the notation $(g, [h], u)$ for the elements of $\overline{V}$.
Whenever $v_1, v_2 \in V$ lie in the same $G_M^0$-orbit we use the notation $v_1 \sim v_2$.
We denote by $\overline{p}$ the function $\overline{V} \to \overline{G}^{\geq 0}_M/G_M^0$ sending each $(g, [h], u) \in V$ to the left coset $p(g, h, u)G_M^0$.

The isomorphism between the sets $\overline{V}$ and $\overline{V}_T$ is given by the following two maps, which are easily seen to be mutually inverse to each other:
\begin{equation} \label{eq:VVT} \xymatrix @R0.3pc {\overline{V} \ar[r]^{\cong} & \ar[l] \overline{V}_T \\ (g, [h], u) \ar@{|->}[r] & [p(g, h, u), h^{-1}, u] \\ (p \cdot j_-(h) \cdot \{u, X\}, [h^{-1}], u) & \ar@{|->}[l] [p, h, u]. } \end{equation}
The above isomorphism allows us to regard $\overline{V}$ and $\overline{V}_T$ as one and the same object, for which we can interchangeably use either of the two notations,
 depending on which of them is more convenient in a given situation.
For example, specifying the action of $\St(\Phi, B)$ in terms of $\overline{V}$ leads to much shorter calculations in Lemmas~\ref{R3_leq0_1}--\ref{R2_0_1},
 while the statements of~\cref{prop43} and~\cref{lem:action} look more natural when formulated in terms of~$\overline{V}_T$.

Now we are ready to proceed with the construction of the action of $\St(\Phi, B)$ on $\overline{V}$. We start by defining for $\alpha \in \Phi$, $a \in A$ a partial function $t_\alpha(aX^{-1}) \colon V \not\to V$.
This function is defined for the triples $(g, h, u)$ satisfying $p(g, h, u) \in P_\alpha(*) \subseteq \overline{G}_M^{\geq 0}$.
If $p(g, h, u)$ belongs to $P_\alpha(m)$ for some $m \in M$, then $t_\alpha(aX^{-1})$ is defined via the following identity:
\begin{equation} \label{T_1} t_\alpha(aX^{-1}) (g, h, u) = \left( x_\alpha(aX^{-1})\cdot g ,\ h \cdot x_\alpha\left(-\tfrac{aX^{-1}}{1 + am}\right),\ u \cdot (1 + am)\right).\end{equation}

From~\cref{SR:additivity} it follows that
\[p\left(t_\alpha(aX^{-1}) (g, h, u)\right) = S_\alpha(a, p(g, h, u)) \in P_\alpha(*) \cdot h_{\alpha}^{-1}(1+am) \subseteq G_M^{\geq 0},\]
therefore $t_\alpha(aX^{-1})$ is well-defined.

\begin{lemma}\label{lem:orbit-action} Let $(A, M)$ be a local pair. 
Then for any $\alpha \in \Phi$ and $a \in A$ the partial function $t_\alpha(aX^{-1}) \colon V \not \to V$ gives rise to a well-defined total function $T_\alpha(aX^{-1}) \colon \overline{V} \to \overline{V}$. \end{lemma}
\begin{proof}
 First of all, let us show that the resulting function is total. 
 Fix $v_0 = (g, h, u) \in V$. Since $p(g, h, u) \in \overline{G}^{\geq 0}_M$ there exists $g_1 \in \overline{\St}(\Phi, A[X], M[X])$ such that $j_+(g_1) = p(g, h, u).$
 Set $h_0 = ev^*_{X=0}(g_1)^{-1}$, then, clearly, $g_1 \cdot i_+(h_0) \in K(A[X], M[X])$ and 
 $p(g, h \cdot i_-(h_0), u) = j_+(g_1) \cdot j_-(i_-(h_0)) = j_+(g_1 \cdot i_+(h_0)) \in j_+(K(A[X], M[X])) \subseteq P_\alpha(*)$, which shows that $t_\alpha(aX^{-1})$ is defined on the representative $(g, h \cdot i_+(h_0), u)$ lying in the same $G_M^0$-orbit as $v_0$.
  
 Next, let us show that the value of $T_\alpha(aX^{-1})$ does not depend on a choice of the representative.
 Let $v_1 = (g, h_1, u)$ and $v_2 = (g, h_2, u)$ be two elements of the same $G_M^0$-orbit for which both $p(v_1)$ and $p(v_2)$ belong to $P_\alpha(*)$.
 By definition, $h_1^{-1} h_2 = i_-(h_0)$, for some $h_0 \in G^0_M$, moreover, 
  $p(v_1)^{-1} \cdot p(v_2) = j_- i_-(h_0) = j_+ i_+(h_0) \in P_\alpha(*)$.
 By~\cref{Pstar-char} there exists $g_1 \in \overline{\St}(\Phi, A[X], M[X])$ such that $g_0 := ev^*_{X=0}(g_1) \in Z_\alpha(A, M)$ and $j_+(g_1) = j_+ i_+ (h_0)$.
 From the last equality and the injectivity of the map $G(\Phi, A[X]) \to G(\Phi, R)$ we obtain that the projections of $g_0$, $g_1$ and $h_0$ in $G(\Phi, R)$ are equal, which shows that $g_0 \cdot h_0^{-1} \in \overline{\K_2}(\Phi, A, M)$. It follows from~\cref{thm:Stein} that the latter subgroup is generated by relative Steinberg symbols $\{ a, 1 + m \}$ and hence by~\cref{Z-DS} it is contained in $Z_\alpha(A, M)$. Thus, we have obtained that $h_0 \in Z_\alpha(A, M)$ and hence that $i_-(h_0)$ is centralized by $X_\alpha(A[X^{-1}])$, which allows us to conclude that
 \begin{multline} \nonumber
  t_\alpha(aX^{-1})(v_1) = \left( x_\alpha(aX^{-1})\cdot g ,\ h_1 \cdot x_\alpha\left(-\tfrac{aX^{-1}}{1 + am}\right),\ u \cdot (1 + am)\right) \sim \\
  \sim \left( x_\alpha(aX^{-1})\cdot g ,\ h_1 i_-(h_0) \cdot x_\alpha\left(-\tfrac{aX^{-1}}{1 + am}\right),\ u \cdot (1 + am)\right) = t_\alpha(aX^{-1})(v_2). \qedhere
 \end{multline} \end{proof}

For $a + Xf \in A + XM[X]$ we can define the operator $T_\alpha(a + Xf) \colon \overline{V} \to \overline{V}$ by the following identity:
\begin{equation} \label{T_leq0} T_\alpha(a + Xf) \cdot (g, [h], u) = (x_\alpha(a + Xf) \cdot g, [h \cdot x_{\alpha}(-a)], u).  \end{equation}
It is easy to see that this definition is correct.

Thus far we have specified the action of the generators of the "truncated'' Steinberg group $\St^{\leq 1}(\Phi, B)$ from~\cref{sec:presentation} on $\overline{V}$ using formulae~\eqref{T_1} and~\eqref{T_leq0}. We need to verify that this action respects the defining relations of the group $\St^{\leq 1}(\Phi, B)$ (notice that $\leq 1$ here stands for the degree of relations with respect to $t = X^{-1}$). This is accomplished in the series of lemmas below.

\begin{lemma} \label{R3_leq0_1} The operators $T_\alpha$ satisfy Steinberg relations of type $\mathrm{R3}_{d, 1}$ for $d\leq 0$. \end{lemma}
\begin{proof} We need to show that $[T_\alpha(aX^{-1}), T_\beta(b)](g, [h], u) = (g, [h], u)$ for $\beta\in Z(\alpha)$, $a\in A$, $b\in A + XM[X]$.
Write $b = b_0 + Xf$ for some $b_0 \in A$, $f \in M[X]$.
We may assume that $p(g, h, u) \in P_\alpha(m)$ for some $m \in M$  (cf. the first part of the proof of~\cref{lem:orbit-action}).
It follows from~\cref{P0-conj-invariant} that $P_\alpha\left(\tfrac{m}{1-am}\right)$ is stable under conjugation with $x_\beta(b_0)$, therefore by~\cref{SR:additivity} we have
\begin{multline} \nonumber
  \left[T_\beta(b),\ T_\alpha(aX^{-1}) \right] (g, [h], u) = \\
  = T_\beta(b) \cdot T_\alpha(aX^{-1}) \left(x_\beta(-b) \cdot x_\alpha(-aX^{-1}) \cdot g, [h'], u(1-am)\right) = \\ = \left( [x_\beta(b), x_{\alpha}(aX^{-1})] \cdot g, [h''], u] \right) = \left(g, [h], u\right), \end{multline}
where $h' = h \cdot x_{\alpha}\left(\tfrac{aX^{-1}}{1-am}\right) \cdot h^{-1}_\alpha(1-am) \cdot x_\beta(b_0)$ and
\begin{multline} \nonumber
 h'' = h' \cdot x_{\alpha}\left(-a(1-am)X^{-1}\right) \cdot x_{\beta}(-b_0) = \\
 = h \cdot h_\alpha^{-1}(1-am) \cdot \left[x_\alpha\left(a(1-am)X^{-1}\right),\ x_\beta(b_0)\right] = h \cdot h_\alpha^{-1}(1-am). \qedhere \end{multline}
\end{proof}

\begin{lemma} \label{R3_leqm1_1} The operators $T_\alpha$ satisfy Steinberg relations of type $\mathrm{R2}_{d, 1}$ for $d \leq -1$ . \end{lemma}
\begin{proof} We need to show that for $a\in A$, $f \in M[X]$ and $\alpha, \beta \in \Phi$ forming an obtuse angle holds
 \[ [T_\beta(Xf),\ T_\alpha(aX^{-1})] (g, [h], u) = \left(x_{\alpha+\beta}(N_{\beta,\alpha} \cdot af) \cdot g, [h], u\right). \]

As before, we may assume that $p(g, h, u) \in P_\alpha(m)$ for some $m \in M$.
Since $x_\beta(Xf)$ belongs to $P_\alpha(0)$ we obtain from~\cref{SR:additivity} that:
\begin{multline} \nonumber [T_\beta(Xf),\ T_\alpha(aX^{-1})] \left(g,\ [h],\ u\right) = \\ 
= T_\beta(Xf) \cdot T_\alpha(aX^{-1}) \left(x_{\beta}(-Xf) \cdot x_\alpha(-aX^{-1})\cdot g,\ \left[h \cdot x_\alpha\left(\tfrac{aX^{-1}}{1 - am}\right) \cdot h^{-1}_\alpha(1-am)\right],\ u(1-am)\right) = \\ = \left([x_\beta(Xf),\ x_\alpha(aX^{-1})] \cdot g,\ \left[h \cdot x_\alpha\left(-\tfrac{aX^{-1}}{1-am}\right) \cdot h^{-1}_\alpha(1-am) \cdot x_\alpha\left(\tfrac{aX^{-1}}{1 + \tfrac{am}{1-am}}\right)\right],\ u \right) = \\ = (x_{\alpha+\beta}(N_{\beta,\alpha} \cdot af) \cdot g,\ [h],\ u). \qedhere \end{multline} \end{proof}

\begin{lemma} \label{R3_1_1} The operators $T_\alpha(aX^{-1})$ satisfy Steinberg relations $\mathrm{R3}_{1,1}^{\angle}$. \end{lemma}
\begin{proof}
Let $\alpha, \beta$ be a pair of roots such that $\alpha$ and $\beta$ form a sharp angle.
From~\cref{K-a-b} and the proof of the first part of~\cref{lem:orbit-action} it follows that \[p(g, h, u) \in j_+\left(K(A[X], M[X]) \cap P_\alpha(m) \cap P_{\beta}(m')\right)\text{ for some }m,m' \in M.\]
Set $\epsilon = N_{\alpha,-\beta}$. Applying~\cref{SR:sharp} we obtain that
\begin{multline} \nonumber
 T_\beta(bX^{-1}) \cdot T_\alpha(aX^{-1}) \left(g, [h], u\right) = 
 T_\beta(bX^{-1}) \left(x_\alpha(aX^{-1})\cdot g,\ \left[h \cdot x_\alpha\left(-\tfrac{aX^{-1}}{1 + am}\right)\right],\ u(1 + am)\right) = \\ 
 = T_\beta(bX^{-1}) \left(x_\alpha(aX^{-1})\cdot g,\ \left[h \cdot x_\alpha\left(-\tfrac{aX^{-1}}{1 + am}\right) \cdot h^{-1}_\alpha(1 + am) \cdot x_{\alpha-\beta}(-\epsilon am')\right],\ u(1 + am)\right) = \\
 = (x_\beta(bX^{-1}) x_\alpha(aX^{-1}) \cdot g, [h'], u(1 + am + bm')), \end{multline}
where 
\begin{multline} \nonumber
 h' = h \cdot x_\alpha\left(-\tfrac{aX^{-1}}{1 + am}\right) \cdot h^{-1}_\alpha(1 + am) \cdot x_{\alpha-\beta}(-\epsilon am') \cdot x_\beta\left(-\tfrac{bX^{-1}(1+am)}{1 + am + bm'}\right) \sim \\
    \sim h \cdot x_\alpha\left(-\tfrac{aX^{-1}}{1 + am}\right) \cdot h^{-1}_\alpha(1 + am) \cdot x_{\alpha}\left(\tfrac{m'abX^{-1}(1+am)}{1 + am + bm'}\right) \cdot x_\beta\left(-\tfrac{bX^{-1}(1+am)}{1 + am + bm'}\right) = \\
 = h \cdot x_\alpha\left(-\tfrac{aX^{-1}}{1 + am} + \tfrac{m'abX^{-1}}{(1+am)(1 + am + bm')}\right) \cdot x_\beta\left(-\tfrac{bX^{-1}}{1 + am + bm'}\right) \cdot h^{-1}_\alpha(1 + am) \sim \\
 \sim h \cdot x_\alpha\left(-\tfrac{aX^{-1}}{1 + am + bm'}\right) \cdot x_\beta\left(-\tfrac{bX^{-1}}{1 + am + bm'}\right).
\end{multline}
In the last computation notation $h \sim h'$ means that $h$ and $h'$ lie in the same left $G_M^0$-coset.
If we swapped $(a,\alpha,m)$ with $(b,\beta,m')$, 
the expressions in the right-hand sides of the above formulae would remain unchanged. This implies the required assertion.
\end{proof}

\begin{lemma} \label{R2_0_1} The operators $T_\alpha(aX^{-1})$ satisfy Steinberg relations $\mathrm{R2}_{0,1}$. \end{lemma}
\begin{proof} 
Let $\alpha, \beta$ be a pair of roots such that $\alpha + \beta \in \Phi$.
As before, we may assume $p(g, h, u) \in j_+(K(A[X], M[X]) \cap P_\alpha(m) \cap P_{\alpha + \beta}(m'))$ for some $m, m' \in M$.
Now for $a, b \in A$ set $\epsilon = N_{\alpha, \beta},$ $c = 1 - \epsilon abm'$.
Applying~\cref{SR:obtuse} to the pair of roots $\alpha, \alpha+\beta$ we obtain that 
\begin{multline} \nonumber
[T_\beta(b), T_\alpha(aX^{-1})] \left(g, [h], u \right) = \\
 = T_\beta(b) \cdot T_\alpha(aX^{-1}) \left(x_\beta(-b) \cdot x_\alpha (-a X^{-1}) g,\ [h'],\ u (1 - am) \right) = \\
 = \left([x_\beta(b), x_\alpha(aX^{-1})]\cdot g,\ [h''],\ u c\right) = T_{\alpha+\beta}(-\epsilon abX^{-1})(g,\ [h],\ u),\end{multline} 
 where $h' = h\cdot x_\alpha\left(\tfrac{a X^{-1}}{1-am}\right) \cdot h_{\alpha}^{-1}(1-am)\cdot h_{\beta}(c^{-1}) \cdot x_{\beta}(bc)$  
  and the last equality is obtained as follows:
 \begin{multline} \nonumber
 h'' = h' \cdot x_\alpha\left(-\tfrac{a(1 - am)X^{-1}}{c}\right) \cdot x_\beta(-b) \sim \\ \sim  h\cdot x_\alpha\left(\tfrac{a X^{-1}}{1-am}\right) \cdot h_{\alpha}^{-1}(1-am)\cdot h_{\beta}(c^{-1})\cdot x_{\alpha + \beta}\left(\epsilon ab(1-am)X^{-1}\right)\cdot x_\alpha\left(-\tfrac{a(1 - am)X^{-1}}{c}\right) = \\
 = h\cdot x_\alpha\left(\tfrac{a X^{-1}}{1-am}\right) \cdot x_{\alpha + \beta}\left(\tfrac{\epsilon \cdot abX^{-1}}{c}\right)\cdot x_\alpha\left(-\tfrac{aX^{-1}}{1 - am}\right) \cdot h_{\alpha}^{-1}(1-am)\cdot h_{\beta}(c^{-1}) \sim h\cdot x_{\alpha + \beta}\left(\tfrac{\epsilon   abX^{-1}}{1 - \epsilon a b m'}\right). \qedhere
\end{multline}
\end{proof}

Now we are ready to prove the main result of this subsection.
\begin{prop} \label{prop43}
For $\Phi$ as in the statement of~\cref{lemma33}.
the operators $T_\alpha$ defined above specify a well-defined action of $\St(\Phi, B)$ on $\overline{V}$. 

This action satisfies the following additional properties.
\begin{enumerate}
 \item For any $h_1 \in \St(\Phi, A[X\inv])$ holds $j_B^-(h_1) \cdot [1, h, u] = [1, h_1 h, u]$, where $j_B^-$ denotes the map $\St(\Phi, A[X\inv]) \to \St(\Phi, B)$ (we identify $\overline{V}$ with $\overline{V}_T$ using the isomorphism~\eqref{eq:VVT}).
 \item If we consider $\St(\Phi, A[X, X\inv])$ as a group with the left multiplication action of $\St(\Phi, B)$ then the map $\overline{V} \to \St(\Phi, A[X, X\inv])$ given by $(g, [h], u) \mapsto g$ is a map of $\St(\Phi, B)$-groups.
\end{enumerate}
\end{prop}
\begin{proof} By~\cref{superfluous-relations} and Lemmas~\ref{R3_leq0_1}--\ref{R2_0_1} the action of $\St^{\leq 1}(\Phi, B)$ on $\overline{V}$ given by~\eqref{T_1} and~\eqref{T_leq0} is well-defined. On the other hand, the group $\St^{\leq 1}(\Phi, B)$ is isomorphic to $\St(\Phi, B)$ by~\cref{lemma33} (set $t = X^{-1}$).
  
The first property can be verified directly using~\eqref{T_1}, \eqref{T_leq0} and the fact that $\St(\Phi, A[X\inv])$ is generated by $x_\alpha(a)$ and $x_\alpha(aX\inv)$ for $\alpha\in\Phi$, $a\in A$. The second property can be verified in a similar fashion. \end{proof}

From the second property we also immediately obtain the following group factorization.
\begin{corollary} For $\Phi$ as above one has $\E(\Phi, A[X\inv] + M[X]) = \E(\Phi, A[X], M[X]) \cdot \E(\Phi, A[X\inv]).$ \end{corollary}

\subsection{Proof of Horrocks theorem.} \label{sec:P1glueing}
We retain the notation of the previous section and assume additionally that the map $j_R \colon \St(\Phi, B) \to \St(\Phi, R)$ is injective. 
Being considered as a subgroup of $\St(\Phi, R)$ the group $\St(\Phi, B)$ contains the subgroup $\overline{G}^{\geq 0}_M$ of $\St(\Phi, R)$. This follows from a consideration of the following commutative diagram, in which the map $t$ is obtained from~\cref{lem:lemma32}:
\[ \xymatrix{ \St(\Phi, A[X], M[X]) \ar[d] \ar[r] & \St(\Phi, R, M[X, X\inv]) \ar[r]^(.65){t} & \St(\Phi, B) \ar[d]^{j_R} \\
   \St(\Phi, A[X]) \ar[rr]^{j_+} & & \St(\Phi, R). } \]

\begin{lemma}\label{lem:action} For any $[p, h, u]\in \overline{V}_T$ and any $p_1 \in \overline{G}^{\geq 0}_M$ holds $p_1 \cdot [p, h, u] = [p_1p, h, u]$. \end{lemma}
\begin{proof} First of all, notice that the assertion of the lemma can be verified directly in the case $p_1 = z_\alpha(f, a)$ for all $a\in A$, $f\in M[X]$. Using~\cref{SR:additivity} and the fact that $x_\beta(X^2f) \in P_\alpha(0)$ for all $\beta \in \Phi$ it is not hard to show that the assertion also holds for $p_1 = z_\alpha(X^2f, aX^{-1})$.

It suffices to verify the assertion of the lemma for the generators $p_1 = z_\alpha(f, \xi)$ of $\overline{G}_M^{\geq 0}$ for all $f \in M[X]$, $\xi\in A[X]$, $\alpha \in \Phi$. We accomplish this by 
induction on the degree of $\xi$ in $X$. The base of induction is clear.

Suppose that the assertion holds for all $p_1 = z_\alpha(f, \xi)$ for which $\xi$ has degree $\leq n$.
Substituting $s := fX$, $\eta := X^{-1}$, $\xi := X\xi$ in~\eqref{Z5} we obtain the following equality in $\St(\Phi, B)$:
\begin{multline*}
 z_{\alpha+\beta}(f, X\xi) = x_\alpha(\epsilon Xf) \cdot x_{-\beta}(-faX^2) \cdot x_{\beta}(af) \cdot x_{\alpha+\beta}(f) \cdot \\ \cdot z_\alpha(-\epsilon Xf, -\epsilon \xi) \cdot x_{-\alpha}(-\epsilon \xi^2fX) \cdot x_{-\alpha-\beta}(- f \xi^2X^2) \cdot z_{-\beta}(X^2f\xi, -X^{-1}),\text{ where $\epsilon = N_{\alpha, \beta}$.}
\end{multline*}
From the first part of the proof and inductive assumption we obtain that the assertion of the lemma holds for all the factors in the right-hand side and, therefore, also holds  for $p_1 = z_\alpha(f, X\xi)$. It is easy to deduce from this that the assertion also holds for $p_1 = z_\alpha(f, X\xi + a) = x_\alpha(-a) \cdot z_\alpha(f, X\xi) \cdot x_\alpha(a)$. 
\end{proof}

\begin{rem}
Although we will not need this for our main result, it can be noted that $\overline{V}$ is a left $\St(\Phi, B)$-torsor, i.\,e. the action of $\St(\Phi, B)$ on $\overline{V}$ is both transitive and faithful. The faithfulness follows from the second property of~\cref{prop43} and our assumption that $j_R$ is injective.
The transitivity follows from~\cref{lem:action}, the first property of~\cref{prop43} and the following computation, which we leave as an exercise to the reader:
\begin{equation*} \langle 1, m \rangle^{-1} \langle X^{-1}, mX \rangle \cdot [1, 1, 1] = [1, 1, 1+m]. \end{equation*} \end{rem}

\begin{externaltheorem} \label{thm:P1glueing}
 Let $\Phi$ be a root system of type $\rA_{\geq 4}, \rD_{\geq 5}$ or $\rE_{6,7,8}$.
 Suppose that the map $j_R \colon \St(\Phi, B) \to \St(\Phi, A[X, X\inv])$ is injective.  
 Then the map $j_-$ is injective and the following commutative square is pullback
 \begin{equation} \nonumber
  \xymatrix{ \St(\Phi, A) \ar[r] \ar[d] & \St(\Phi, A[X]) \ar[d]^{j_+} \\ \St(\Phi, A[X\inv]) \ar[r]^{j_-} & \St(\Phi, A[X, X\inv]).}
 \end{equation}
\end{externaltheorem}
\begin{proof} Since $j_- = j_R j_B^-$, where $j_B^-$ is as in~\cref{prop43}, it suffices to show that $j_B^-$ is injective.
 Set $v_0 = [1, 1, 1] \in \overline{V}_T$ and suppose $g \in \Ker(j_B)$.
 By~\cref{prop43} holds $v_0 = j_B^-(g) \cdot v_0 = [1, g, 1]$ therefore $g \in G_M^0$ and consequently $g = 1$.
 
 Now suppose $g_+ \in \St(\Phi, A[X])$ and $g_- \in \St(\Phi, A[X\inv])$ are such that $j_B^-(g_-) = j_+(g_+)$.
 From~\cref{field-injectivity} it follows that the image of $j_+(g_+) = j_-(g_-)$ in $\St(\Phi, k[X, X\inv])$ belongs to $\St(\Phi, k)$ and therefore coincides with the image in $\St(\Phi, k)$ of some $g_0 \in \St(\Phi, A)$.
 
 Set $h_+ = g_+g_0^{-1}$ and $h_- = g_-g_0^{-1}$.
 It is clear that \[ h_+ \in \overline{\St}(\Phi, A[X], M[X]), h_- \in \overline{\St}(\Phi, A[X\inv], M[X\inv])\text{ and that }j_-(h_-) = j_+(h_+).\]
 From~\cref{prop43} and~\cref{lem:action} we obtain that
 \[ [j_+(h_+)^{-1}, h_-, 1] = j_+(h_+)^{-1} \cdot [1, h_-, 1] = j_+(h_+)^{-1} j_B^-(h_-) \cdot [1, 1, 1] = [1, 1, 1]. \]
 Thus, $h_-$ (and hence $h_+$) come from some element of $G_M^0$, which implies that $g_+$ and $g_-$ are images of some element of $\St(\Phi, A)$.
 \end{proof}
  
\begin{proof}[Proof of~\cref{thm:main}]
 Notice that the assertions of the theorem for $\KO_2(2\ell, -)$ and $\K_2(\rD_\ell, -)$ follows from the assertion for $\St(\rD_\ell, -)$.
 In turn, the latter assertion follows from~\cref{thm41} and~\cref{thm:P1glueing} in the special case when $A$ is a local ring.
 
 Now if $g \in \St(\rD_\ell, A[X])$ is such that its image in $\St(\rD_\ell, A[X, X\inv])$ is trivial, then so is its image in all localizations $\St(\rD_\ell, A_M[X, X\inv])$, where $M$ ranges over the maximal ideals of $A$. By the just established local case the images of $g$ in all $\St(\rD_\ell, A_M[X])$ are also trivial. Now by the local-global principle \cite[Theorem~2]{LS17} the element $g$ is trivial as well.  
 
 The proof of the fact the square of Steinberg groups is pullback is very similar   to the above argument (cf. also with~\cite[Theorem~5.1a]{Tu83}).
\end{proof}

\printbibliography
\end{document}
